\documentclass[12pt,a4paper]{article}
\usepackage{cmap} % Makes the PDF copiable. See http://tex.stackexchange.com/a/64198/25761
\usepackage[T1]{fontenc}
\usepackage[brazil]{babel}
\usepackage[utf8]{inputenc}
\usepackage{amsmath}
\usepackage{amsfonts}
\usepackage{amssymb}
\usepackage{amsthm}
\usepackage{textcomp} % \degree
\usepackage{gensymb} % \degree
\usepackage[usenames,svgnames,dvipsnames]{xcolor}
\usepackage{hyperref}
\usepackage{multicol}
\usepackage{graphicx}
\usepackage[margin=2cm]{geometry}
\usepackage{systeme}

\hypersetup{
   colorlinks = true,
   allcolors = {blue}
}

% TODO: Consider using exsheets
% http://linorg.usp.br/CTAN/macros/latex/contrib/exsheets/exsheets_en.pdf
%
% http://ctan.org/tex-archive/macros/latex/contrib/exercise/
% Options: answerdelayed,lastexercise,noanswer
\usepackage[answerdelayed,lastexercise]{exercise}

\addto\captionsbrazil{%
\def\listexercisename{Lista de exerc\'icios}%
\def\ExerciseName{Exerc\'icio}%
\def\AnswerName{Solu\c{c}\~ao do exerc\'icio}%
\def\ExerciseListName{Ex.}%
\def\AnswerListName{Solu\c{c}\~ao}%
\def\ExePartName{Parte}%
\def\ArticleOf{de\ }%
}

\renewcommand{\ExerciseHeaderTitle}{(\ExerciseTitle)\ }
\renewcommand{\ExerciseListHeader}{%\ExerciseHeaderDifficulty%
\textbf{%\ExerciseListName\
\ExerciseHeaderNB.\ %
%\ --- \
\ExerciseHeaderTitle}%
%\ExerciseHeaderOrigin
\ignorespaces}
\renewcommand{\AnswerListHeader}{\textbf{\ExerciseHeaderNB.\ (\AnswerListName)\ }}
\newcommand*\R{\mathbb{R}}

\renewcommand{\theenumi}{\alph{enumi}}
\renewcommand\labelenumi{(\theenumi) }

\newcommand*\tipo{Prova I}
\newcommand*\turma{CCI192-04U}
\newcommand*\disciplina{ANN0001}
\newcommand*\eu{Helder G. G. de Lima}
\newcommand*\data{10/04/2024}

\author{\eu}
\title{\tipo - \disciplina}
\date{\data}

\begin{document}
\thispagestyle{empty}
\newgeometry{margin=2cm,bottom=0.5cm}
\begin{center}
\includegraphics[width=9.0cm]{marca} \\
\textbf{\tipo\ (\disciplina / \turma)} \\
Prof. \eu\footnote{
Este é um material de acesso livre distribuído sob os termos da licença \href{https://creativecommons.org/licenses/by-sa/4.0/deed.pt_BR}{Creative Commons BY-SA 4.0}}
\end{center}

\noindent Nome do(a) aluno(a): \underline{\hspace{9,7cm}} Data: \underline{\data}

%\section*{Instruções}
\begin{center}\fbox{
\begin{minipage}{14cm}
\begin{footnotesize}
\begin{itemize}
\renewcommand{\theenumi}{\Roman{enumi}}
\item Identifique-se em todas as folhas.
\item Mantenha o celular e os demais equipamentos eletrônicos desligados durante a prova.
\item Justifique cada resposta com cálculos ou argumentos baseados na teoria estudada.
\item Resolva $5$ das $6$ questões (deixe claro que questão não deverá ser corrigida).
\end{itemize}
\end{footnotesize}
\end{minipage}
}
\end{center}

%\section*{Questões}
\begin{ExerciseList}
\Exercise[title={2,0}] Sabe-se que a constante de Euler $e \approx 2,7182818284$ pode ser obtida de forma exata por meio da série infinita
$e = \lim_{n\to \infty}  \sum_{k=0}^{n} \frac{1}{k!} = 1 + \frac{1}{1!} + \frac{1}{2!} + \frac{1}{3!} + \ldots + \frac{1}{n!} + \ldots$, em que $n! = n\cdot (n-1) \cdots \cdot 3\cdot 2 \cdot 1$ representa o fatorial de $n$. Considere a soma parcial $s_n = \sum_{k=0}^{n} \frac{1}{k!}$.
\begin{enumerate}
\item Qual é a menor soma $s_n$ que aproxima $e$ com erro absoluto estimado $|s_n - s_{n-1}| \leq 0,05$?
{\color{blue} \textit{(Apresente sua resposta truncando-a com 6 algarismos após a vírgula.)}}
\item Escreva em binário o termo $s_n$ encontrado no item anterior.\\
{\color{blue} \textit{(Apresente sua resposta truncando-a com 6 algarismos binários corretos após a vírgula.)}}
\end{enumerate}

\Answer
\begin{enumerate}
\item \textbf{Solução 1}: Tem-se:
\begin{center}
\begin{tabular}{llll}
\hline
  $\mathbf{n}$
& $\mathbf{s_n}$ (exato)
& $\mathbf{s_n}$ (truncado)
& $\mathbf{|s_n - s_{n-1}|}$\\ \hline
0 & $1$ & $1,000000$ & $-$ \\
1 & $1 + 1 = 2$ & $2,000000$ & $1,000000$ \\
2 & $1 + 1 + \frac{1}{2} = \frac{5}{2}$ & $2,500000$ & $0,500000$ \\
3 & $1 + 1 + \frac{1}{2} + \frac{1}{6} = \frac{8}{3}$ & $2,666666$ & $0,166666$ \\
4 & $1 + 1 + \frac{1}{2} + \frac{1}{6} + \frac{1}{24} = \frac{65}{24}$ & $\textbf{2,708333}$ & $\textbf{0,041666}$ \\ \hline
\end{tabular}
\end{center}

Logo, $s_4 = \frac{65}{24} \approx 2,708333$.

\textbf{Solução 2}: Como $|s_n - s_{n-1}|
= |\sum_{k=0}^{n} \frac{1}{k!} - \sum_{k=0}^{n-1} \frac{1}{k!}|
= \frac{1}{n!}$ e $0,05 = \frac{1}{20}$, basta determinar o menor $n$ tal que $\frac{1}{n!} \leq \frac{1}{20}$, isto é, $n! \geq 20$. Mas $1! = 1 < 20$, $2! = 2 < 20$, $3! = 6 < 20$ e $4! = 24 \geq 20$, então $n = 4$. Neste caso, tem-se $s_4 = 1+1+\frac{1}{2} + \frac{1}{6} + \frac{1}{24} = \frac{65}{24} \approx 2,708333$.

\item Tem-se $s_4 = 2,708333 = 2 + 0,708333$, onde $2 = 1 \cdot 2^1 + 0 \cdot 2^0 = (10)_{2}$. Além disso, para a parte fracionária tem-se:

\begin{center}
\begin{tabular}{|c|l|l|l|l|l|l|l|l|}
\hline
$\mathbf{s}$
& 0,708333
& 0,416666
& 0,833332
& 0,666664
& 0,333328
& 0,666656
 \\ \hline
$\mathbf{2\cdot s}$
& \textbf{1},416666
& \textbf{0},833332
& \textbf{1},666664
& \textbf{1},333328
& \textbf{0},666656
& \textbf{1},333312
\\ \hline
\end{tabular}
\end{center}

Logo,
\[
s_4
= \dfrac{65}{24}
\approx (2,708333)_{10}
\approx (10)_2 + (0,101101)_2
= (10,101101)_2.
\]
\end{enumerate}

\Exercise[title={2,0}] Estime qual deve ser o tamanho do intervalo inicial $[a_0, b_0]$ contendo um zero da função contínua $f(x)$ para que, na décima iteração do método da \textbf{bisseção}, tenha-se $|b_{10} - a_{10}| < 10^{-9}$.
\Answer Como o tamanho do intervalo é reduzido pela metade a cada iteração, tem-se:
\[
  |b_{10} - a_{10}|
  = \frac{1}{2}|b_9 - a_9|
  = \frac{1}{2^2}|b_8 - a_8|
  = \ldots
  = \frac{1}{2^9}|b_1 - a_1|
  = \frac{1}{2^{10}}|b_0 - a_0|,
\]
e em geral $|b_{k} - a_{k}| = \frac{1}{2^{k}}|b_0 - a_0|$. Então,
\[
  |b_{10}-a_{10}| < 10^{-9}
  \Leftrightarrow
  \frac{1}{2^{10}}|b_0 - a_0| < 10^{-9}
  \Leftrightarrow
  |b_0 - a_0| < 2^{10} \times 10^{-9}
  = 1,024 \times 10^{-6}
  = 0,000001024.
\]

\Exercise[title={2,0}] Seja $f(x) = \ln(x)$. Aplique o método da \textbf{posição falsa} para obter uma aproximação $x_k$ do zero da função $f(x)$ tal que $|f(x_k)| \leq 0,005$. Considere o intervalo inicial $[a_0, b_0] = [\frac{1}{2}, \frac{3}{2}]$.
{\color{blue} \textit{(Arredonde os valores obtidos com 4 algarismos após a vírgula)}}
\Answer
Os primeiros termos da sequência $(x_k)_{k=0}^\infty$, produzida pelo método da posição falsa são obtidos como segue (com arredondamento no quarto dígito decimal a cada iteração):
\medskip
\begin{center}
\begin{tabular}{rrrrrrrc}
\hline
$k$ & $a_k$ & $x_k$ & $b_k$ & $f(a_k)$ & $f(x_k)$ & $f(b_k)$ & $f(a_k)\cdot f(x_k)$ \\
\hline
0 & 0,5000 & 1,1309 & 1,5000 & -0,6931 & 0,1230 & 0,4055 & < 0 \\
1 & 0,5000 & 1,0358 & 1,1309 & -0,6931 & 0,0352 & 0,1230 & < 0 \\
2 & 0,5000 & 1,0099 & 1,0358 & -0,6931 & 0,0099 & 0,0352 & < 0 \\
3 & 0,5000 & 1,0027 & 1,0099 & -0,6931 & \textbf{0,0027} & 0,0099 & < 0 \\
\hline
\end{tabular}
\end{center}
\medskip
Como $|f(x_3)| = 0,0027 < 0,005$ (ao arredondar na quarta casa decimal), conclui-se que a aproximação $x_3 = 1,0027$ da raiz de $f$ tem a precisão desejada.

\Exercise[title={2,0}]
Sabe-se que a função $f(x) = x^2 - e^x$ possui uma única raiz $\overline{x} \in \R$. Localize um intervalo $I = (a, b)$ que contenha essa raiz, e partindo da aproximação inicial $x_0 = 1,25 \in I$, aplique o método de \textbf{Newton-Raphson} para obter um $x_k \approx \overline{x}$ tal que $\varepsilon_{rel} \approx \dfrac{|x_k - x_{k-1}|}{|x_k|} < 0,005$.
{\color{blue} \textit{(Arredonde os valores obtidos com 4 algarismos após a vírgula)}}
\Answer
Observe que
\[
  f(-1) = 1 - \frac{1}{e} \approx 0,6 > 0 > -3,4 \approx 4 - e^2 = f(2).
\]
Logo, como $f$ é uma função contínua em $[-1, 2]$ e tem sinais opostos nos extremos desse intervalo, o teorema de Bolzano garante que há uma raiz de $f$ no intervalo $I = (-1, 2)$. Partindo da aproximação inicial $x_0 = 1,25 \in I$, estas são as iterações do método de Newton-Raphson, com arredondamento no quarto dígito decimal a cada iteração:
\medskip
\begin{center}
\begin{tabular}{rrrrrrr}
\hline
  $k$
& $x_{k-1}$
& $f(x_{k-1})$
& $f^\prime(x_{k-1})$
& $\frac{f(x_{k-1})}{f^\prime(x_{k-1})}$
& $x_{k} = x_{k-1} - \frac{f(x_{k-1})}{f^\prime(x_{k-1})}$
& $\frac{|x_k - x_{k-1}|}{|x_k|}$ \\
\hline
1 &  1,2500 & -1,9278 & -0,9903 & 1,9467 & -0,6967 & 2,7942 \\
2 & -0,6967 & -0,0128 & -1,8916 & 0,0068 & -0,7035 & 0,0097 \\
3 & -0,7035 &  0,0001 & -1,9019 & -0,0001 & -0,7034 & \textbf{0,0001} \\
\hline
\end{tabular}
\end{center}
\medskip
Assim, a aproximação $x_3 \approx -0,7034$ tem um erro relativo aproximado $\frac{|x_k - x_{k-1}|}{|x_k|} < 0,005$.



\Exercise[title={2,0}]
Seja $f:\mathbb{R} \to \mathbb{R}$ definida por $f(x) = x^5-x + \frac{1}{5}$ e considere a equação $f(x)=0$.
\begin{enumerate}
\item Obtenha uma função de iteração $\varphi(x)$ para a qual seja possível mostrar, por meio de argumentos teóricos, que se $x_0 = \frac{3}{5}$ e $x_k = \varphi(x_{k-1})$, para $k \in \mathbb{N}$, então a sequência $\{x_k\}_{k \in \mathbb{N}}$ converge para algum $\overline{x}$ tal que $f(\overline{x}) = 0$. Apresente todos os detalhes desta argumentação.
\item Utilize a função de iteração obtida para obter, pelo método da \textbf{iteração de ponto fixo}, uma aproximação $x_k \approx \overline{x}$ cujo erro relativo percentual estimado por $\varepsilon_{per} \approx \dfrac{|x_k - x_{k-1}|}{|x_k|} \times 100\%$ satisfaça $|\varepsilon_{per}| \leq 1\%$.
{\color{blue} \textit{(Arredonde os valores obtidos com 4 algarismos após a vírgula)}}
\end{enumerate}
\Answer \textbf{Solução 1}:
\begin{enumerate}
\item Como $f(x) = x^5 - x + \frac{1}{5} = 0$ equivale a $x = x^5 + \frac{1}{5}$, a função $\varphi(x) = x^5 + \frac{1}{5}$ é uma função de iteração para $f$. Além disso, tem-se $\varphi^\prime(x) = 5x^4$ e as funções $\varphi$ e $\varphi^\prime$ são contínuas em $\mathbb{R}$. Considerando que
\[
|\varphi^\prime(x)| < 1
\Leftrightarrow
\left|5x^4\right| < 1
\Leftrightarrow
x^4 < \frac{1}{5}
\Leftrightarrow
x \in I = \left(-\sqrt[4]{\frac{1}{5}}, \sqrt[4]{\frac{1}{5}}\right),
\]
tem-se em particular que, para todo $x \in (-0,5, 0,5) \subset I$, vale $|\varphi^\prime(x)| < 1$. Como
\[
f(-0,5) \approx 0,66875 > 0 > -0,26875 = f(0,5),
\]
e $f$ é contínua em $[-0,5, 0,5]$, segue do teorema de Bolzano que há uma raiz de $f$ em $I$. Essa raiz pode ser obtida pelo método de iteração de ponto fixo como o limite da sequência dada por $x_k = \varphi(x_{k-1})$, para qualquer $x_0 \in I$, inclusive $x_0 = \frac{3}{5} = 0,6$.

\item Os primeiros termos dessa sequência são os seguintes (arredondados no quinto dígito decimal a cada iteração).

\begin{center}
\begin{tabular}{cccc}
\hline
$k$ & $x_k$ & $\varphi(x_k)$ & $\varepsilon_{per}(x_k)$\\
\hline
0 & 0,6000 & 0,2778 & - \\
1 & 0,2778 & 0,2017 & 115,98\% \\
2 & 0,2017 & 0,2003 &  37,73\% \\
3 & 0,2003 & - &  0,70\% \\
\hline
\end{tabular}
\end{center}
Portanto um valor aproximado de $\overline{x}$ nas condições exigidas é $x_3 = 0,2003$, que tem um erro relativo percentual aproximado inferior a $1\%$.
\end{enumerate}

\textbf{Solução 2}:

\begin{enumerate}
\item Como $f(x) = x^5 - x + \frac{1}{5} = 0$ equivale a $x = \sqrt[5]{x- \frac{1}{5}}$, a função $\varphi(x) = \sqrt[5]{x- \frac{1}{5}}$ é uma função de iteração para $f$, e $\varphi$ é contínua em $\mathbb{R}$. Além disso, tem-se $\varphi^\prime(x) = \frac{1}{5\sqrt[5]{\left(x- \frac{1}{5}\right)^4}}$ e a função $\varphi^\prime$ é contínua em $\mathbb{R}\setminus \{\frac{1}{5}\}$. Considerando que
\begin{align*}
  |\varphi^\prime(x)| < 1
  & \Leftrightarrow
  \left| \frac{1}{5\sqrt[5]{\left(x- \frac{1}{5}\right)^4}} \right| < 1
  \Leftrightarrow
  \frac{1}{5} < \sqrt[5]{\left(x- \frac{1}{5}\right)^4}, \text{ com } x \neq \frac{1}{5}\\
  & \Leftrightarrow
  \frac{1}{\sqrt[4]{5^5}} < \left|x- \frac{1}{5}\right|
  \Leftrightarrow
  x < \frac{1}{5} - \frac{1}{\sqrt[4]{5^5}}
  \text{ ou }
  x > \frac{1}{5} + \frac{1}{\sqrt[4]{5^5}} \\
  & \Leftrightarrow
  x \in \left(-\infty, 0,0663\right)
    \cup \left(0,3337, +\infty\right),
\end{align*}
tem-se em particular que, para todo $x \in I = (0,5, 1,0) \subset \left(0,3337, +\infty\right)$, vale $|\varphi^\prime(x)| < 1$. Como
\[
f(0,5) \approx -0,2688 < 0 < 0,2 = f(1,0),
\]
e $f$ é contínua em $[0,5, 1,0]$, segue do teorema de Bolzano que há uma raiz de $f$ em $I$. Essa raiz pode ser obtida pelo método de iteração de ponto fixo como o limite da sequência dada por $x_k = \varphi(x_{k-1})$, para qualquer $x_0 \in I$, inclusive $x_0 = \frac{3}{5} = 0,6$.

\item Os primeiros termos dessa sequência são os seguintes (arredondados no quinto dígito decimal a cada iteração).

\begin{center}
\begin{tabular}{cccc}
\hline
$k$ & $x_k$ & $\varphi(x_k)$ & $\varepsilon_{per}(x_k)$\\
\hline
0 & 0,6000 & 0,8326 & - \\
1 & 0,8326 & 0,9125 & 27,94\% \\
2 & 0,9125 & 0,9345 &  8,76\% \\
3 & 0,9345 & 0,9402 &  2,35\% \\
4 & 0,9402 & - &  0,61\% \\
\hline
\end{tabular}
\end{center}
Portanto um valor aproximado de $\overline{x}$ nas condições exigidas é $x_4 = 0,9402$, que tem um erro relativo percentual aproximado inferior a $1\%$.
\end{enumerate}

\Exercise[title={2,0}] A função $f(x) = -1 + \sqrt{x}$ possui um zero em $\bar{x} = 1$. Calcule uma aproximação $x_k \approx \bar{x}$ que satisfaça $|f(x_k)| \leq 0,5$ e $|x_k - x_{k-1}| \leq 0,5$ usando o método da \textbf{secante}, com aproximações iniciais $x_{-1} = 0$ e $x_0 = 2$.
{\color{blue} \textit{(Arredonde os valores obtidos com 4 algarismos após a vírgula)}}
\Answer
Os primeiros termos da sequência $(x_k)_{k=0}^\infty$, produzida pelo método da secante são obtidos como segue (com arredondamento no quarto dígito decimal a cada iteração):
\medskip
\begin{center}
\begin{tabular}{rrrrrrrc}
\hline
$k$ & $x_{k-2}$ & $x_{k-1}$ & $x_k$ & $f(x_{k-2})$ & $f(x_{k-1})$ & $f(x_k)$ & $|x_k - x_{k-1}|$ \\
\hline
1 & 0,0000 & 2,0000 & 1,4142 & -1,0000 & 0,4142 & 0,1892 & 0,5858 \\
2 & 2,0000 & 1,4142 & 0,9216 & 0,4142 & 0,1892 & \textbf{-0,0400} & \textbf{0,4926} \\
\hline
\end{tabular}
\end{center}
\medskip
Como $|f(x_2)| = 0,0400 < 0,5$ e além disso $|x_2 - x_1| = 0,4926 \leq 0,5$, conclui-se que a aproximação $x_2 = 0,9216$ da raiz de $f$ tem a precisão desejada.
\end{ExerciseList}

\vspace{0.5cm}
\begin{center}
BOA PROVA!
\end{center}

\newpage
\restoregeometry
\section*{Respostas}
\shipoutAnswer
\end{document}
