\documentclass[12pt,a4paper]{article}
\usepackage{cmap} % Makes the PDF copiable. See http://tex.stackexchange.com/a/64198/25761
\usepackage[T1]{fontenc}
\usepackage[brazil]{babel}
\usepackage[utf8]{inputenc}
\usepackage{amsmath}
\usepackage{amsfonts}
\usepackage{amssymb}
\usepackage{amsthm}
\usepackage{textcomp} % \degree
\usepackage{gensymb} % \degree
\usepackage[usenames,svgnames,dvipsnames]{xcolor}
\usepackage{hyperref}
\usepackage{multicol}
\usepackage{graphicx}
\usepackage[margin=2cm]{geometry}
\usepackage{systeme}

\hypersetup{
    colorlinks = true,
    allcolors = {blue}
}

% TODO: Consider using exsheets
% http://linorg.usp.br/CTAN/macros/latex/contrib/exsheets/exsheets_en.pdf
%
% http://ctan.org/tex-archive/macros/latex/contrib/exercise/
% Options: answerdelayed,lastexercise,noanswer
\usepackage[answerdelayed,lastexercise]{exercise}

\addto\captionsbrazil{%
\def\listexercisename{Lista de exerc\'icios}%
\def\ExerciseName{Exerc\'icio}%
\def\AnswerName{Solu\c{c}\~ao do exerc\'icio}%
\def\ExerciseListName{Ex.}%
\def\AnswerListName{Solu\c{c}\~ao}%
\def\ExePartName{Parte}%
\def\ArticleOf{de\ }%
}

\renewcommand{\ExerciseHeaderTitle}{(\ExerciseTitle)\ }
\renewcommand{\ExerciseListHeader}{%\ExerciseHeaderDifficulty%
\textbf{%\ExerciseListName\
\ExerciseHeaderNB.\ %
%\ --- \
\ExerciseHeaderTitle}%
%\ExerciseHeaderOrigin
\ignorespaces}
\renewcommand{\AnswerListHeader}{\textbf{\ExerciseHeaderNB.\ (\AnswerListName)\ }}

\newcommand*\R{\mathbb{R}}

\renewcommand{\theenumi}{\alph{enumi}}
\renewcommand\labelenumi{(\theenumi) }

\newcommand*\tipo{Prova I}
\newcommand*\turma{CCI122-03U}
\newcommand*\disciplina{ANN0001}
\newcommand*\eu{Helder G. G. de Lima}
\newcommand*\data{12/09/2017}

\author{\eu}
\title{\tipo - \disciplina}
\date{\data}

\begin{document}
\thispagestyle{empty}
\newgeometry{margin=2cm,bottom=0.5cm}
\begin{center}
\includegraphics[width=9.0cm]{marca} \\
\textbf{\tipo\ (\disciplina / \turma)} \\
Prof. \eu\footnote{
Este é um material de acesso livre distribuído sob os termos da licença \href{https://creativecommons.org/licenses/by-sa/4.0/deed.pt_BR}{Creative Commons BY-SA 4.0}}
\end{center}

\noindent Nome do(a) aluno(a): \underline{\hspace{9,7cm}} Data: \underline{\data}

%\section*{Instruções}
\begin{center}\fbox{
\begin{minipage}{14cm}

{\footnotesize
\begin{itemize}
\renewcommand{\theenumi}{\Roman{enumi}}
\item Identifique-se em todas as folhas.
\item Mantenha o celular e os demais equipamentos eletrônicos desligados durante a prova.
\item Justifique cada resposta com cálculos ou argumentos baseados na teoria estudada.
\item Resolva apenas os itens de que precisar para somar 10,0 pontos.
\end{itemize}
}

\end{minipage}
}
\end{center}

%\section*{Questões}
\begin{ExerciseList}
\Exercise[title={2,5}] Seja $\overline{x} = \dfrac{33}{20}$.
\begin{enumerate}
\item Obtenha a representação de $\overline{x}$ em binário, com 8 algarismos corretos após a vírgula.
\item Quantos desses algarismos (binários) são necessários após a vírgula para representar $\overline{x}$ com erro relativo percentual inferior a $5\%$?
\end{enumerate}
\Answer
\begin{enumerate}
\item Tem-se:
\begin{center}
\begin{tabular}{|c|l|l|l|l|l|l|l|l|}
\hline
$\mathbf{x}$
& 0,65 & 0,3 & 0,6 & 0,2 & 0,4 & 0,8 & 0,6 & 0,2 \\ \hline
$\mathbf{2\cdot x}$
& \textbf{1},3  & \textbf{0},6 & \textbf{1},2 & \textbf{0},4 & \textbf{0},8 & \textbf{1},6 & \textbf{1},2 & \textbf{0},4 \\ \hline
\end{tabular}
\end{center}

Logo,
\[
\overline{x}
= \dfrac{33}{20}
= (1,65)_{10}
= (1,10\overline{1001})_2
\approx (1,10100110)_2.
\]
\item \textbf{Solução 1}: Como se pode ver na tabela a seguir, é preciso no mínimo 3 dos algarismos da representação binária de $\overline{x}$ para que o erro seja inferior a $5\%$:
\medskip
\begin{center}
\begin{tabular}{|l|l|l|l|}
\hline
  $\mathbf{n}$
& \textbf{Binário}
& \textbf{Decimal}
& \textbf{Erro relativo percentual} \\ \hline
0 & $(1)_2$     & $(1)_{10}$ & $39,39\%$ \\ \hline
1 & $(1,1)_2$   & $(1,5)_{10}$ & $9,09\%$ \\ \hline
2 & $(1,10)_2$  & $(1,5)_{10}$ & $9,09\%$ \\ \hline
3 & $(1,101)_2$ & $(1,625)_{10}$ & $\mathbf{1,52\% < 5\%}$ \\ \hline
\end{tabular}
\end{center}
\textbf{Solução 2}: Já que $\frac{|\overline{x} - 1.65|}{|1.65|} < 5 \%$ equivale a $1.5675< \overline{x} < 1.7325$, basta truncar a soma
\begin{align*}
\overline{x} & =
  1 \times 2^0
+ 1 \times 2^{-1}
+ 0 \times 2^{-2}
+ 1 \times 2^{-3}
+ 0 \times 2^{-4}
+ 0 \times 2^{-5}
+ 1 \times 2^{-6}
+ \ldots \\
& =
  \underbrace{1
+ 0.5
+ 0
+ 0.125}_{=1.625}
+ 0
+ 0
+ 0.015625
+ \ldots
\end{align*}
assim que a soma parcial estiver no intervalo acima, isto é, na parcela $2^{-3}$.
\end{enumerate}

\Exercise[title={2,5}] Em relação ao método da \textbf{posição falsa}:
\begin{enumerate}
\item Interprete geometricamente e deduza a fórmula recursiva utilizada a cada iteração.
\item Se $f(x) = -2 + \sqrt[3]{3x}$, quantas iterações são necessárias para obter $x_k$ tal que $|f(x_k)| < 10^{-4}$, partindo do intervalo inicial $I = [2, 4]$? \textit{(utilize 5 algarismos após a vírgula nos cálculos)}
\end{enumerate}
\Answer
\begin{enumerate}
\item A cada iteração subdivide-se um intervalo fechado que contém a raiz em dois subintervalos, e escolhe-se um deles para uso na próxima iteração. O que muda em relação ao método da bissecção o ponto em que é feita a divisão: em vez de usar sempre o ponto médio dos extremos do intervalo $[a,b]$, é feita uma média ponderada dessas extremidades, levando em conta o valor de $f$ em cada uma. Geometricamente, isso corresponde a ligar os pontos $(a, f(a))$ e $(b, f(b))$ por uma reta, e encontrar a interseção desta com o eixo horizontal.
\item Os primeiros termos da sequência $(x_k)_{k=0}^\infty$, produzida pelo método da posição falsa são obtidos como segue (com arredondamento no quinto dígito decimal a cada iteração):
\begin{center}
\begin{tabular}{|r|r|r|r|r|r|r|c|}
\hline
$k$ & $a_k$ & $b_k$ & $f(a_k)$ & $f(b_k)$ & $x_k$ & $f(x_k)$ & $f(a_k)\cdot f(x_k)$ \\
\hline
0 & 2 & 4 & -0.18288 & 0.28943 & 2.77441 & 0.02658 & > 0 \\
\hline
1 & 2 & 2.77441 & -0.18288 & 0.02658 & 2.67614 & 0.00237 & > 0 \\
\hline
2 & 2 & 2.67614 & -0.18288 & 0.00237 & 2.66749 & 0.00021 & > 0 \\
\hline
3 & 2 & 2.66749 & -0.18288 & 0.00021 & 2.66672 & 0.00001 & > 0 \\
\hline
4 & 2 & 2.66672 & -0.18288 & 0.00001 & 2.66668 & \textbf{0.00000} & = 0 \\
\hline
\end{tabular}
\end{center}
\medskip
Como $|f(x_4)| = 0.00000 < 10^{-4}$ (ao arredondar na quinta casa decimal), conclui-se que $x_4 = 2.66672$ é uma aproximação da raiz de $f$ com a precisão desejada. Portanto, são necessárias 4 iterações.
\end{enumerate}

\Exercise[title={2,5}]
A função $f(x) = 2 e^x + 3x^3$ possui uma única raiz $\overline{x} \in \R$. Identifique um intervalo que contenha essa raiz, e partindo de uma aproximação inicial neste intervalo, aplique o método de \textbf{Newton-Raphson} para obter um $x_k \approx \overline{x}$ tal que $|f(x_k)| < 10^{-4}$. Ao final, estime o erro relativo percentual cometido. \textit{(utilize 5 algarismos após a vírgula nos cálculos)}
\Answer
Sendo $f$ contínua, $f(-1) = \frac{2}{e} - 3 \approx -2.3 < 0$ e $f(0) = 2 > 0$, pode-se concluir que há uma raiz de $f$ no intervalo $I = (-1, 0)$. Escolhendo como aproximação inicial da raiz o ponto $x_0 = -1/2$, estas são as iterações do método de Newton-Raphson, com arredondamento no quinto dígito decimal a cada iteração:

\begin{center}
\begin{tabular}{|r|r|r|r|r|r|}
\hline
$k$ &  $x_{k-1}$ & $f(x_{k-1})$ & $f^\prime(x_{k-1})$ & $\frac{f(x_{k-1})}{f^\prime(x_{k-1})}$ & $x_k = x_{k-1} - \frac{f(x_{k-1})}{f^\prime(x_{k-1})}$ \\
\hline
1 & -0.50000 & 0.83806 & 3.46306 & 0.242 & -0.742 \\
\hline
2 & -0.74200 & -0.27323 & 5.9074 & -0.04625 & -0.69575 \\
\hline
3 & -0.69575 & -0.01297 & 5.35401 & -0.00242 &-0.69333 \\
\hline
4 & -0.69333 & \textbf{-0.00005} & - & - & - \\
\hline
\end{tabular}
\end{center}
\medskip
Assim, a aproximação $x_4 \approx -0.69333$ satisfaz $| f(x_4) | < 10^{-4}$ e estima-se que
\[
\varepsilon_{\text{per}}
\approx
\frac{|x_4 - x_3|}{|x_4|} \times 100\%
= \frac{|-0.69333 - (-0.69575)|}{|-0.69333|} \times 100\%
= \frac{0.00242}{0.69333} \times 100\%
\approx 0.3490\%.
\]

\Exercise[title={2,5}]
Seja $f: \R \to \R$ a função definida por $f(x) = 4e^x + x^3$.
\begin{enumerate}
\item Obtenha uma função de iteração $\varphi$ para $f$, tal que se
$x_0 = -1$ for a aproximação inicial, poderá garantir a convergência do método da \textbf{iteração de ponto fixo} para a raiz de $f$.
\item Obtenha uma aproximação $x_k$ tal que o erro absoluto estimado seja $|x_k - x_{k-1}| < 0,05$, usando o item anterior. Indique os valores de $x_k$, $f(x_k)$ e $|x_k - x_{k-1}|$ a cada iteração.
\end{enumerate}
\Answer
\begin{enumerate}
\item A equação $f(x) = 4e^x + x^3 = 0$ é equivalente a
$x = -\sqrt[3]{4e^x} = -\sqrt[3]{4}e^{x/3}$. Definindo $\varphi(x) = -\sqrt[3]{4}e^{x/3}$, tem-se uma função contínua em $\R$ e $|\varphi^\prime(x)| = \frac{\sqrt[3]{4}}{3}e^{x/3}$. Então $|\varphi^\prime(x)| < 1$ equivale a
$e^{x/3} < \frac{3}{\sqrt[3]{4}}$,
ou ainda,
$x < 3 \ln\left( \frac{3}{\sqrt[3]{4}} \right) \approx 1.90954$.

Então para qualquer aproximação inicial $x_0 < 1.9$, o método da iteração de ponto fixo com esta função de iteração $\varphi$ produzirá uma sequência convergente, cujo limite será a raiz de $f$. Em particular, isso vale para $x_0 = -1$.
\item Usando a função de iteração escolhida anteriormente, os primeiros termos da sequência $(x_k)_{k=0}^\infty$, definida por $x_0 = -1$ e $x_k = -\sqrt[3]{4}e^{(x_{k-1})/3}$ para $k \geq 1$ são os seguintes (arredondados no quinto dígito decimal a cada iteração).
\begin{center}
\begin{tabular}{|r|r|r|r|}
\hline
$k$ & $x_k$ & $f(x_k)$ & $|x_k - x_{k-1}|$ \\
\hline
0 & -1.000 & 0,472 & - \\
\hline
1 & -1,137 & -0,187 & 0,137 \\
\hline
2 & -1,087 &  0,065 & 0,05 \\
\hline
3 & -1,105 & -0,024 & 0,018 \\
\hline
\end{tabular}
\end{center}

Como $|x_3 - x_2| \approx 0.018 < 0.05$, conclui-se que $x_3 = -1.105$ é uma aproximação da raiz de $f$ com a precisão desejada.
\end{enumerate}

\Exercise[title={2,5}]
Utilize o método da secante para obter uma raiz de $f(x) = 2\ln(x) - 3\ln(x - 1)$ com um erro relativo estimado de no máximo $10^{-2}$.
\Answer Atribuindo alguns valores para $x$, obtém-se:
\begin{center}
\begin{tabular}{|r|r|r|r|r|r|}
\hline
$x$    & 2   & 3   & 4    &  5 \\
\hline
$f(x)$ & 1.4 & 0.1 & -0.5 & -0.9 \\
\hline
\end{tabular}
\end{center}
Assim, no intervalo $I = [3, 4]$ deve existir uma raiz de $f$.

Estas são as primeiras iterações do método da secante com aproximações iniciais $x_{-1} = 3$ e $x_{0} = 4$, com arredondamento no quarto dígito decimal a cada iteração:

\begin{center}
\begin{tabular}{|r|r|r|r|r|r|}
\hline
$k$ & $x_{k-1}$ & $x_k$ & $x_k - x_{k-1}$ & $f(x_k)$ & $\frac{|x_k - x_{k-1}|}{|x_k|}$ \\
\hline
-1 &      - & 3.0000 & -       &  0.1178 & - \\
\hline
 0 & 3.0000 & 4.0000 & -0.8163 & -0.5233 & 0.2500 \\
\hline
 1 & 4.0000 & 3.1837 & -0.0444 & -0.0270 & 0.2564 \\
\hline
 2 & 3.1837 & 3.1393 &  0.0087 &  0.0066 & 0.0141 \\
\hline
 3 & 3.1393 & 3.1480 &  0.0001 & -0.0001 & \textbf{0.0028} \\
\hline
\end{tabular}
\end{center}
\medskip
Nesta etapa, obtém-se a aproximação $x_3 = 3.1393$, com $|x_3 - x_2| / |x_3| \approx 0.0028 < 10^{-2}$.

\Exercise[title={2,5}]
O volume $V$ de líquido em um tanque esférico de raio $r$ está relacionado com a profundidade $h$ do líquido por $V = \dfrac{\pi h^2(3r-h)}{3}$. Determine $h$, com erro absoluto menor do que $10^{-2}$, dado que $r=1\ m$ e $V = 2\ m^3$.
\Answer Substituindo os valores de $r$ e $V$, obtém-se a equação
$2 = \dfrac{\pi h^2(3-h)}{3}$, que deve ser satisfeita por algum valor $h \in [0, 2r] = [0, 2]$ (já que a altura nunca é negativa, e nem pode ultrapassar o diâmetro do tanque). Buscar uma solução para essa equação é equivalente a procurar um zero da função
\[
f(x)
= \dfrac{\pi x^2(3-x)}{3} - 2
= -\dfrac{\pi}{3}x^3 + \pi x^2 - 2.
\]
Para isso, pode ser utilizado qualquer um dos métodos estudados. Por exemplo, por Newton-Raphson, obtém-se:
\begin{center}
\begin{tabular}{|r|r|r|r|r|r|}
\hline
$k$ &  $x_{k-1}$ & $f(x_{k-1})$ & $f^\prime(x_{k-1})$ & $\frac{f(x_{k-1})}{f^\prime(x_{k-1})}$ & $x_k = x_{k-1} - \frac{f(x_{k-1})}{f^\prime(x_{k-1})}$ \\
\hline
1 & 1.0000 & 0.0944 & 3.1416 & 0.0300 & 0.9700 \\
\hline
2 & 0.9700 & 0.0002 & 3.1388 & 0.0001 & 0.9699 \\
\hline
\end{tabular}
\end{center}
Neste ponto, o erro absoluto é $\varepsilon_{abs} \approx |x_2 - x_1| = 0.0001 < 10^{-2}$.
\end{ExerciseList}

\vspace{0.5cm}
\begin{center}
BOA PROVA!
\end{center}

\newpage
\restoregeometry
\section*{Respostas}
\shipoutAnswer
\end{document}
