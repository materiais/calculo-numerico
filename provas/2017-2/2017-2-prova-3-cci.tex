\documentclass[12pt,a4paper]{article}
\usepackage{cmap} % Makes the PDF copiable. See http://tex.stackexchange.com/a/64198/25761
\usepackage[T1]{fontenc}
\usepackage[brazil]{babel}
\usepackage[utf8]{inputenc}
\usepackage{amsmath}
\usepackage{amsfonts}
\usepackage{amssymb}
\usepackage{amsthm}
\usepackage{textcomp} % \degree
\usepackage{gensymb} % \degree
\usepackage[usenames,svgnames,dvipsnames]{xcolor}
\usepackage{hyperref}
\usepackage{multicol}
\usepackage{graphicx}
\usepackage[margin=2cm]{geometry}
\usepackage{systeme}

\hypersetup{
    colorlinks = true,
    allcolors = {blue}
}

% TODO: Consider using exsheets
% http://linorg.usp.br/CTAN/macros/latex/contrib/exsheets/exsheets_en.pdf
%
% http://ctan.org/tex-archive/macros/latex/contrib/exercise/
% Options: answerdelayed,lastexercise,noanswer
\usepackage[answerdelayed,lastexercise]{exercise}

\addto\captionsbrazil{%
\def\listexercisename{Lista de exerc\'icios}%
\def\ExerciseName{Exerc\'icio}%
\def\AnswerName{Solu\c{c}\~ao do exerc\'icio}%
\def\ExerciseListName{Ex.}%
\def\AnswerListName{Solu\c{c}\~ao}%
\def\ExePartName{Parte}%
\def\ArticleOf{de\ }%
}

\renewcommand{\ExerciseHeaderTitle}{(\ExerciseTitle)\ }
\renewcommand{\ExerciseListHeader}{%\ExerciseHeaderDifficulty%
\textbf{%\ExerciseListName\
\ExerciseHeaderNB.\ %
%\ --- \ 
\ExerciseHeaderTitle}%
%\ExerciseHeaderOrigin
\ignorespaces}
\renewcommand{\AnswerListHeader}{\textbf{\ExerciseHeaderNB.\ (\AnswerListName)\ }}

\newcommand*\sen{\operatorname{sen}}

\renewcommand{\theenumi}{\alph{enumi}}
\renewcommand\labelenumi{(\theenumi) }

\newcommand*\tipo{Prova III}
\newcommand*\turma{CCI122-03U}
\newcommand*\disciplina{ANN0001}
\newcommand*\eu{Helder G. G. de Lima}
\newcommand*\data{07/12/2017}

\author{\eu}
\title{\tipo - \disciplina}
\date{\data}

\begin{document}
\thispagestyle{empty}
\newgeometry{margin=2cm,bottom=0.5cm}
\begin{center}
\includegraphics[width=9.0cm]{marca} \\
\textbf{\tipo\ (\disciplina / \turma)} \\
Prof. \eu\footnote{
Este é um material de acesso livre distribuído sob os termos da licença \href{https://creativecommons.org/licenses/by-sa/4.0/deed.pt_BR}{Creative Commons Atribuição-CompartilhaIgual 4.0 Internacional}}
\end{center}

\noindent Nome do(a) aluno(a): \underline{\hspace{9,7cm}} Data: \underline{\data}

%\section*{Instruções}
\begin{center}\fbox{
\begin{minipage}{16.3cm}

{\footnotesize
\begin{itemize}
\renewcommand{\theenumi}{\Roman{enumi}}
\item Identifique-se em todas as folhas.
\item Mantenha o celular e os demais equipamentos eletrônicos desligados durante a prova.
\item Justifique cada resposta com cálculos ou argumentos baseados na teoria estudada.
\item Utilize números decimais em vez de frações e arredonde as respostas finais com 4 casas após a vírgula.
\item Resolva apenas os itens de que precisar para somar 10,0 pontos.
\end{itemize}
}

\end{minipage}
}
\end{center}

%\section*{Questões}
\begin{ExerciseList}
\Exercise[title={1,0}] Interprete geometricamente o método de Euler explícito e relacione essa interpretação com a fórmula recursiva utilizada pelo método.
\Answer Rever referências básicas de análise numérica e cálculo numérico.

\Exercise[title={3,0}] Obtenha a reta que melhor se ajusta (por mínimos quadrados) aos seguintes pontos:
$A = (-1, 1)$,
$B = (1, 3)$,
$C = (2, 2)$,
$D = (3, 4)$,
$E = (5, 2)$.
\Answer Sejam $A = (x_1,y_1)$, $B = (x_2,y_2)$, $C = (x_3,y_3)$, $D = (x_4,y_4)$ e $E = (x_5,y_5)$ e denote $g_0(x) = 1$, $g_1(x) = x$. Para encontrar uma função da forma $f(x) = a_0 g_0(x) + a_1 g_1(x)$ que melhor se ajusta aos pontos $(x_i, y_i)$, basta resolver o sistema $A^T A X = A^T B$, em que
\[
A
= \begin{bmatrix}
g_0(x_1) & g_1(x_1) \\
\vdots & \vdots\\
g_0(x_5) & g_1(x_5) \\
\end{bmatrix}
= \begin{bmatrix}
1 & x_1 \\
\vdots & \vdots\\
1 & x_5 \\
\end{bmatrix},
%=\begin{bmatrix}
%  1 & -1 \\
%  1 &  1 \\
%  1 &  2 \\
%  1 &  3 \\
%  1 &  5
%\end{bmatrix},
\quad
X =
\begin{bmatrix}
a_0\\a_1
\end{bmatrix},
\quad
B = \begin{bmatrix}
y_1 \\
\vdots \\
y_5 \\
\end{bmatrix},
%=
%\begin{bmatrix}
%1 \\ 3 \\ 2 \\ 4 \\ 2
%\end{bmatrix},
\]
\[
A^T A
= \begin{bmatrix}
 5                & \sum_{i=1}^5 x_i \\
\sum_{i=1}^5 x_i  & \sum_{i=1}^5 x_i^2
\end{bmatrix}
=\begin{bmatrix}
 1 & 1 & 1 & 1 & 1 \\
-1 & 1 & 2 & 3 & 5
\end{bmatrix}
\cdot
\begin{bmatrix}
  1 & -1 \\
  1 &  1 \\
  1 &  2 \\
  1 &  3 \\
  1 &  5
\end{bmatrix}
=\begin{bmatrix}
5 & 10 \\ 10 & 40
\end{bmatrix},
\]
e
\[
A^T B
= \begin{bmatrix}
 \sum_{i=1}^5 y_i \\
 \sum_{i=1}^5 x_i y_i
\end{bmatrix}
= \begin{bmatrix}
 1 & 1 & 1 & 1 & 1 \\
-1 & 1 & 2 & 3 & 5
\end{bmatrix}
\begin{bmatrix}
1 \\ 3 \\ 2 \\ 4 \\ 2
\end{bmatrix}
= \begin{bmatrix}
12 \\ 28
\end{bmatrix}.
\]
Então, 
$
A^T A X = A^T B
\Leftrightarrow
\begin{bmatrix}
5 & 10 \\ 10 & 40
\end{bmatrix}
\cdot
\begin{bmatrix}
a_0\\
a_1
\end{bmatrix}
=
\begin{bmatrix}
12 \\ 28
\end{bmatrix} 
\Leftrightarrow
\begin{bmatrix}
a_0\\
a_1
\end{bmatrix}
=
\begin{bmatrix}
2\\
1/5
\end{bmatrix}.
$

Portanto, a solução é $f(x) = 2 + \dfrac{x}{5} = 2 + 0.2x$.

\Exercise[title={3,0}] Identifique qual é o melhor e o pior método para estimar $\displaystyle I = \int_{-3}^3 \cos(x) dx$ (em termos do erro relativo), dado que o valor exato é $2 \sen(3) \approx 0.2822400161$: 

\textit{(Configure sua calculadora para radianos)}
\begin{multicols}{3}
\begin{enumerate}
\item Trapézios repetido em\\3 subintervalos
\item 3/8 de Simpson em\\ um único intervalo
\item Gauss-Legendre com\\ 4 pontos
\end{enumerate}
\end{multicols}
\Answer
\begin{enumerate}
\item Repetindo a regra dos trapézios em 3 subintervalos, obtém-se:
\[
I \approx \frac{2}{2} \left( \cos(-3) + 2\cos(-1) + 2\cos(1) + \cos(3) \right)
\approx \textbf{0.1812}2
\]
\[
\varepsilon_{rel} = |0.18122 - 0.28224|/|0.28224|
\approx \textbf{0.3579}
\]

\item Aplicando a regra 3/8 de Simpson em um único intervalo, tem-se:
\[
I \approx \frac{3}{8} \cdot 2 \cdot \left( \cos(-3) + 3\cos(-1) + 3\cos(1) + \cos(3) \right)
\approx \textbf{0.9463}7
\]
\[
\varepsilon_{rel} = |0.94637 - 0.28224|/|0.28224|
\approx \textbf{2.3531}
\]

\item Fazendo a mudança de variáveis $x = 3t$ obtém-se:
$\int_{-3}^3 \cos(x) dx = 3 \int_{-1}^1 \cos(3t) dt$.
Então, pela regra de Gauss-Legendre com 4 pontos, tem-se:
\begin{align*}
I & \approx
    3\cdot \left[ 0.347855 \cdot \cos(3\cdot(-0.861136))
  + 0.652145 \cdot \cos(3\cdot(-0.339981)) \right.\\
&  \qquad\left. + 0.652145 \cdot \cos(3\cdot 0.339981)
  + 0.347855 \cdot \cos(3\cdot 0.861136)\right]\\
& \approx \textbf{0.2777}1
\end{align*}
\[
\varepsilon_{rel} = |0.27771- 0.28224|/|0.28224|
\approx \textbf{0.0161}
\]
\end{enumerate}
Portanto, a pior estimativa é a do método 3/8 de Simpson e a melhor é a do de Gauss-Legendre.

\Exercise[title={3,0}] Aplique o método de Romberg para obter uma estimativa $\displaystyle R_{4,4} \approx \int_1^9 \ln(x) dx$, e o respectivo erro relativo percentual, considerando que o valor exato da integral é $9\ln(9)-8$.
\Answer
Calculando os termos $R_{k,j}$, obtêm-se:
\begin{itemize}
\item $R_{1,1}
= \frac{8}{2}(\ln(1) + \ln(9))
= 8.788898$
\item $R_{2,1}
= \frac{4}{2}(\ln(1) + 2\ln(5) + \ln(9))
= 10.832201$
\item $R_{2,2}
= 10.832201+\dfrac{10.832201-8.788898}{3}
= 11.513302$
\item $R_{3,1}
= \frac{2}{2}(\ln(1) + 2[\ln(3) + \ln(5) + \ln(7)] + \ln(9))
= 11.505145$
\item $R_{3,2}
= 11.505145+\dfrac{11.505145 - 10.832201}{3}
= 11.729460$
\item $R_{3,3}
= 11.729460+\dfrac{11.729460-11.513302}{15}
= 11.743871$
\item $R_{4,1}
= \frac{1}{2}(\ln(1) + 2[\ln(2) + \ln(3) + \ln(4) + \ln(5) + \ln(6) + \ln(7) + \ln(8)] + \ln(9))
= 11.703215$
\item $R_{4,2}
= 11.703215+\dfrac{11.703215-11.505145}{3}
= 11.769238$
\item $R_{4,3}
= 11.769238+\dfrac{11.769238-11.729460}{15}
= 11.771890$
\item $R_{4,4}
= 11.771890+\dfrac{11.771890-11.743871}{63}
= 11.772335$
\end{itemize}

Os resultados anteriores são resumidos na tabela a seguir:
\begin{center}
\begin{tabular}{|c|c|c|c|c|r|}
\hline 
$\mathbf{k}$ & $\mathbf{ R_{k,1} }$ & $\mathbf{ R_{k,2} }$ & $\mathbf{ R_{k,3} }$ & $\mathbf{ R_{k,4} }$\\ 
\hline 
1& \textbf{8.788898} &  &  & \\ 
\hline 
2& 10.832201 & \textbf{11.513302} & & \\ 
\hline 
3& 11.505145 & 11.729460 & \textbf{11.743871} & \\ 
\hline 
4& 11.703215 & 11.769238 & 11.771890 & \textbf{11.772335} \\ 
\hline 
\end{tabular}
\end{center}
\medskip
Portanto, a aproximação $R_{4,4} = 11.772335$ tem um erro relativo percentual de $-0.0228\%$.


\Exercise[title={3,0}] 
Considere a equação diferencial $y^\prime(t) = t/y(t)$. Use os métodos de Runge-Kutta de ordem 2 e 4 para estimar $y(t)$ conforme $t$ percorre o intervalo $[0, 1.5]$ com passos de tamanho $h = 0.5$, dada a condição inicial $y(0) = 0.1$. Identifique o(s) ponto(s) onde ocorre o maior erro absoluto, levando em conta que a solução exata é $y(t)=\sqrt{t^2 + 0.01}$.
\Answer Pelo método de Runge-Kutta de ordem 2, obtêm-se:
\begin{center}
\begin{tabular}{|c|c|r|r|c|r|c|}
\hline
  $i$
& $t_i$
& $k_1$
& $k_2$
& $y_i$
& $y_{exato}(t_i)$
& $\varepsilon_i = y_i-y_{exato}(t_i)$ \\ \hline\hline
$0$ & $0.00$ &   -     &   -     & $ 0.100000$ & $ 0.100000$ & $0.000000$ \\ \hline
$1$ & $0.5$ & $0.000000$ & $5.000000$ & $ 1.350000$ & $0.509902$ & $0.840098$ \\ \hline
$2$ & $1.0$ & $0.370370$ & $0.651387$ & $ 1.605439$ & $1.004988$ & $0.600451$ \\ \hline
$3$ & $1.5$ & $0.622883$ & $0.782521$ & $ 1.956790$ & $1.503330$ & $0.453460$ \\ \hline
\end{tabular}
\end{center}
\medskip
Já pelo método de Runge-Kutta de ordem 4, obtêm-se:
\begin{center}
\begin{tabular}{|c|c|r|r|r|r|c|r|c|}
\hline
  $i$
& $t_i$
& $k_1$
& $k_2$
& $k_3$
& $k_4$
& $y_i$
& $y_{exato}(t_i)$
& {\footnotesize
$\varepsilon_i = y_i-y_{exato}(t_i)$ } \\ \hline\hline
$0$ & $0.0$ &   -     &   -     &   -     &   -     & $ 0.100000$ & $ 0.100000$ & $0.000000$ \\ \hline
$1$ & $0.5$ & $ 0.000000$ & $2.500000$ & $0.344828$ & $1.835442$ & $0.727092$ & $0.509902$ & $0.217190$ \\ \hline
$2$ & $1.0$ & $0.687671$ & $0.834251$ & $0.801578$ & $0.886618$ & $1.130921$ & $1.004988$ & $0.125933$ \\ \hline
$3$ & $1.5$ & $0.884235$ & $0.924570$ & $0.917725$ & $0.943525$ & $1.590284$ & $1.503330$ & $0.086954$ \\ \hline
\end{tabular}
\end{center}
\medskip
Em ambos os casos pode-se observar que o erro absoluto diminui conforme $t_i$ se afasta de $t_0$. Portanto, o maior erro ocorre em $t_1=0.5$.
\end{ExerciseList}

\vspace{0.4cm}
\begin{center}
BOA PROVA E BOAS FÉRIAS!
\end{center}

\newpage
\restoregeometry
\section*{Respostas}
\shipoutAnswer
\end{document}
