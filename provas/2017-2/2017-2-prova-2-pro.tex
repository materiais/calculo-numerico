\documentclass[12pt,a4paper]{article}
\usepackage{cmap} % Makes the PDF copiable. See http://tex.stackexchange.com/a/64198/25761
\usepackage[T1]{fontenc}
\usepackage[brazil]{babel}
\usepackage[utf8]{inputenc}
\usepackage{amsmath}
\usepackage{amsfonts}
\usepackage{amssymb}
\usepackage{amsthm}
\usepackage{textcomp} % \degree
\usepackage{gensymb} % \degree
\usepackage[usenames,svgnames,dvipsnames]{xcolor}
\usepackage{hyperref}
\usepackage{multicol}
\usepackage{graphicx}
\usepackage[margin=2cm]{geometry}
\usepackage{systeme}

\hypersetup{
    colorlinks = true,
    allcolors = {blue}
}

% TODO: Consider using exsheets
% http://linorg.usp.br/CTAN/macros/latex/contrib/exsheets/exsheets_en.pdf
%
% http://ctan.org/tex-archive/macros/latex/contrib/exercise/
% Options: answerdelayed,lastexercise,noanswer
\usepackage[answerdelayed,lastexercise]{exercise}

\addto\captionsbrazil{%
\def\listexercisename{Lista de exerc\'icios}%
\def\ExerciseName{Exerc\'icio}%
\def\AnswerName{Solu\c{c}\~ao do exerc\'icio}%
\def\ExerciseListName{Ex.}%
\def\AnswerListName{Solu\c{c}\~ao}%
\def\ExePartName{Parte}%
\def\ArticleOf{de\ }%
}

\renewcommand{\ExerciseHeaderTitle}{(\ExerciseTitle)\ }
\renewcommand{\ExerciseListHeader}{%\ExerciseHeaderDifficulty%
\textbf{%\ExerciseListName\
\ExerciseHeaderNB.\ %
%\ --- \
\ExerciseHeaderTitle}%
%\ExerciseHeaderOrigin
\ignorespaces}
\renewcommand{\AnswerListHeader}{\textbf{\ExerciseHeaderNB.\ (\AnswerListName)\ }}

\newcommand*\arcsen{\operatorname{arcsen}}
\newcommand*\R{\mathbb{R}}

\renewcommand{\theenumi}{\alph{enumi}}
\renewcommand\labelenumi{(\theenumi) }

\newcommand*\tipo{Prova II}
\newcommand*\turma{PRO112-04U}
\newcommand*\disciplina{CAN0001}
\newcommand*\eu{Helder G. G. de Lima}
\newcommand*\data{17/10/2017}

\author{\eu}
\title{\tipo - \disciplina}
\date{\data}

\begin{document}
\thispagestyle{empty}
\newgeometry{margin=2cm,bottom=0.5cm}
\begin{center}
\includegraphics[width=9.0cm]{marca} \\
\textbf{\tipo\ (\disciplina / \turma)} \\
Prof. \eu\footnote{
Este é um material de acesso livre distribuído sob os termos da licença \href{https://creativecommons.org/licenses/by-sa/4.0/deed.pt_BR}{Creative Commons Atribuição-CompartilhaIgual 4.0 Internacional}}
\end{center}

\noindent Nome do(a) aluno(a): \underline{\hspace{9,7cm}} Data: \underline{\data}

%\section*{Instruções}
\begin{center}\fbox{
\begin{minipage}{14cm}

{\footnotesize
\begin{itemize}
\renewcommand{\theenumi}{\Roman{enumi}}
\item Identifique-se em todas as folhas.
\item Mantenha o celular e os demais equipamentos eletrônicos desligados durante a prova.
\item Justifique cada resposta com cálculos ou argumentos baseados na teoria estudada.
\item Ao escrever números decimais, arredonde-os com 4 casas depois da vírgula.
\item Resolva apenas os itens de que precisar para somar 10,0 pontos.
\end{itemize}
}

\end{minipage}
}
\end{center}

%\section*{Questões}
\begin{ExerciseList}
\Exercise[title={2,5}]
Resolva os sistemas lineares a seguir utilizando a fatoração $A = LU$:
\begin{multicols}{2}
\begin{enumerate}
\item \systeme{
 x_1  - x_2               =  0,
       2x_2 - 2x_3        = -2,
5x_1 - 5x_2 + 3x_3 - 3x_4 = -6,
            - 6x_3 +10x_4 =  8.
}
\item \systeme{
 x_1  - x_2               =  2,
       2x_2 - 2x_3        = -6,
5x_1 - 5x_2 + 3x_3 - 3x_4 = 10,
            - 6x_3 +10x_4 =  4.
}
\end{enumerate}
\end{multicols}
\Answer Usando as operações elementares $L_3 \to L_3 - 5L_1$ e $L_4 \to L_4 + 2L_3$ obtém-se a decomposição $A = LU$, em que
\[
L=
\begin{bmatrix}
1 & 0 &  0 & 0\\
0 & 1 &  0 & 0\\
5 & 0 &  1 & 0\\
0 & 0 & -2 & 1
\end{bmatrix}
\text{ e }
U=\begin{bmatrix}
1 & -1 &  0 &  0\\
0 &  2 & -2 &  0\\
0 &  0 &  3 & -3\\
0 &  0 &  0 &  4
\end{bmatrix}.
\]

Como $A$ é a matriz de coeficientes de ambos os sistemas, pode-se usar a mesma fatoração nos dois casos. Para isso, resolve-se primeiramente um sistema $LY=B$ e com a solução $Y$ obtida resolve-se $UX = Y$. Os resultados em cada caso serão os seguintes:

\begin{enumerate}
\item
$
 Y  =
\begin{bmatrix}
0 & -2 & -6 & -4
\end{bmatrix}^T
\text{ e }
X =\begin{bmatrix}
-4 & -4 & -3 & -1
\end{bmatrix}^T
$
\item
$
 Y  =
\begin{bmatrix}
2 & -6 & 0 & 4
\end{bmatrix}^T
\text{ e }
 X  =
\begin{bmatrix}
0 & -2 & 1 & 1
\end{bmatrix}^T
$
\end{enumerate}

\Exercise[title={2,5}]
Considere um sistema cuja representação matricial é:
$
\begin{bmatrix}
 4 &  2 & -1 \\
\mathbf{k} & -4 &  0 \\
-1 &  1 &  \mathbf{k}
\end{bmatrix}
\cdot
\begin{bmatrix}
x_1 \\ x_2 \\ x_3
\end{bmatrix}
=
\begin{bmatrix}
9 \\ -5 \\ 1
\end{bmatrix}
$.
\begin{enumerate}
\item Para quais valores de $k \in \mathbb{Z}$ o sistema tem uma matriz estritamente diagonal dominante?
\item Considerando o maior valor de $k$ encontrado acima, e tomando$X^{(0)} = (1, 2, 0)$, mostre que na quarta iteração do método de Gauss-Seidel, o erro relativo percentual é inferior a $3\%$.
\end{enumerate}
\Answer
\begin{enumerate}
\item  A matriz do sistema é estritamente diagonal dominante se:
\begin{align*}
|4|  & > |2|  + |-1|,\\
|-4| & > |k|  + |0|,\\
|k|  & > |-1| + |1|,
\end{align*}
isto é, se $2 < |k| < 4$. Os únicos valores de $k \in \mathbb{Z}$ nestas condições são $k = \pm 3$.
\item Para $k=3$, o método de Gauss-Seidel usa as equações $
\begin{cases}
x_1^{(n)} = ( 9 - 2x_2^{(n-1)} + x_3^{(n-1)})/4\\
x_2^{(n)} = (-5 - 3x_1^{(n)})/(-4)\\
x_3^{(n)} = ( 1 +  x_1^{(n)} - x_2^{(n-1)})/3.
\end{cases}
$
Os valores obtidos a cada iteração são os seguintes:
\begin{center}
\begin{tabular}{|c|r|r|r|r|r|}
\hline
$\mathbf{n}$     & 0 & 1 & 2 & 3 & 4 \\
\hline
$\mathbf{x_1^{(n)}}$ & 1.0000 & 1.2500 & 1.1615 & 1.1928 & 1.1817 \\
\hline
$\mathbf{x_2^{(n)}}$ & 2.0000 & 2.1875 & 2.1211 & 2.1446 & 2.1363 \\
\hline
$\mathbf{x_3^{(n)}}$ & 0.0000 & 0.0208 & 0.0135 & 0.0161 & 0.0151 \\
\hline
\end{tabular}
\end{center}
\medskip
Na quarta iteração, o erro absoluto é estimado como
\begin{align*}
\varepsilon_{abs}
& \approx || x^{(4)} - x^{(3)} ||
= \max\{| 1.1817 - 1.1928 |,
        | 2.1363 - 2.1446 |,
        | 0.0151 - 0.0161 |\}\\
& = \max\{ |-0.0111|, |-0.0083|, |-0.0010| \}
= 0.0111.
\end{align*}
Então o erro relativo percentual é:
\[
\varepsilon_{per}
\approx
\frac{ || x^{(4)} - x^{(3)} || }{ || x^{(4)} || } \times 100 \%
= \frac{ 0.0111 }{ 2.1363 } \times 100 \%
= 0.005195 \times 100 \%
= 0.5196\% < 3\%.
\]
\end{enumerate}

\Exercise[title={2,5}]
Verifique se é garantido que o método de Gauss-Seidel convergirá, se for aplicado a
\[
\systeme{
 x_1 + 2x_2 = 4,
3x_1 + 7x_2 = 13.
}
\]
Em caso afirmativo, determine se com 4 iterações do método, partindo de $x_1^{(0)} = x_2^{(0)} = 1$, é possível obter um erro relativo $\varepsilon_{rel} \leq 0.6$.
\Answer O sistema linear pode ser escrito na forma matricial como
$
\begin{bmatrix}
1 & 2\\3 &7
\end{bmatrix}
\begin{bmatrix}
x_1\\x_2
\end{bmatrix}
=
\begin{bmatrix}
4\\13
\end{bmatrix}
$, mas sua matriz de coeficientes não é estritamente diagonal dominante, então este critério não é suficiente para determinar se as iterações do método de Gauss-Seidel produzirão uma sequência convergente a partir das equações
\[
\begin{cases}
x_1^{(k)} =  4 - 2x_2^{(k-1)}\\
x_2^{(k)} = \frac{13}{7} - \frac{3}{7} x_1^{(k)}.\\
\end{cases}
\]
A outra alternativa é reescrever essas equações matricialmente, na forma $X^{(k)} = C X^{(k-1)} + D$, pois a sequência produzida pelo método é convergente se, e somente se, cada autovalor de $C$ tem módulo menor do que um. Sendo $x_2^{(k)}
= \frac{13}{7} - \frac{3}{7}\left(4 - 2x_2^{(k-1)}\right)
= \frac{1}{7} + \frac{6}{7}x_2^{(k-1)}$, resulta que
\[
\begin{bmatrix}
x_1^{(k)}\\x_2^{(k)}
\end{bmatrix}
=
\underbrace{\begin{bmatrix}
0 & -2 \\ 0 & 6/7
\end{bmatrix}}_{C} \cdot
\begin{bmatrix}
x_1^{(k-1)}\\x_2^{(k-1)}
\end{bmatrix}
+
\begin{bmatrix}
4 \\ 1/7
\end{bmatrix}.
\]

E a matriz $C$ obtida tem a propriedade desejada:
\[
0 =
\det{(\lambda I - C)}
= \begin{vmatrix}
\lambda & 2 \\ 0 & \lambda - 6/7
\end{vmatrix}
= \lambda(\lambda - 6/7)
\Rightarrow
\begin{cases}
\lambda = 0 < 1\\
 \text{ ou }\\
\lambda = 6/7 < 1
\end{cases}
\]

Os valores obtidos a cada iteração são os seguintes:
\begin{center}
\begin{tabular}{|c|r|r|r|r|r|}
\hline
$\mathbf{k}$     & 0 & 1 & 2 & 3 & 4 \\
\hline
$\mathbf{x_1^{(k)}}$ & 1.0000 & 2.0000 & 2.0000 & 2.0000 & 2.0000 \\
\hline
$\mathbf{x_2^{(k)}}$ & 1.0000 & 1.0000 & 1.0000 & 1.0000 & 1.0000 \\
\hline
\end{tabular}
\end{center}
\medskip
Como os valores obtidos na segunda iteração correspondem à solução exata, e ela é um ponto fixo da função de iteração utilizada pelo método, as próximas iterações sempre produzem os mesmos valores, e consequentemente $\varepsilon_{rel} = 0 < 0.6$.

\Exercise[title={2,5}] Em relação à função $f(x) = \log_2(x)$: qual dos seguintes polinômios fornece o valor mais próximo de $f(3) = \log_2(3)$?
\begin{enumerate}
\item O polinômio $p(x)$, que interpola $f$ nos pontos $x_1 = 1$,  $x_2 = 2$ e  $x_3 = 4$
\item O polinômio $q(x)$, que interpola $f$ nos pontos $x_1 = 2$,  $x_2 = 4$ e  $x_3 = 8$
\end{enumerate}
\Answer
\begin{enumerate}
\item Considerando $x_1 = 1$,  $x_2 = 2$, $x_3 = 4$ e $y_1 = 0$, $y_2 = 1$ e  $y_3 = 2$, tem-se:
\begin{align*}
p(x)
& = 0 \frac{(x-2)(x-4)}{(1-2)(1-4)}
  + 1 \frac{(x-1)(x-4)}{(2-1)(2-4)}
  + 2 \frac{(x-1)(x-2)}{(4-1)(4-2)}\\
& = - \frac{(x - 1)(x - 4)}{2}
    + \frac{(x - 1)(x - 2)}{3}
  = -\frac{1}{2}\left( x^2 - 5x + 4 \right)
  +  \frac{1}{3}\left( x^2 - 3x + 2 \right)\\
& = -\frac{1}{6}x^2 + \frac{3}{2}x - \frac{4}{3}.
\end{align*}
Logo, o erro absoluto em $x = 3$ é dado por
$
\varepsilon_{abs} = |f(3) - p(3)| = |1.5850 - 5/3| = 0.0817
$.
\item Considerando $x_1 = 2$,  $x_2 = 4$, $x_3 = 8$ e $y_1 = 1$, $y_2 = 2$ e  $y_3 = 3$, tem-se:
\begin{align*}
q(x)
& = 1 \frac{(x-4)(x-8)}{(2-4)(2-8)}
  + 2 \frac{(x-2)(x-8)}{(4-2)(4-8)}
  + 3 \frac{(x-2)(x-4)}{(8-2)(8-4)}\\
& = \frac{(x - 4)(x - 8)}{12}
  - \frac{(x - 2)(x - 8) }{4}
  + \frac{(x - 2)(x - 4)}{8}\\
& = \frac{1}{12}\left( x^2 - 12x + 32 \right)
  - \frac{1}{4} \left( x^2 - 10x + 16 \right)
  + \frac{1}{8} \left( x^2 -  6x +  8 \right)
  = -\frac{1}{24}x^2 + \frac{3}{4}x - \frac{1}{3}.
\end{align*}
Logo, o erro absoluto em $x = 3$ é dado por
$
\varepsilon_{abs} = |f(3) - q(3)| = |1.5850 - 1.5417| = 0.0433
$.
\end{enumerate}
Portanto, o valor de $\log_2(3)$ está mais próximo de $q(3)$ do que de $p(3)$.

\Exercise[title={2,5}]
Utilize diferenças divididas para encontrar um polinômio $p(x)$ que interpola a função $f(x) = \arcsen{x}$ e estimar $\arcsen(\sqrt{2}/2)$, considerando a tabela a seguir:
\begin{center}
\begin{tabular}{|c|c|c|c|c|}
\hline
         $x$ & -1    & -1/2  & 0 &  1/2 \\
\hline
$\arcsen(x)$ & $-\pi/2$ & $-\pi/6$ & 0 & $\pi/6$ \\
\hline
\end{tabular}
\end{center}
\Answer A partir dos pontos dados, obtém-se:
\[
	\begin{array}{cccccc}
   x_i & y_i=f[x_i] & f[x_i,x_{i+1}] & f[x_i,x_{i+1},x_{i+2}]  & f[x_i,x_{i+1},x_{i+2},x_{i+3}] \\
   -1.0000 & \mathbf{-1.5708} \\
	    &            & \mathbf{2.0944} \\
   -0.5000 & -0.5236 &             & \mathbf{-1.0472}\\
	    &            & 1.0472 &              & \mathbf{0.6981}. \\
	0.0000 &  0.0000 &             & 0.0000\\
	    &            & 1.0472 \\
	0.5000 &  0.5236
	\end{array}
\]

Então:
\begin{align*}
p(x)
&=-1.5708
 +2.0944(x+1)
 -1.0472(x+1)(x+1/2)
 +0.6981(x+1)(x+1/2)x\\
& = 0.6981 x^3 + 0.8727 x.
\end{align*}
Usando este polinômio para estimar o valor pedido, resulta que:
\begin{align*}
\arcsen\left(0.7071\right)
& \approx p(0.7071) \\
& =-1.5708\\
& \quad +2.0944 \cdot 1.7071\\
& \quad -1.0472 \cdot 1.7071 \cdot 1.2071\\
& \quad +0.6981 \cdot 1.7071 \cdot 1.2071 \cdot 0.7071 \\
& \approx 0.8638.
\end{align*}

\textbf{Solução 2}: Se durante os cálculos não é usada a aproximação $\pi \approx 3.1415$, tem-se:
\[
	\begin{array}{cccccc}
   x_i & y_i=f[x_i] & f[x_i,x_{i+1}] & f[x_i,x_{i+1},x_{i+2}]  & f[x_i,x_{i+1},x_{i+2},x_{i+3}] \\
   -1 & \mathbf{-\pi/2} \\
	    &     & \mathbf{2\pi/3} \\
 -1/2 & -\pi/6 &             & \mathbf{-\pi/3}\\
	    &     & \pi/3 &              & \mathbf{2\pi/9}. \\
	0 & 0 &             & 0\\
	    &     & \pi/3 \\
	1/2 & \pi/6
	\end{array}
\]
Então:
\begin{align*}
p(x)
&= -\frac{\pi}{2}
 + \frac{2\pi}{3}(x+1)
 - \frac{ \pi}{3}(x+1)(x+1/2)
 + \frac{2\pi}{9}(x+1)(x+1/2)x\\
&= \frac{2 \pi}{9} x^3 + \frac{5 \pi}{18} x
\approx 0.6981 x^3 + 0.8727 x.
\end{align*}
Usando este polinômio para estimar o valor pedido, resulta que:
\[
\arcsen\left(\sqrt{2}/2\right)
\approx p\left(\sqrt{2}/2\right)
= \frac{2 \pi}{9} \left(\sqrt{2}/2\right)^3 + \frac{5 \pi}{18} \left(\sqrt{2}/2\right)
= \frac{7\pi}{18 \sqrt{2}}
\approx 0.8639.
\]
\end{ExerciseList}

\vspace{0.4cm}
\begin{center}
BOA PROVA!
\end{center}

\newpage
\restoregeometry
\section*{Respostas}
\shipoutAnswer
\end{document}
