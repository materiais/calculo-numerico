\documentclass[12pt,a4paper]{article}
\usepackage{cmap} % Makes the PDF copiable. See http://tex.stackexchange.com/a/64198/25761
\usepackage[T1]{fontenc}
\usepackage[brazil]{babel}
\usepackage[utf8]{inputenc}
\usepackage{amsmath}
\usepackage{amsfonts}
\usepackage{amssymb}
\usepackage{amsthm}
\usepackage{textcomp} % \degree
\usepackage{gensymb} % \degree
\usepackage[usenames,svgnames,dvipsnames]{xcolor}
\usepackage{hyperref}
\usepackage{multicol}
\usepackage{graphicx}
\usepackage[margin=2cm]{geometry}
\usepackage{systeme}

\hypersetup{
    colorlinks = true,
    allcolors = {blue}
}

% TODO: Consider using exsheets
% http://linorg.usp.br/CTAN/macros/latex/contrib/exsheets/exsheets_en.pdf
%
% http://ctan.org/tex-archive/macros/latex/contrib/exercise/
% Options: answerdelayed,lastexercise,noanswer
\usepackage[answerdelayed,lastexercise]{exercise}

\addto\captionsbrazil{%
\def\listexercisename{Lista de exerc\'icios}%
\def\ExerciseName{Exerc\'icio}%
\def\AnswerName{Solu\c{c}\~ao do exerc\'icio}%
\def\ExerciseListName{Ex.}%
\def\AnswerListName{Solu\c{c}\~ao}%
\def\ExePartName{Parte}%
\def\ArticleOf{de\ }%
}

\renewcommand{\ExerciseHeaderTitle}{(\ExerciseTitle)\ }
\renewcommand{\ExerciseListHeader}{%\ExerciseHeaderDifficulty%
\textbf{%\ExerciseListName\
\ExerciseHeaderNB.\ %
%\ --- \
\ExerciseHeaderTitle}%
%\ExerciseHeaderOrigin
\ignorespaces}
\renewcommand{\AnswerListHeader}{\textbf{\ExerciseHeaderNB.\ (\AnswerListName)\ }}

\newcommand*\sen{\operatorname{sen}}

\renewcommand{\theenumi}{\alph{enumi}}
\renewcommand\labelenumi{(\theenumi) }

\newcommand*\tipo{Prova III}
\newcommand*\turma{PRO112-04U}
\newcommand*\disciplina{CAN0001}
\newcommand*\eu{Helder G. G. de Lima}
\newcommand*\data{27/06/2017}

\author{\eu}
\title{\tipo - \disciplina}
\date{\data}

\begin{document}
\thispagestyle{empty}
\newgeometry{margin=2cm,bottom=0.5cm}
\begin{center}
\includegraphics[width=9.0cm]{marca} \\
\textbf{\tipo\ (\disciplina / \turma)} \\
Prof. \eu\footnote{
Este é um material de acesso livre distribuído sob os termos da licença \href{https://creativecommons.org/licenses/by-sa/4.0/deed.pt_BR}{Creative Commons BY-SA 4.0}}
\end{center}

\noindent Nome do(a) aluno(a): \underline{\hspace{9,7cm}} Data: \underline{\data}

%\section*{Instruções}
\begin{center}\fbox{
\begin{minipage}{14cm}

{\footnotesize
\begin{itemize}
\renewcommand{\theenumi}{\Roman{enumi}}
\item Identifique-se em todas as folhas.
\item Mantenha o celular e os demais equipamentos eletrônicos desligados durante a prova.
\item Justifique cada resposta com cálculos ou argumentos baseados na teoria estudada.
\item Ao escrever números decimais, arredonde-os com 4 casas depois da vírgula.
\item Resolva apenas os itens de que precisar para somar 10,0 pontos.
\end{itemize}
}

\end{minipage}
}
\end{center}

%\section*{Questões}
\begin{ExerciseList}
\Exercise[title={2,5}] Explique uma das regras de integração/quadratura numéricas, e deduza a fórmula correspondente a partir de sua interpretação geométrica.
\Answer A resposta varia conforme o método escolhido. As interpretações e deduções podem ser consultadas nas referências bibliográficas.

\Exercise[title={2,5}] Verifique qual dos métodos a seguir produz o menor erro percentual relativo ao estimar o valor da integral $I = \int_1^7 \sen\left(\sqrt{x}\right)\, dx$, cujo valor com 10 casas decimais corretas é $5.0034465813$.
\begin{enumerate}
\item Regra 1/3 de Simpson, repetida em 3 subintervalos de mesmo comprimento
\item Regra 3/8 de Simpson, repetida em 2 subintervalos de mesmo comprimento
\end{enumerate}
\Answer Aplicando os métodos indicados, obtemos:
\begin{enumerate}
\item Pela regra 1/3 de Simpson, repetida em 3 subintervalos: $I \approx 5.0019$, $\varepsilon \approx 0.0311 \%$.
\item Pela regra 3/8 de Simpson, repetida em 2 subintervalos: $I \approx 5.0006$, $\varepsilon \approx 0.0569 \%$.
\end{enumerate}
Nesta situação, o erro é menor no primeiro caso, com a regra 1/3 de Simpson.

\Exercise[title={2,5}] Considere uma função $g$ que assume os valores dados na tabela a seguir:
%g(x) = (x² - 1) ln(x / 6)
\begin{center}
\begin{tabular}{|c|c|c|c|c|c|c|c|}
\hline
   $x_i$ & 1 & 2 & 3 & 4 & 5 & 6 & 7 \\ \hline
$g(x_i)$ & 0.00 & -3.30 & -5.55 & -6.08 & -4.38 & 0.00 & 7.40 \\ \hline
\end{tabular}
\end{center}
Estime o valor de $\int_1^7 g(x)\,dx$ utilizando os métodos a seguir no maior número de subintervalos que for possível:
\begin{multicols}{2}
\begin{enumerate}
\item A regra do trapézio.
\item A regra 1/3 de Simpson.
\end{enumerate}
\end{multicols}
\Answer Usando os valores dados, obtemos:
\begin{enumerate}
\item Pela regra do trapézio, aplicada aos subintervalos $[1,2]$, $[2,3]$, $[3,4]$, $[4,5]$, $[5,6]$, $[6,7]$: $\int_1^7 g(x)\,dx \approx-15.6$.
\item Pela regra 1/3 de Simpson, repetida nos 3 subintervalos $[1,3]$, $[3,5]$, $[5,7]$: $\int_1^7 g(x)\,dx \approx -16.65$.
\end{enumerate}

\Exercise[title={2,5}] Utilize a quadratura gaussiana com 2 e 3 pontos para estimar $\int_1^2 \frac{\cos(x)}{x^2} \,dx$.
\Answer Fazendo uma mudança de variáveis, resulta que
\[
  \int_1^2 \frac{\cos(x)}{x^2} \,dx
= \frac{1}{2} \int_{-1}^1
 \frac
 { \cos\left( \frac{t + 3}{2} \right) }
 { \left(  \frac{t + 3}{2} \right)^2 }
 \,dt
= \int_{-1}^1
 \frac
 { 2 \cos\left( \frac{t + 3}{2} \right) }
 { \left( t + 3 \right)^2 }
 \,dt
= \int_{-1}^1
 2 \cos\left( \frac{t + 3}{2} \right)
 \left( t + 3 \right)^{-2}
 \,dt
\]
Consequentemente, usando a quadratura de Gauss-Legendre:
\begin{enumerate}
\item Com 2 pontos:
\begin{align*}
\int_1^2 \frac{\cos(x)}{x^2} \,dx
& \approx
 2\cos\left(  \frac{-\frac{1}{\sqrt{3}} + 3}{2}  \right)
 \left( -\frac{1}{\sqrt{3}} + 3 \right)^{-2}
+
 2\cos\left(  \frac{\frac{1}{\sqrt{3}} + 3}{2}  \right)
 \left( \frac{1}{\sqrt{3}} + 3 \right)^{-2} \\
& \approx
 0.1199 + (-0.0338) \approx 0.0861.
\end{align*}

\item Com 3 pontos:
\begin{align*}
\int_{-1}^1 f(x)
& \approx
\frac{5}{9}\cdot
 2\cos\left( \frac{-\sqrt{\frac{3}{5}} + 3}{2} \right)
 \left( -\sqrt{\frac{3}{5}} + 3 \right)^{-2}
+
\frac{8}{9}\cdot
 2\cos\left( \frac{3}{2} \right)
 \left(3 \right)^{-2} \\
& +
\frac{5}{9}\cdot
 2\cos\left( \frac{\sqrt{\frac{3}{5}} + 3}{2} \right)
 \left( \sqrt{\frac{3}{5}} + 3 \right)^{-2} \\
& \approx
\frac{5}{9} 0.1786 +
\frac{8}{9} 0.0157 +
\frac{5}{9} (-0.0437) \\
& \approx
  0.0992 +
  0.0140 +
(-0.0243)
  \approx
0.08892.
\end{align*}
\end{enumerate}

\Exercise[title={2,5}] Considere o problema de valor inicial
\[
\begin{cases}
y^\prime = x (y + 1) \\
y(0) = 1
\end{cases}
\]
Use o método de Euler para estimar os valores de $y(x)$ conforme $x$ percorre o intervalo $[a,b] = [0, 1]$, usando um passo $h = 0,2$. Faça uma tabela com os valores de $x_i$, $y_i$ e $f(x_i, y_i)$ a cada iteração $i$. Compare os resultados obtidos com a solução exata $y= 2e^{\frac{x^2}{2}}-1$, e determine o maior erro (absoluto) cometido nos pontos considerados.

\Answer Estes são os valores obtidos a cada passo:
\begin{center}
\begin{tabular}{|c|c|c|c|c|c|}
\hline
$i$ & $x_i$ & $y_i$ & $f(x_i,y_i)$ & $y_{exato}(x_i)$ & $\varepsilon_i$ \\ \hline\hline
$0$ & $0.0$ & $1.0000$ & $0.0000$ & $1.0000$ & $0.0000$ \\ \hline
$1$ & $0.2$ & $1.0000$ & $0.4000$ & $1.0404$ & $0.0404$ \\ \hline
$2$ & $0.4$ & $1.0800$ & $0.8320$ & $1.1666$ & $0.0866$ \\ \hline
$3$ & $0.6$ & $1.2464$ & $1.3478$ & $1.3944$ & $0.1480$ \\ \hline
$4$ & $0.8$ & $1.5160$ & $2.0128$ & $1.7543$ & $0.2383$ \\ \hline
$5$ & $1.0$ & $1.9185$ & $2.9185$ & $2.2974$ & $\textbf{0.3789}$ \\ \hline
\end{tabular}
\end{center}
O maior erro ocorre no ponto $x_5 = 1$, onde vale $\varepsilon_5 = 0.3789$.
\end{ExerciseList}

\vspace{0.4cm}
\begin{center}
BOA PROVA!
\end{center}

\newpage
\restoregeometry
\section*{Respostas}
\shipoutAnswer
\end{document}
