
\documentclass[12pt,a4paper]{article}
\usepackage{cmap} % Makes the PDF copiable. See http://tex.stackexchange.com/a/64198/25761
\usepackage[T1]{fontenc}
\usepackage[brazil]{babel}
\usepackage[utf8]{inputenc}
\usepackage{amsmath}
\usepackage{amsfonts}
\usepackage{amssymb}
\usepackage{amsthm}
\usepackage{textcomp} % \degree
\usepackage{gensymb} % \degree
\usepackage[usenames,svgnames,dvipsnames]{xcolor}
\usepackage{hyperref}
\usepackage{multicol}
\usepackage{graphicx}
\usepackage[margin=2cm]{geometry}
\usepackage{systeme}
\usepackage{icomma}

\hypersetup{
    colorlinks = true,
    allcolors = {blue}
}

% TODO: Consider using exsheets
% http://linorg.usp.br/CTAN/macros/latex/contrib/exsheets/exsheets_en.pdf
%
% http://ctan.org/tex-archive/macros/latex/contrib/exercise/
% Options: answerdelayed,lastexercise,noanswer
\usepackage[answerdelayed,lastexercise]{exercise}

\addto\captionsbrazil{%
\def\listexercisename{Lista de exerc\'icios}%
\def\ExerciseName{Exerc\'icio}%
\def\AnswerName{Solu\c{c}\~ao do exerc\'icio}%
\def\ExerciseListName{Ex.}%
\def\AnswerListName{Solu\c{c}\~ao}%
\def\ExePartName{Parte}%
\def\ArticleOf{de\ }%
}

\renewcommand{\ExerciseHeaderTitle}{(\ExerciseTitle)\ }
\renewcommand{\ExerciseListHeader}{%\ExerciseHeaderDifficulty%
\textbf{%\ExerciseListName\
\ExerciseHeaderNB.\ %
%\ --- \ 
\ExerciseHeaderTitle}%
%\ExerciseHeaderOrigin
\ignorespaces}
\renewcommand{\AnswerListHeader}{\textbf{\ExerciseHeaderNB.\ (\AnswerListName)\ }}

\newcommand*\R{\mathbb{R}}

\renewcommand{\theenumi}{\alph{enumi}}
\renewcommand\labelenumi{(\theenumi) }

\newcommand*\tipo{Prova II}
\newcommand*\turma{CIV122-04U}
\newcommand*\disciplina{CAN0001}
\newcommand*\eu{Helder G. G. de Lima}
\newcommand*\data{18/09/2018}

\author{\eu}
\title{\tipo - \disciplina}
\date{\data}

\begin{document}
\thispagestyle{empty}
\newgeometry{margin=2cm,bottom=0.5cm}
\begin{center}
\includegraphics[width=9.0cm]{marca} \\
\textbf{\tipo\ (\disciplina / \turma)} \\
Prof. \eu\footnote{
Este é um material de acesso livre distribuído sob os termos da licença \href{https://creativecommons.org/licenses/by-sa/4.0/deed.pt_BR}{Creative Commons BY-SA 4.0}}
\end{center}

\noindent Nome do(a) aluno(a): \underline{\hspace{9,7cm}} Data: \underline{\data}

%\section*{Instruções}
\begin{center}\fbox{
\begin{minipage}{14cm}
\begin{footnotesize}
\begin{itemize}
\renewcommand{\theenumi}{\Roman{enumi}}
\item Identifique-se em todas as folhas.
\item Mantenha o celular e os demais equipamentos eletrônicos desligados durante a prova.
\item Justifique cada resposta com cálculos ou argumentos baseados na teoria estudada.
\item Resolva apenas os itens de que precisar para somar 10,0 pontos.
\end{itemize}
\end{footnotesize}
\end{minipage}
}
\end{center}

%\section*{Questões}
\begin{ExerciseList}
\Exercise[title={2,5}]
Reorganize as equações do sistema linear a seguir (se necessário), e apresente argumentos teóricos que sejam suficientes para garantir a convergência da sequência gerada pelo método de Gauss-Seidel a partir da aproximação inicial $X^{(0)} = (0, 0, 0)$.
\[
\systeme{
 x_1-8x_2 + 2x_3 = 0,
2x_1+ x_2 + 8x_3 = 8,
8x_1- x_2        = 1
}
\]
Depois, obtenha uma solução aproximada com erro relativo percentual estimado inferior a $1\%$.\\
(arredonde cada resultado com \textbf{3 dígitos} após a vírgula)
\Answer
Para que o método de Jacobi produza uma sequência convergente, é suficiente (embora não seja necessário) que o sistema ao qual é aplicado tenha uma matriz de coeficientes estritamente diagonalmente dominante. Na ordem em que as equações foram apresentadas, a representação matricial do sistema é
\[
\begin{bmatrix}
1 & -8 & 2 \\
2 &  1 & 8 \\
8 & -1 & 0
\end{bmatrix}
\cdot
\begin{bmatrix}
x_1 \\ x_2 \\ x_3
\end{bmatrix}
=
\begin{bmatrix}
0 \\ 8 \\ 1
\end{bmatrix}.
\]
Observa-se que a matriz de coeficientes não é estritamente diagonalmente dominante, mas isso pode ser resolvido permutando as equações, de modo a obter o seguinte:
\[
\begin{bmatrix}
8 & -1 & 0 \\
1 & -8 & 2 \\
2 &  1 & 8
\end{bmatrix}
\cdot
\begin{bmatrix}
x_1 \\ x_2 \\ x_3
\end{bmatrix}
=
\begin{bmatrix}
1 \\ 0 \\ 8
\end{bmatrix}.
\]
Agora, tem-se uma matriz estritamente diagonalmente dominante, pois
\begin{align*}
| 8| & > |-1| + |0| \\
|-8| & > |1| + |2|\\
| 8| & > |2| + |1|.
\end{align*}
Então, por este critério, é garantido que o método iterativo produzirá uma sequência convergente, qualquer que seja a aproximação inicial escolhida. As equações utilizadas pelo Método de Gauss-Seidel são as seguintes:
\[
\begin{cases}
x_1^{(k)} = (1 + x_2^{(k-1)})/8\\
x_2^{(k)} = (x_1^{(k)} + 2x_3^{(k-1)})/8\\
x_3^{(k)} = (8 - 2x_1^{(k)} - x_2^{(k)})/8\\
\end{cases}
\]
Consequentemente, os valores obtidos a cada iteração são os seguintes:
\medskip
\begin{center}
\begin{tabular}{crrrrr}
\hline
$\boldsymbol{k}$     & 0 & 1 & 2 & 3 & 4\\
\hline
$\boldsymbol{x_1^{(k)}}$ & 0,000 & 0,125 & 0,127 & 0,157 & 0,157 \\
$\boldsymbol{x_2^{(k)}}$ & 0,000 & 0,016 & 0,258 & 0,254 & 0,252 \\
$\boldsymbol{x_3^{(k)}}$ & 0,000 & 0,967 & 0,936 & 0,929 & 0,929 \\
\hline
$\varepsilon_{abs}$ & - & 0,967 & 0,242 & 0,030 & 0,002 \\
\hline
$\varepsilon_{per}$ & - & 100,0\% & 25,90\% & 3,20\% & 0,20\% \\
\hline
\end{tabular}
\end{center}
\medskip
Assim, o erro relativo da aproximação $X^{(4)} = (0,157, 0,252, 0,929)$ obtida pelo método de Gauss-Seidel é de aproximadamente $0,20\%$.


\Exercise[title={2,5}]
Determine o erro absoluto ao resolver o sistema a seguir pela eliminação de Gauss:
\begin{multicols}{2}
\begin{enumerate}
\item Sem pivoteamento
\item Com pivoteamento parcial
\end{enumerate}
$\begin{cases}
0,03x+1,00y=1,00\\
0,06x+0,01y=0,00
\end{cases}$
\end{multicols}
Considere que a solução exata é $(\overline{x},\overline{y}) = (\frac{-100}{597}, \frac{200}{199})$, e arredonde o resultado de cada operação (adição, subtração, multiplicação ou divisão), com \textbf{2 dígitos} após a vírgula.
\Answer
\begin{enumerate}
\item Estes são os passos da eliminação de Gauss sem pivoteamento:
\[
\begin{bmatrix}
0,03 & 1,00 & 1,00\\
0,06 & 0,01 & 0,00
\end{bmatrix}
\rightarrow
\begin{bmatrix}
0,01 &  1,00 &  1,00\\
0,00 & -1,99 & -2,00
\end{bmatrix}.
\]
Consequentemente, $y = \frac{-2,00}{-1,99} \approx 1,01$ e
$x = \frac{1,00 - 1,00 \cdot 1,01}{0,03}
= \frac{1,00 - 1,01}{0,03}
\approx \frac{-0,01}{0,03}\approx -0,33$. Neste caso, o erro absoluto é:
\begin{align*}
\varepsilon_{abs}
& = ||(x,y) - (\overline{x},\overline{y})||
  = \max \left\{ \left|-0,33 - \frac{-100}{597}\right|, \left|1,01 - \frac{200}{199}\right| \right\}\\
& = \max \left\{ \left|-0,33 + 0,17\right|, \left|1,01 - 1,01\right| \right\} 
  = \max \left\{ \left|-0,16\right|, \left|0,00\right| \right\}
  = 0,16.
\end{align*}

\item Estes são os passos da eliminação de Gauss com pivoteamento parcial:
\[
\begin{bmatrix}
0,03 & 1,00 & 1,00\\
0,06 & 0,01 & 0,00
\end{bmatrix}
\rightarrow
\begin{bmatrix}
0,06 & 0,01 & 0,00\\
0,03 & 1,00 & 1,00
\end{bmatrix}
\rightarrow
\begin{bmatrix}
0,06 & 0,01 & 0,00\\
0,00 & 1,00 & 1,00
\end{bmatrix}
\]
Consequentemente, $y = \frac{1,00}{1,00} = 1,00$ e
$x = \frac{0,00 - 0,01 \cdot 1,00}{0,06}
\approx \frac{0,00 - 0,01}{0,06}
\approx \frac{-0,01}{0,06}
\approx -0,17$. Neste caso, o erro absoluto é:
\begin{align*}
\varepsilon_{abs}
& = ||(x,y) - (\overline{x},\overline{y})||
  = \max \left\{ \left|-0,17 - \frac{-100}{597}\right|, \left|1,00 - \frac{200}{199}\right| \right\}\\
& = \max \left\{ \left|-0,17 + 0,17\right|, \left|1,00 - 1,01\right| \right\} 
  = \max \left\{ \left|0,00\right|, \left|0,01\right| \right\}
  = 0,01.
\end{align*}
\end{enumerate}


\Exercise[title={2,5}]
Utilize a forma de Newton do polinômio interpolador para estimar o volume de chuvas em certa cidade no mês de Outubro (10), considerando o quanto choveu nos 4 meses anteriores:
\medskip
\begin{center}
\begin{tabular}{cc}
\hline
Mês & Volume de chuva (mm) \\ 
\hline
6 & 90 \\
7 & 75 \\
8 & 96 \\
9 & 126 \\
\hline
\end{tabular}
\end{center}
\Answer
A partir dos pontos dados, obtém-se:
\[
\begin{array}{ccccc}
x_i
& y_i=f[x_i]
& f[x_i,x_{i+1}]
& f[x_i,x_{i+1},x_{i+2}]
& f[x_i,\ldots,x_{i+3}]\\
6 & \mathbf{90} \\
     & & \mathbf{-15} \\
7 & 75 & & \mathbf{18} \\
     & & 21 & & \mathbf{-4,5}. \\
8 & 96 & & 4,5 \\
     & & 30 \\
9 & 126
\end{array}
\]

Então:
\[
p(x) = 90 +(-15)(x-6) + 18(x-6)(x-7) +(-4,5)(x-6)(x-7)(x-8).
\]
Usando este polinômio para estimar o valor pedido, resulta que:
\[
p(x) = 90 +(-15)(4) + 18(4)(3) +(-4,5)(4)(3)(2)
= 138.
\]

\Exercise[title={2,5}]
Obtenha, pelo método de Lagrange, o polinômio $p(x)$ que interpola $f(x) = \dfrac{3^{x}}{x}$ em $x_0 = 1$, $x_1 = 3$ e $x_2 = 4$ e calcule o erro absoluto da aproximação $f(2,5) \approx p(2,5)$.
\\(arredonde cada resultado com \textbf{4 dígitos} após a vírgula)
\Answer
Considerando que $f(1) = 3$, $f(3) = 9$ e $f(4) = \frac{81}{4} = 20,25$, o método de Lagrange permite que o polinômio que interpola $f$ nestes pontos seja descrito da seguinte forma:
\begin{align*}
p(x)
& = 3 L_0(x) + 9 L_1(x) + 20,25 L_2(x) \\
& = 3 \cdot \frac{(x-3)(x-4)}{(1-3)(1-4)}
  + 9 \cdot \frac{(x-1)(x-4)}{(3-1)(3-4)}
  + 20,25 \cdot \frac{(x-1)(x-3)}{(4-1)(4-3)}\\
& = \frac{x^2 - 7 x + 12}{2}
    - \frac{9x^2 - 45x + 36}{2}
    + \frac{27x^2 - 108x + 81}{4}
  = \frac{11}{4} x^2 - 8x + \frac{33}{4}.
\end{align*}
Logo,
\[
p(2,5) = \frac{11}{4} (2,5)^2 - 8 \cdot (2,5) + \frac{33}{4} \approx 5,4375.
\]
e o erro da aproximação $f(2,5) \approx p(2,5)$ é
\[
\varepsilon_{abs}
= \left|5,4375 - \frac{18}{5} \sqrt{3}\right|
\approx \left|5,4375 -6,2354\right|
= 0,7979.
\]



\Exercise[title={2,5}]
Considere os sistemas lineares definidos por
$
\begin{bmatrix}
\mathbf{a} & 3 \\2 & \mathbf{a}
\end{bmatrix}
\cdot
\begin{bmatrix}
x_1 \\ x_2
\end{bmatrix}
=
\begin{bmatrix}
0 \\-1
\end{bmatrix}$,
sendo $\mathbf{a} \in \R$. Escreva a relação de recorrência do método de Jacobi na forma matricial,
$X^{(k)} = C \cdot X^{(k-1)} +D$
e mostre que para todo $\mathbf{a}$ tal que $|\mathbf{a}| > \sqrt{6}$, o método sempre produzirá sequências convergentes.
\Answer
As equações utilizadas no método de Gauss-Seidel são as seguintes:
\[
\begin{cases}
x_1^{(k)} = - \frac{3}{a} x_2^{(k-1)},\\
x_2^{(k)} = -\frac{1}{a} - \frac{2}{a}x_1^{(k-1)},
\end{cases}
\]
ou seja,
\[
\begin{bmatrix}
x_1^{(k)}\\
x_2^{(k)}
\end{bmatrix}
=
\begin{bmatrix}
0 & -\frac{3}{a}\\
-\frac{2}{a} & 0
\end{bmatrix}
\cdot
\begin{bmatrix}
x_1^{(k-1)}\\
x_2^{(k-1)}
\end{bmatrix}
+
\begin{bmatrix}
0\\
-\frac{1}{a}
\end{bmatrix}
\]
As sequências produzidas pelo método de Gauss-Seidel serão convergentes se, e somente se, todos os autovalores da matriz $C = 
\begin{bmatrix}
0 & -\frac{3}{a}\\
-\frac{2}{a} & 0
\end{bmatrix}$ tiverem módulo menor do que um. Como o polinômio característico de $C$ é
\[
p_C(\lambda) = \det(\lambda I - C)
=
\begin{vmatrix}
\lambda & \frac{3}{a}\\
\frac{2}{a} & \lambda
\end{vmatrix}
= \lambda^2-\frac{6}{a^2},
\]
conclui-se que os autovalores de $C$ são $\lambda_1 = - \frac{\sqrt{6}}{a}$ e $\lambda_2 = \frac{\sqrt{6}}{a}$. Assim, tem-se a convergência se, e somente se, $\left|\frac{\sqrt{6}}{a}\right| < 1$, isto é, $|a| > \sqrt{6}$. Note que neste caso, os sistemas lineares dados são possíveis e determinados, pois $\det(A) = a^2-6 \neq 0$.
\end{ExerciseList}

\vfill
\begin{center}
BOA PROVA!
\end{center}

\newpage
\restoregeometry
\section*{Respostas}
\shipoutAnswer
\end{document}
