\documentclass[12pt,a4paper]{article}
\usepackage{cmap} % Makes the PDF copiable. See http://tex.stackexchange.com/a/64198/25761
\usepackage[T1]{fontenc}
\usepackage[brazil]{babel}
\usepackage[utf8]{inputenc}
\usepackage{amsmath}
\usepackage{amsfonts}
\usepackage{amssymb}
\usepackage{amsthm}
\usepackage{textcomp} % \degree
\usepackage{gensymb} % \degree
\usepackage[usenames,svgnames,dvipsnames]{xcolor}
\usepackage{hyperref}
\usepackage{multicol}
\usepackage{graphicx}
\usepackage[margin=2cm]{geometry}
\usepackage{systeme}
\usepackage{icomma}

\hypersetup{
    colorlinks = true,
    allcolors = {blue}
}

% TODO: Consider using exsheets
% http://linorg.usp.br/CTAN/macros/latex/contrib/exsheets/exsheets_en.pdf
%
% http://ctan.org/tex-archive/macros/latex/contrib/exercise/
% Options: answerdelayed,lastexercise,noanswer
\usepackage[answerdelayed,lastexercise]{exercise}

\addto\captionsbrazil{%
\def\listexercisename{Lista de exerc\'icios}%
\def\ExerciseName{Exerc\'icio}%
\def\AnswerName{Solu\c{c}\~ao do exerc\'icio}%
\def\ExerciseListName{Ex.}%
\def\AnswerListName{Solu\c{c}\~ao}%
\def\ExePartName{Parte}%
\def\ArticleOf{de\ }%
}

\renewcommand{\ExerciseHeaderTitle}{(\ExerciseTitle)\ }
\renewcommand{\ExerciseListHeader}{%\ExerciseHeaderDifficulty%
\textbf{%\ExerciseListName\
\ExerciseHeaderNB.\ %
%\ --- \ 
\ExerciseHeaderTitle}%
%\ExerciseHeaderOrigin
\ignorespaces}
\renewcommand{\AnswerListHeader}{\textbf{\ExerciseHeaderNB.\ (\AnswerListName)\ }}

\newcommand*\R{\mathbb{R}}

\renewcommand{\theenumi}{\alph{enumi}}
\renewcommand\labelenumi{(\theenumi) }

\newcommand*\tipo{Prova II}
\newcommand*\turma{PRO112-04U}
\newcommand*\disciplina{CAN0001}
\newcommand*\eu{Helder G. G. de Lima}
\newcommand*\data{16/10/2018}

\author{\eu}
\title{\tipo - \disciplina}
\date{\data}

\begin{document}
\thispagestyle{empty}
\newgeometry{margin=2cm,bottom=0.5cm}
\begin{center}
\includegraphics[width=9.0cm]{marca} \\
\textbf{\tipo\ (\disciplina / \turma)} \\
Prof. \eu\footnote{
Este é um material de acesso livre distribuído sob os termos da licença \href{https://creativecommons.org/licenses/by-sa/4.0/deed.pt_BR}{Creative Commons BY-SA 4.0}}
\end{center}

\noindent Nome do(a) aluno(a): \underline{\hspace{9,7cm}} Data: \underline{\data}

%\section*{Instruções}
\begin{center}\fbox{
\begin{minipage}{14cm}
\begin{footnotesize}
\begin{itemize}
\renewcommand{\theenumi}{\Roman{enumi}}
\item Identifique-se em todas as folhas.
\item Mantenha o celular e os demais equipamentos eletrônicos desligados durante a prova.
\item Justifique cada resposta com cálculos ou argumentos baseados na teoria estudada.
\item Sempre que calcular o valor de uma das funções consideradas em um ponto $x$, arredonde o resultado para o número de dígitos especificado, e só então use esse valor (arredondado) nas fórmulas dos métodos iterativos.
\item Resolva apenas os itens de que precisar para somar 10,0 pontos.
\end{itemize}
\end{footnotesize}
\end{minipage}
}
\end{center}

%\section*{Questões}
\begin{ExerciseList}
\Exercise[title={2,5}]
Reorganize as equações do sistema linear a seguir (se necessário), e apresente argumentos teóricos que sejam suficientes para garantir a convergência da sequência gerada pelo método de Jacobi a partir da aproximação inicial $X^{(0)} = (0,0,0,0)$.
\[
\begin{cases}
10x_4+x_1 = 6\\
10x_3+x_4 = 5\\
10x_2-x_3 = 4\\
10x_1-x_2 = 3.
\end{cases}
\]
Depois, obtenha uma solução aproximada com erro relativo percentual estimado inferior a $1\%$.\\
(arredonde cada resultado com \textbf{3 dígitos} após a vírgula)
\Answer
Para que o método de Jacobi produza uma sequência convergente, é suficiente (embora não seja necessário) que o sistema ao qual é aplicado tenha uma matriz de coeficientes estritamente diagonalmente dominante. Na ordem em que as equações foram apresentadas, a representação matricial do sistema é
\[
\begin{bmatrix}
 1 &  0 &  0 & 10 \\
 0 &  0 & 10 & 1 \\
 0 & 10 & -1 & 0 \\
10 & -1 &  0 & 0
\end{bmatrix}
\cdot
\begin{bmatrix}
x_1 \\ x_2 \\ x_3 \\ x_4
\end{bmatrix}
=
\begin{bmatrix}
6 \\ 5 \\ 4 \\ 3
\end{bmatrix}.
\]
Observa-se que a matriz de coeficientes não é estritamente diagonalmente dominante, mas isso pode ser resolvido permutando as equações, de modo a obter o seguinte:
\[
\begin{bmatrix}
10 & -1 &  0 & 0 \\
 0 & 10 & -1 & 0 \\
 0 &  0 & 10 & 1 \\
 1 &  0 &  0 & 10
\end{bmatrix}
\cdot
\begin{bmatrix}
x_1 \\ x_2 \\ x_3 \\ x_4
\end{bmatrix}
=
\begin{bmatrix}
3 \\ 4 \\ 5 \\ 6
\end{bmatrix}.
\]
Agora, tem-se uma matriz estritamente diagonalmente dominante, pois
\begin{align*}
|10| & > |-1| + |0| + |0|\\
|10| & > |0| + |-1| + |0|\\
|10| & > |0| + |0| + |1|\\
|10| & > |1| + |0| + |0|.
\end{align*}
Então, por este critério, é garantido que o método iterativo produzirá uma sequência convergente, qualquer que seja a aproximação inicial escolhida. As equações utilizadas pelo Método de Jacobi são as seguintes:
\[
\begin{cases}
x_1^{(k)} = (3 + x_2^{(k-1)})/10\\
x_2^{(k)} = (4 + x_3^{(k-1)})/10\\
x_3^{(k)} = (5 - x_4^{(k-1)})/10\\
x_4^{(k)} = (6 - x_1^{(k-1)})/10
\end{cases}
\]
Consequentemente, os valores obtidos a cada iteração são os seguintes:
\medskip
\begin{center}
\begin{tabular}{crrrrr}
\hline
$\boldsymbol{k}$     & 0 & 1 & 2 & 3 & 4\\
\hline
$\boldsymbol{x_1^{(k)}}$ & 0,000 & 0,300 & 0,340 & 0,345 & 0,344\\
$\boldsymbol{x_2^{(k)}}$ & 0,000 & 0,400 & 0,450 & 0,444 & 0,444 \\
$\boldsymbol{x_3^{(k)}}$ & 0,000 & 0,500 & 0,440 & 0,443 & 0,443 \\
$\boldsymbol{x_4^{(k)}}$ & 0,000 & 0,600 & 0,570 & 0,566 & 0,566 \\
\hline
$\varepsilon_{abs}$ & - & 0,600 & 0,060 & 0,006 & 0,001 \\
\hline
$\varepsilon_{per}$ & - & 100,0\% & 10,50\% & 1,10\% & 0,20\% \\
\hline
\end{tabular}
\end{center}
\medskip

Assim, o erro relativo da aproximação $X^{(4)} = (0,344, 0,444, 0,443, 0,566)$ obtida pelo método de Jacobi é de aproximadamente $0,20\%$.


\Exercise[title={2,5}]
Considerando que $X = A^{-1}$ é a solução de $A\cdot X = I_{3 \times 3}$, utilize a \textbf{fatoração LU} para calcular $A^{-1}$, sendo
$A = \begin{bmatrix}
 2&  5&  5\\
-1& -4& -8\\
 1&  3&  4
\end{bmatrix}$.
\Answer
Através das operações $L_2 \to L_2 + \frac{1}{2} L_1$, $L_3 \to L_3 -\frac{1}{2} L_1$ e $L_3 \to L_3 +\frac{1}{3} L_2$, obtém-se $A = LU$, sendo
\[
L =
\begin{bmatrix}
 1   &  0   & 0\\
-1/2 &  1   & 0\\
 1/2 & -1/3 & 1
\end{bmatrix}
\text{ e }
U =
\begin{bmatrix}
2 &   5  &   5  \\
0 & -3/2 & -11/2\\
0 &   0  &  -1/3
\end{bmatrix}
\]
Se $X = \begin{bmatrix}
x_{11} & x_{12} & x_{13}\\
x_{21} & x_{22} & x_{23}\\
x_{31} & x_{32} & x_{33}
\end{bmatrix}$,
então a equação $A\cdot X = I_{3 \times 3}$ equivale a três sistemas lineares, um para cada coluna de $X$ e de $I_{3 \times 3}$:
\begin{small}
\[
\begin{cases}
2x_{11}+5x_{12}+5x_{13} = 1,\\
-x_{11}-4x_{12}-8x_{13} = 0,\\
 x_{11}+3x_{12}+4x_{13} = 0
\end{cases}
\quad
\begin{cases}
2x_{21}+5x_{22}+5x_{23} = 0,\\
-x_{21}-4x_{22}-8x_{23} = 1,\\
 x_{21}+3x_{22}+4x_{23} = 0
\end{cases}
\quad
\begin{cases}
2x_{31}+5x_{32}+5x_{33} = 0,\\
-x_{31}-4x_{32}-8x_{33} = 0,\\
 x_{31}+3x_{32}+4x_{33} = 1
\end{cases}
\]
\end{small}
Como $A$ é a matriz de coeficientes de todos os sistemas, pode-se usar a mesma fatoração em todos os casos. Para isso, resolve-se primeiramente um sistema $LY=B$ e com a solução $Y$ obtida resolve-se $UX = Y$. Os resultados em cada caso serão os seguintes:
\begin{multicols}{3}
\begin{footnotesize}
\begin{enumerate}
\item
$
Y = \begin{bmatrix}
1 \\ 1/2 \\ -1/3
\end{bmatrix}
\text{ e }
X = \begin{bmatrix}
8 \\ -4 \\ 1
\end{bmatrix}
$
\item
$
Y  =
\begin{bmatrix}
0 \\ 1 \\ 1/3
\end{bmatrix}
\text{ e }
 X  =
\begin{bmatrix}
-5 \\ 3 \\ -1
\end{bmatrix}$
\item
$
Y  =
\begin{bmatrix}
0 \\ 0 \\ 1
\end{bmatrix}
\text{ e }
 X  =
\begin{bmatrix}
-20 \\ 11 \\-3
\end{bmatrix}$
\end{enumerate}
\end{footnotesize}
\end{multicols}
Portanto, $X = A^{-1} = 
\begin{bmatrix}
 8 & -5 & -20\\
-4 &  3 &  11\\
 1 & -1 &  -3
\end{bmatrix}$.


\Exercise[title={2,5}]
A tabela a seguir mostra a evolução do salário mínimo nos últimos anos. Utilize a forma de Newton do polinômio interpolador para estimar o salário mínimo em 2019, arredondado com \textbf{2 dígitos} após a vírgula.
\medskip
\begin{center}
\begin{tabular}{cc}
\hline
Ano & Salário mínimo (R\$) \\ 
\hline
2015 & 788,00 \\
2016 & 880,00 \\
2017 & 937,00 \\
2018 & 954,00 \\
\hline
\end{tabular}
\end{center}
\Answer
A partir dos pontos dados, obtém-se:
\[
\begin{array}{ccccc}
x_i
& y_i=f[x_i]
& f[x_i,x_{i+1}]
& f[x_i,x_{i+1},x_{i+2}]
& f[x_i,\ldots,x_{i+3}]\\
2015 & \mathbf{788,00} \\
     & & \mathbf{92,00} \\
2016 & 880,00 & & \mathbf{-17,50} \\
     & & 57,00 & & \mathbf{-0,83}. \\
2017 & 937,00 & & -20,00 \\
     & & 17,00 \\
2018 & 954,00
\end{array}
\]

Então:
\begin{align*}
p(x)
&= 788,00
 +  92,00  (x-2015)
 +(-17,50) (x-2015)(x-2016)\\
&\quad -0,83 (x-2015)(x-2016)(x-2017).
\end{align*}
Usando este polinômio para estimar o valor pedido, resulta que:
\begin{align*}
p(2019)
&= 788,00
 +  92,00  \cdot (4)
 +(-17,50) \cdot (4)\cdot (3)\\
&\quad -0,83 \cdot (4)\cdot (3)\cdot (2)\\
& = 926.
\end{align*}

\Exercise[title={2,5}]
Obtenha, pelo método de Lagrange, o polinômio $p(x)$ que interpola $f(x) = x \cdot 2^{-x}$ em $x_0 = 0$, $x_1 = 1$ e $x_2 = 3$ e calcule o erro absoluto da aproximação $f(2) \approx p(2)$.
\Answer
Considerando que $f(0) = 0$, $f(1) = \frac{1}{2} = 0,5$ e $f(3) = \frac{3}{8} = 0,375$, o método de Lagrange permite que o polinômio que interpola $f$ nestes pontos seja dado na seguinte forma:
\begin{align*}
p(x)
& = 0 L_0(x) + \frac{1}{2}L_1(x) + \frac{3}{8} L_2(x) \\
& = 0 \cdot \frac{(x-1)(x-3)}{(0-1)(0-3)}
  + \frac{1}{2} \cdot \frac{(x-0)(x-3)}{(1-0)(1-3)}
  + \frac{3}{8} \cdot \frac{(x-0)(x-1)}{(3-0)(3-1)}\\
& = \frac{-x^2 + 3x}{4}
    + \frac{x^2 - x}{16}
  = \frac{-3}{16} x^2 + \frac{11}{16} x
  = -0,1875 x^2 + 0,6875 x.
\end{align*}
Logo,
\[
p(2) = \frac{-3}{16}\cdot 2^2 + \frac{11}{16} \cdot 2 = \frac{5}{8} \approx 0,625.
\]
e o erro da aproximação $f(2) \approx p(2)$ é $\varepsilon_{abs}
= \left| \frac{1}{2} - \frac{5}{8}\right|
= \left|-\frac{1}{8}\right|
= \frac{1}{8}
\approx \left|0,5 - 0,625\right|
\approx 0,125$.


\Exercise[title={2,5}]
Considere os sistemas lineares definidos por
$
\begin{bmatrix}
\mathbf{a} & 2\\3 & 4
\end{bmatrix}
\cdot
\begin{bmatrix}
x_1 \\ x_2
\end{bmatrix}
=
\begin{bmatrix}
5 \\6
\end{bmatrix}$,
sendo $\mathbf{a} \in \R$. Escreva a relação de recorrência do método de Gauss-Seidel na forma matricial,
$X^{(k)} = C \cdot X^{(k-1)} +D$
e mostre que para todo $a$ tal que $|\mathbf{a}| > 3/2$, o método sempre produzirá sequências convergentes.
\Answer

As equações utilizadas no método de Gauss-Seidel são as seguintes:
\[
\begin{cases}
x_1^{(k)} = \frac{5}{a} - \frac{2}{a} x_2^{(k-1)}\\
x_2^{(k)} = \frac{6}{4} - \frac{3}{4}x_1^{(k)}
\end{cases}
\]
Substituindo a primeira equação na segunda, resulta que
\[
x_2^{(k)}
= \frac{6}{4} - \frac{3}{4}\left(\frac{5}{a} - \frac{2}{a} x_2^{(k-1)}\right)
= \frac{6a-15}{4a} + \frac{6}{4a} x_2^{(k-1)},
\]
ou seja,
\[
\begin{bmatrix}
x_1^{(k)}\\
x_2^{(k)}
\end{bmatrix}
=
\begin{bmatrix}
0 & -\frac{2}{a}\\
0 & \frac{3}{2a}
\end{bmatrix}
\cdot
\begin{bmatrix}
x_1^{(k-1)}\\
x_2^{(k-1)}
\end{bmatrix}
+
\begin{bmatrix}
\frac{5}{a}\\
\frac{6a-15}{4a}
\end{bmatrix}
\]
As sequências produzidas pelo método de Gauss-Seidel serão convergentes se, e somente se, todos os autovalores da matriz $C = 
\begin{bmatrix}
0 & -\frac{2}{a}\\
0 & \frac{3}{2a}
\end{bmatrix}$ tiverem módulo menor do que um. Como o polinômio característico de $C$ é
\[
p_C(\lambda) = \det(\lambda I - C)
=
\begin{vmatrix}
\lambda & -\frac{2}{a}\\
0 & \lambda-\frac{3}{2a}
\end{vmatrix}
= \lambda\left(\lambda-\frac{3}{2a}\right),
\]
conclui-se que os autovalores de $C$ são $\lambda_1 = 0$ e $\lambda_2 = \frac{3}{2a}$. Assim, tem-se a convergência se, e somente se, $\left|\frac{3}{2a}\right| < 1$, isto é, $|a| > \frac{3}{2}$. Note que neste caso, os sistemas lineares dados são possíveis e determinados, pois $\det(A) = 4a-6 \neq 0$.
\end{ExerciseList}

\vfill
\begin{center}
BOA PROVA!
\end{center}

\newpage
\restoregeometry
\section*{Respostas}
\shipoutAnswer
\end{document}
