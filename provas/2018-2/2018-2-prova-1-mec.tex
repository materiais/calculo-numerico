\documentclass[12pt,a4paper]{article}
\usepackage{cmap} % Makes the PDF copiable. See http://tex.stackexchange.com/a/64198/25761
\usepackage[T1]{fontenc}
\usepackage[brazil]{babel}
\usepackage[utf8]{inputenc}
\usepackage{amsmath}
\usepackage{amsfonts}
\usepackage{amssymb}
\usepackage{amsthm}
\usepackage{textcomp} % \degree
\usepackage{gensymb} % \degree
\usepackage[usenames,svgnames,dvipsnames]{xcolor}
\usepackage{hyperref}
\usepackage{multicol}
\usepackage{graphicx}
\usepackage[margin=2cm]{geometry}
\usepackage{systeme}
\usepackage{icomma}

\hypersetup{
    colorlinks = true,
    allcolors = {blue}
}

% TODO: Consider using exsheets
% http://linorg.usp.br/CTAN/macros/latex/contrib/exsheets/exsheets_en.pdf
%
% http://ctan.org/tex-archive/macros/latex/contrib/exercise/
% Options: answerdelayed,lastexercise,noanswer
\usepackage[answerdelayed,lastexercise]{exercise}

\addto\captionsbrazil{%
\def\listexercisename{Lista de exerc\'icios}%
\def\ExerciseName{Exerc\'icio}%
\def\AnswerName{Solu\c{c}\~ao do exerc\'icio}%
\def\ExerciseListName{Ex.}%
\def\AnswerListName{Solu\c{c}\~ao}%
\def\ExePartName{Parte}%
\def\ArticleOf{de\ }%
}

\renewcommand{\ExerciseHeaderTitle}{(\ExerciseTitle)\ }
\renewcommand{\ExerciseListHeader}{%\ExerciseHeaderDifficulty%
\textbf{%\ExerciseListName\
\ExerciseHeaderNB.\ %
%\ --- \
\ExerciseHeaderTitle}%
%\ExerciseHeaderOrigin
\ignorespaces}
\renewcommand{\AnswerListHeader}{\textbf{\ExerciseHeaderNB.\ (\AnswerListName)\ }}

\newcommand*\sen{\operatorname{sen}}
\newcommand*\tg{\operatorname{tg}}

\renewcommand{\theenumi}{\alph{enumi}}
\renewcommand\labelenumi{(\theenumi) }

\newcommand*\tipo{Prova I}
\newcommand*\turma{MEC151-03U}
\newcommand*\disciplina{CAN0001}
\newcommand*\eu{Helder G. G. de Lima}
\newcommand*\data{12/09/2018}

\author{\eu}
\title{\tipo - \disciplina}
\date{\data}

\begin{document}
\thispagestyle{empty}
\newgeometry{margin=2cm,bottom=0.5cm}
\begin{center}
\includegraphics[width=9.0cm]{marca} \\
\textbf{\tipo\ (\disciplina / \turma)} \\
Prof. \eu\footnote{
Este é um material de acesso livre distribuído sob os termos da licença \href{https://creativecommons.org/licenses/by-sa/4.0/deed.pt_BR}{Creative Commons BY-SA 4.0}}
\end{center}

\noindent Nome do(a) aluno(a): \underline{\hspace{9,7cm}} Data: \underline{\data}

%\section*{Instruções}
\begin{center}\fbox{
\begin{minipage}{14cm}

\begin{footnotesize}
\begin{itemize}
\renewcommand{\theenumi}{\Roman{enumi}}
\item Identifique-se em todas as folhas.
\item Mantenha o celular e os demais equipamentos eletrônicos desligados durante a prova.
\item Justifique cada resposta com cálculos ou argumentos baseados na teoria estudada.
\item Sempre que calcular o valor de uma das funções consideradas em um ponto $x$, arredonde o resultado para o número de dígitos especificado, e só então use esse valor (arredondado) nas fórmulas dos métodos iterativos.
\item Resolva apenas os itens de que precisar para somar 10,0 pontos.
\end{itemize}
\end{footnotesize}

\end{minipage}
}
\end{center}

%\section*{Questões}
\begin{ExerciseList}
\Exercise[title={2,5}] Sejam
$\overline{a} = 0,46$,
$\overline{b} = 2,14$ e
$\overline{c} = 0,65$ e considere as aproximações $a\approx \overline{a}$,
$b\approx \overline{b}$ e
$c\approx \overline{c}$,
com apenas um dígito após a vírgula, obtidas por arredondamento (conforme a ABNT). Verifique se é verdade que $a\cdot (b + c) = a\cdot b + a\cdot c$ caso \textbf{todas} as adições e multiplicações intermediárias sejam efetuadas com valores arredondados, e indique o erro relativo percentual nos resultados finais de $a\cdot (b + c)$ e também de $a\cdot b + a\cdot c$.
\Answer
Considerando
$a = 0,5$,
$b = 2,1$ e
$c = 0,6$, tem-se:
\[
a\cdot (b + c)
= 0,5 \cdot (2,1 + 0,6)
= 0,5 \cdot 2,7
= 1,35 \approx 1,4.
\]
\[
a\cdot b + a\cdot c
= 0,5\cdot 2,1 + 0,5\cdot 0,6
= 1,05 + 0,3
\approx 1,0 + 0,3
= 1,3.
\]
Portanto, $a\cdot (b + c) \neq a\cdot b + a\cdot c$. Comparando os resultados acima com o valor exato, que é $\overline{a}\cdot (\overline{b} + \overline{c}) = 0,46\cdot (2,14 + 0,65) = 1,2834$, percebe-se que os erros relativos percentuais dos resultados obtidos são:
\[
|\varepsilon_{per}(a\cdot (b + c))|
= \frac{|1,4-1,2834|}{|1,2834|} \times 100 \%
= 9,0852 \%
= 9,1 \%.
\]
\[
|\varepsilon_{per}(a\cdot b + a\cdot c)|
= \frac{|1,3-1,2834|}{|1,2834|} \times 100 \%
= 1,2934 \%
= 1,3 \%.
\]


\Exercise[title={2,5}] Mostre que toda função da forma $f(x) = m(2-x)-x^3$, para algum $m > 0$, zera em algum ponto $\overline{x} \in (0, 2)$. Calcule uma aproximação $x_k \approx \overline{x}$ pelo método da bisseção no caso em que $m = 9$, de modo que o erro absoluto aproximado seja $|\varepsilon_{abs}(x_k)| \leq 0,05$.

\textit{(Arredonde os valores utilizados no cálculo de cada $x_k$ com 4 dígitos após a vírgula)}
\Answer
Considerando que, para todo $m > 0$, tem-se $f(0) = 2m > 0 > -8 = f(2)$. Como $f$ muda de sinal em $[0,2]$ e $f$ é contínua neste intervalo, segue do teorema de Bolzano que $f$ possui pelo menos uma raiz em $(0,2)$, isto é, existe $\overline{x} \in (0,2)$ tal que $f(\overline{x}) = 0$.

Pelo método da bisseção, obtêm-se as seguintes aproximações para $\overline{x}$, quando $m = 9$:
\medskip
\begin{center}
\begin{tabular}{cccccccc}
\hline
$k$ & $a_k$ & $x_k$ & $b_k$ & $f(a_k)$ & $f(x_k)$ & $f(b_k)$ & $\varepsilon_{abs}(x_k)$\\\hline
0 & 0,0000 & 1,0000 & 2,0000 & 18,0000 & 8,0000 & -8,0000 & -\\
1 & 1,0000 & 1,5000 & 2,0000 & 8,0000 & 1,1250 & -8,0000 & 0,5000\\
2 & 1,5000 & 1,7500 & 2,0000 & 1,1250 & -3,1094 & -8,0000 & 0,2500\\
3 & 1,5000 & 1,6250 & 1,7500 & 1,1250 & -0,9160 & -3,1094 & 0,1250\\
4 & 1,5000 & 1,5625 & 1,6250 & 1,1250 & 0,1228 & -0,9160 & 0,0625\\
5 & 1,5625 & 1,5938 & 1,6250 & 0,1228 & -0,3928 & -0,9160 & 0,0313\\
\hline
\end{tabular}
\end{center}
\medskip
Portanto, $x_5 = 1,5938 \approx \overline{x}$, com um erro absoluto dentro da tolerância estipulada.

\Exercise[title={2,5}]
Obtenha uma aproximação $x_k$ de uma raiz da função $f(x) = 1/8-\cos(x)$ tal que $|f(x_k)|<10^{-9}$, utilizando o método de Newton-Raphson com $x_0 = 4$.

\textit{(Configure a calculadora em radianos e arredonde cada valor com 10 dígitos após a vírgula)}
\Answer
Considerando que, por exemplo, $f(3) \approx 1,11 > 0 > -0,16 \approx f(5)$ e que $f$ é uma função contínua, resulta do teorema de Bolzano que $f$ possui alguma raiz $\overline{x} \in I = (3, 5)$. Além disso, $f^\prime(x) = \sen(x)$ e $f^{\prime\prime}(x) = \cos(x)$ são contínuas em $I$, e sendo $f^\prime(x) \neq 0$ para $x \in I$, o método de Newton-Raphson produzirá uma sequência convergente, para toda aproximação inicial $x_0$ em algum subintervalo $\overline{I} \subset (3, 5)$. Usando o valor que foi proposto, $x_0 = 4 \in I$, e aplicando o método de Newton-Raphson, obtém-se os seguintes valores:
\medskip
\begin{center}
\begin{tabular}{cccc}
\hline
$k$ & $x_k$ & $f(x_k)$ & $f^\prime(x_k)$ \\\hline
0 & 4,000000000 &  0,778643621 & -0,756802495\\
1 & 5,028859744 & -0,186214532 & -0,950339684\\
2 & 4,832914510 &  0,004766059 & -0,992745586\\
3 & 4,837715397 &  0,000001403 & -0,992156918\\
4 & 4,837716811 &  0,000000001 & -\\\hline
\end{tabular}
\end{center}
\medskip
Portanto, $x_4 = 4,837716811$ é o valor aproximado de uma raiz de $f$, e $|f(x_4)| < 10^{-7}$.


\Exercise[title={2,5}]
Seja $f(x) = \tg(x)$. Aplique o método da secante, partindo das aproximações iniciais $x_{-1} = 8$ e $x_0 = 10$, para obter $x_k$ com erro relativo percentual menor ou igual a $0,05\%$.

\textit{(Configure a calculadora em radianos e arredonde cada $x_k$ e $f(x_k)$ com 4 dígitos após a vírgula)}
\Answer
Considerando que $f(x) = \tg(x)$ é contínua em $[8, 10]$, que $f(8) \approx -6,7997$ e que $f(10) \approx 0,6484$, segue do teorema de Bolzano que há uma raiz $\overline{x}$ de $f$ em $(8, 10)$. Se o método da secante for aplicado para aproximá-la, resultarão os seguintes valores:
\medskip
\begin{center}
\begin{tabular}{cccccccc}
\hline
$k$ & $x_{k-2}$ & $x_{k-1}$ & $x_k$ & $f(x_{k-2})$ & $f(x_{k-1})$ & $f(x_k)$ & $|x_k-x_{k-1}|/|x_k|$ \\
\hline
0 & 8,0000 & 10,0000 & 9,8259 & -6,7997 & 0,6484 & 0,4241 & - \\
1 & 10,0000 & 9,8259 & 9,4967 & 0,6484 & 0,4241 & 0,0720 & 3,4665\%\\
2 & 9,8259 & 9,4967 & 9,4294 & 0,4241 & 0,0720 & 0,0046 & 0,7137\%\\
3 & 9,4967 & 9,4294 & 9,4248 & 0,0720 & 0,0046 & 0,0000 & 0,0488\%\\
\hline
\end{tabular}
\end{center}
\medskip
Assim, a raiz $\overline{x}$ de $f$ é aproximadamente $x_3 = 9,4248$, com $\varepsilon_{per}(x_k) \approx 0,0488\%$.


\Exercise[title={2,5}]
Mostre que a sequência $\{x_n\}_{n=0}^\infty$ definida por $x_n = 3/n^3$, para $n \geq 1$, converge \textit{linearmente} para $\overline{x} = 0$. Qual é o menor valor de $n \in \mathbb{N}$ tal que $|x_n - \overline{x}| \leq 10^{-4}$?
\Answer A convergência para zero é linear pois
\[
\lim_{n \to \infty} \frac{|x_n - \overline{x}|}{|x_{n-1} - \overline{x}|^\alpha}
= \lim_{n \to \infty} \frac{|3/n^3 - 0|}{|3/(n-1)^3 - 0|^1}
= \lim_{n \to \infty} \frac{(n-1)^3}{n^3}
= \lim_{n \to \infty} \frac{n^3-3n^2+3n+1}{n^3}
= 1.
\]
Para determinar p menor $n$ tal que $|x_n - \overline{x}| \leq 10^{-4}$, basta observar o seguinte:
\[
|3/n^3 - 0| \leq 10^{-4}
\Leftrightarrow
3 \leq 10^{-4} n^3
\Leftrightarrow
3/10^{-4} \leq n^3
\Leftrightarrow
n \geq \sqrt[3]{3/10^{-4}} \approx 31,07.
\]
Portanto, $n = 32$.
\end{ExerciseList}

\vfill
\begin{center}
BOA PROVA!
\end{center}

\newpage
\restoregeometry
\section*{Respostas}
\shipoutAnswer
\end{document}
