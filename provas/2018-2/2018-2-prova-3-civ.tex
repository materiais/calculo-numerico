\documentclass[12pt,a4paper]{article}
\usepackage{cmap} % Makes the PDF copiable. See http://tex.stackexchange.com/a/64198/25761
\usepackage[T1]{fontenc}
\usepackage[brazil]{babel}
\usepackage[utf8]{inputenc}
\usepackage{amsmath}
\usepackage{amsfonts}
\usepackage{amssymb}
\usepackage{amsthm}
\usepackage{textcomp} % \degree
\usepackage{gensymb} % \degree
\usepackage[usenames,svgnames,dvipsnames]{xcolor}
\usepackage{hyperref}
\usepackage{multicol}
\usepackage{graphicx}
\usepackage[margin=2cm]{geometry}
\usepackage{systeme}
\usepackage{icomma}
\usepackage{multicol}

\hypersetup{
    colorlinks = true,
    allcolors = {blue}
}

% TODO: Consider using exsheets
% http://linorg.usp.br/CTAN/macros/latex/contrib/exsheets/exsheets_en.pdf
%
% http://ctan.org/tex-archive/macros/latex/contrib/exercise/
% Options: answerdelayed,lastexercise,noanswer
\usepackage[answerdelayed,lastexercise]{exercise}

\addto\captionsbrazil{%
\def\listexercisename{Lista de exerc\'icios}%
\def\ExerciseName{Exerc\'icio}%
\def\AnswerName{Solu\c{c}\~ao do exerc\'icio}%
\def\ExerciseListName{Ex.}%
\def\AnswerListName{Solu\c{c}\~ao}%
\def\ExePartName{Parte}%
\def\ArticleOf{de\ }%
}

\renewcommand{\ExerciseHeaderTitle}{(\ExerciseTitle)\ }
\renewcommand{\ExerciseListHeader}{%\ExerciseHeaderDifficulty%
\textbf{%\ExerciseListName\
\ExerciseHeaderNB.\ %
%\ --- \ 
\ExerciseHeaderTitle}%
%\ExerciseHeaderOrigin
\ignorespaces}
\renewcommand{\AnswerListHeader}{\textbf{\ExerciseHeaderNB.\ (\AnswerListName)\ }}

\renewcommand{\theenumi}{\alph{enumi}}
\renewcommand\labelenumi{(\theenumi) }

\newcommand*\tipo{Prova III}
\newcommand*\turma{CIV122-04U}
\newcommand*\disciplina{CAN0001}
\newcommand*\eu{Helder G. G. de Lima}
\newcommand*\data{01/12/2018}

\author{\eu}
\title{\tipo - \disciplina}
\date{\data}

\begin{document}
\thispagestyle{empty}
\newgeometry{margin=2cm,bottom=0.5cm}
\begin{center}
\includegraphics[width=9.0cm]{marca} \\
\textbf{\tipo\ (\disciplina / \turma)} \\
Prof. \eu\footnote{
Este é um material de acesso livre distribuído sob os termos da licença \href{https://creativecommons.org/licenses/by-sa/4.0/deed.pt_BR}{Creative Commons BY-SA 4.0}}
\end{center}

\noindent Nome do(a) aluno(a): \underline{\hspace{9,7cm}} Data: \underline{\data}

%\section*{Instruções}
\begin{center}\fbox{
\begin{minipage}{14cm}
\begin{footnotesize}
\begin{itemize}
\renewcommand{\theenumi}{\Roman{enumi}}
\item Identifique-se em todas as folhas.
\item Mantenha o celular e os demais equipamentos eletrônicos desligados durante a prova.
\item Justifique cada resposta com cálculos ou argumentos baseados na teoria estudada.
\item Sempre que calcular o valor de uma das funções consideradas em um ponto $x$, arredonde o resultado para o número de dígitos especificado, e só então use esse valor (arredondado) nas fórmulas dos métodos iterativos.
\item Resolva apenas os itens de que precisar para somar 10,0 pontos.
\end{itemize}
\end{footnotesize}
\end{minipage}
}
\end{center}

%\section*{Questões}
\begin{ExerciseList}
\Exercise[title={2,5}]
Obtenha a função do tipo $f(x) = \alpha \cdot (x-1)^2 + \beta (x+1)^2$ que melhor se ajusta aos pontos do conjunto (no sentido dos mínimos quadrados)
\[
D = \{(-1, 1), (0, 0), (1, 1), (2, 8), (2, 12)\}
\]
e utilize-a para obter $f(0,19)$.

{\color{blue} \textit{(Utilize números decimais com 3 dígitos após a vírgula)}}
\Answer
\color{red}
\ldots



\Exercise[title={2,5}]
Aproxime $\int_{-2}^{2} \frac{4}{x^2+4}\ dx$ usando a repetição da regra 1/3 de Simpson em 4 subintervalos.

{\color{blue} (Utilize números decimais com 4 dígitos após a vírgula)}
\Answer
\color{red}
\ldots


\Exercise[title={2,5}]
Qual é o maior valor de $h \in \mathbb{R}$ para o qual a regra de Gauss-Legendre com 2 pontos aproxima $\int_{-h}^h x^6 \ dx$ com um erro absoluto $|\varepsilon_{abs}| \leq 0,1$?

{\color{blue} (Utilize números decimais com 4 dígitos após a vírgula)}
\Answer \ldots



\Exercise[title={2,5}]
Considere a regra de quadratura $\int_0^{4m} f(x)\ dx \approx w_0 f(m) + w_1 f(2m)$, que é exata para polinômios de grau um. Calcule $w_0$, $w_1$ e o erro relativo percentual da aproximação de
\[
I = \int_0^{0,8} \frac{\cos(x-x^2)}{2^x} \ dx,
\]
fornecida por esta regra, sabendo que o valor exato é $I = 0,602778411\ldots$.

{\color{blue} \textit{(Utilize números decimais com 6 dígitos após a vírgula, e configure sua calculadora em radianos)}}
\Answer \ldots


\Exercise[title={2,5}]
Verifique se é verdade que se for usado um passo $h=0,500$ no método de Euler explícito, a solução aproximada do problema de valor inicial
\[
\begin{cases}
y^\prime(x) = x-y(x), \quad x \in [0,000, 2,000]\\
y(0,000) = 3,000
\end{cases}
\]
terá erros absolutos cada vez maiores ao longo do intervalo $[0,000, 2,000]$. Para o cálculo do erro, considere que $y(x)=x-1+4e^{-x}$ é a solução exata.

{\color{blue} \textit{(Utilize números decimais com 3 dígitos após a vírgula)}}
\Answer
Denotando $f(x,y) = y-x$ e $h=0,5$, pode-se expressar a fórmula do método de Euler explícito da seguinte forma:
\[
y_n
= y_{n-1} + h f(x_{n-1}, y_{n-1})
= y_{n-1} + 0,5( y_{n-1} - x_{n-1} )
= 1,5 y_{n-1} - 0,5 x_{n-1}.
\]
Disto resulta que os valores obtidos a cada passo são os seguintes:
\medskip
\begin{center}
\begin{tabular}{|c|c|r|r|c|}
\hline
$n$ & $x_n$ & $y_n= 1,5 y_{n-1} - 0,5 x_{n-1}$ & $y_{exato}(x_n)$ & $\varepsilon_n = y_n-y_{exato}(x_n)$ \\ \hline\hline
$0$ & $0,000$ & $0,100$ & $0,100$ & $0,000$ \\ \hline
$1$ & $0,500$ & $1,5 \cdot 0,100 - 0,5 \cdot 0,000 =  0,150$ & $ 0,016$ & $0,134$ \\ \hline
$2$ & $1,000$ & $1,5 \cdot 0,016 - 0,5 \cdot 0,500 = -0,025$ & $-0,446$ & $0,421$ \\ \hline
$3$ & $1,500$ & $1,5 \cdot(-0,446)-0,5 \cdot 1,000 = -0,538$ & $-1,534$ & $\textbf{0,996}$ \\ \hline
\end{tabular}
\end{center}
\medskip
Em particular, o maior erro absoluto ocorre no ponto $x = 1,5$.

No método de Euler implícito, por sua vez, utiliza-se a relação $y_n = y_{n-1} + h f(x_n, y_n)$ que, no problema considerado, pode ser reescrita  de forma equivalente como:
\[
y_n
= y_{n-1} + 0,5( y_n - x_n )
\Leftrightarrow
0,5y_n = y_{n-1} - 0,5x_n
\Leftrightarrow
y_n = 2y_{n-1} - x_n.
\]
Consequentemente, os valores obtidos a cada passo são:
\medskip
\begin{center}
\begin{tabular}{|c|c|r|r|c|}
\hline
$n$ & $x_n$ & $y_n= 2 y_{n-1} - x_n$ & $y_{exato}(x_n)$ & $\varepsilon_n = y_n-y_{exato}(x_n)$ \\ \hline\hline
$0$ & $0,000$ & $0,100$ & $0,100$ & $0,000$ \\ \hline
$1$ & $0,500$ & $2 \cdot 0,100 - 0,500 =  0,300$ & $ 0,016$ & $-0,316$ \\ \hline
$2$ & $1,000$ & $2 \cdot(-0,300) - 1,000 = -1,600$ & $-0,446$ & $-1,154$ \\ \hline
$3$ & $1,500$ & $2 \cdot(-1,600)- 1,500 = -4,700$ & $-1,534$ & $\textbf{-3,166}$ \\ \hline
\end{tabular}
\end{center}
\medskip
Novamente, o maior erro absoluto (em módulo) ocorre no ponto $x = 1,5$.


\end{ExerciseList}

\vfill
\begin{center}
BOA PROVA E BOAS FÉRIAS!
\end{center}

%\newpage
%\restoregeometry
%\section*{Respostas}
%\shipoutAnswer
\end{document}
