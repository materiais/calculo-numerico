\documentclass[12pt,a4paper]{article}
\usepackage{cmap} % Makes the PDF copiable. See http://tex.stackexchange.com/a/64198/25761
\usepackage[T1]{fontenc}
\usepackage[brazil]{babel}
\usepackage[utf8]{inputenc}
\usepackage{amsmath}
\usepackage{amsfonts}
\usepackage{amssymb}
\usepackage{amsthm}
\usepackage{textcomp} % \degree
\usepackage{gensymb} % \degree
\usepackage[usenames,svgnames,dvipsnames]{xcolor}
\usepackage{hyperref}
\usepackage{multicol}
\usepackage{graphicx}
\usepackage[margin=2cm]{geometry}
\usepackage{systeme}
\usepackage{icomma}

\hypersetup{
    colorlinks = true,
    allcolors = {blue}
}

% TODO: Consider using exsheets
% http://linorg.usp.br/CTAN/macros/latex/contrib/exsheets/exsheets_en.pdf
%
% http://ctan.org/tex-archive/macros/latex/contrib/exercise/
% Options: answerdelayed,lastexercise,noanswer
\usepackage[answerdelayed,lastexercise]{exercise}

\addto\captionsbrazil{%
\def\listexercisename{Lista de exerc\'icios}%
\def\ExerciseName{Exerc\'icio}%
\def\AnswerName{Solu\c{c}\~ao do exerc\'icio}%
\def\ExerciseListName{Ex.}%
\def\AnswerListName{Solu\c{c}\~ao}%
\def\ExePartName{Parte}%
\def\ArticleOf{de\ }%
}

\renewcommand{\ExerciseHeaderTitle}{(\ExerciseTitle)\ }
\renewcommand{\ExerciseListHeader}{%\ExerciseHeaderDifficulty%
\textbf{%\ExerciseListName\
\ExerciseHeaderNB.\ %
%\ --- \
\ExerciseHeaderTitle}%
%\ExerciseHeaderOrigin
\ignorespaces}
\renewcommand{\AnswerListHeader}{\textbf{\ExerciseHeaderNB.\ (\AnswerListName)\ }}

\newcommand*\R{\mathbb{R}}

\renewcommand{\theenumi}{\alph{enumi}}
\renewcommand\labelenumi{(\theenumi) }

\newcommand*\tipo{Prova II}
\newcommand*\turma{MEC151-03U}
\newcommand*\disciplina{CAN0001}
\newcommand*\eu{Helder G. G. de Lima}
\newcommand*\data{17/10/2018}

\author{\eu}
\title{\tipo - \disciplina}
\date{\data}

\begin{document}
\thispagestyle{empty}
\newgeometry{margin=2cm,bottom=0.5cm}
\begin{center}
\includegraphics[width=9.0cm]{marca} \\
\textbf{\tipo\ (\disciplina / \turma)} \\
Prof. \eu\footnote{
Este é um material de acesso livre distribuído sob os termos da licença \href{https://creativecommons.org/licenses/by-sa/4.0/deed.pt_BR}{Creative Commons BY-SA 4.0}}
\end{center}

\noindent Nome do(a) aluno(a): \underline{\hspace{9,7cm}} Data: \underline{\data}

%\section*{Instruções}
\begin{center}\fbox{
\begin{minipage}{14cm}
\begin{footnotesize}
\begin{itemize}
\renewcommand{\theenumi}{\Roman{enumi}}
\item Identifique-se em todas as folhas.
\item Mantenha o celular e os demais equipamentos eletrônicos desligados durante a prova.
\item Justifique cada resposta com cálculos ou argumentos baseados na teoria estudada.
\item Resolva apenas os itens de que precisar para somar 10,0 pontos.
\end{itemize}
\end{footnotesize}
\end{minipage}
}
\end{center}

%\section*{Questões}
\begin{ExerciseList}
\Exercise[title={2,5}]
Reorganize as equações do sistema linear a seguir (se necessário), e apresente argumentos teóricos que sejam suficientes para garantir a convergência da sequência gerada pelo método de Jacobi a partir da aproximação inicial $X^{(0)} = (0, 0, 0, 0)$.
\[
\begin{cases}
5x_4-x_3 = 8\\
5x_3-x_2 = 4\\
5x_1+x_2 = 1\\
5x_2-x_1 = 0.
\end{cases}
\]
Depois, obtenha uma solução aproximada com erro relativo percentual estimado inferior a $1\%$.\\
(arredonde cada resultado com \textbf{3 dígitos} após a vírgula)
\Answer
Para que o método de Jacobi produza uma sequência convergente, é suficiente (embora não seja necessário) que o sistema ao qual é aplicado tenha uma matriz de coeficientes estritamente diagonalmente dominante. Na ordem em que as equações foram apresentadas, a representação matricial do sistema é
\[
\begin{bmatrix}
0 & 0 & -1 & 5 \\
0 & -1 & 5 & 0 \\
5 & 1 & 0 & 0 \\
-1 & 5 & 0 & 0
\end{bmatrix}
\cdot
\begin{bmatrix}
x_1 \\ x_2 \\ x_3 \\ x_4
\end{bmatrix}
=
\begin{bmatrix}
8 \\ 4 \\ 1 \\ 0
\end{bmatrix}.
\]
Observa-se que a matriz de coeficientes não é estritamente diagonalmente dominante, mas isso pode ser resolvido permutando as equações, de modo a obter o seguinte:
\[
\begin{bmatrix}
5 & 1 & 0 & 0 \\
-1 & 5 & 0 & 0 \\
0 & -1 & 5 & 0 \\
0 & 0 & -1 & 5
\end{bmatrix}
\cdot
\begin{bmatrix}
x_1 \\ x_2 \\ x_3 \\ x_4
\end{bmatrix}
=
\begin{bmatrix}
1 \\ 0 \\ 4 \\ 8
\end{bmatrix}.
\]
Agora, tem-se uma matriz estritamente diagonalmente dominante, pois
\begin{align*}
|5| & > |1| + |0| + |0|\\
|5| & > |-1| + |0| + |0|\\
|5| & > |0| + |-1| + |0|\\
|5| & > |0| + |0| + |-1|.
\end{align*}
Então, por este critério, é garantido que o método iterativo produzirá uma sequência convergente, qualquer que seja a aproximação inicial escolhida. As equações utilizadas pelo Método de Jacobi são as seguintes:
\[
\begin{cases}
x_1^{(k)} = (1 - x_2^{(k-1)})/5\\
x_2^{(k)} = (x_1^{(k-1)})/5\\
x_3^{(k)} = (4 + x_2^{(k-1)})/5\\
x_4^{(k)} = (8 + x_3^{(k-1)})/5
\end{cases}
\]
Consequentemente, os valores obtidos a cada iteração são os seguintes:
\medskip
\begin{center}
\begin{tabular}{crrrr}
\hline
$\boldsymbol{k}$     & 0 & 1 & 2 & 3\\
\hline
$\boldsymbol{x_1^{(k)}}$ & 0,000 & 0,200 & 0,200 & 0,192 \\
$\boldsymbol{x_2^{(k)}}$ & 0,000 & 0,000 & 0,040 & 0,040 \\
$\boldsymbol{x_3^{(k)}}$ & 0,000 & 0,800 & 0,800 & 0,808 \\
$\boldsymbol{x_4^{(k)}}$ & 0,000 & 1,600 & 1,760 & 1,760 \\
\hline
$\varepsilon_{abs}$ & - & 1,600 & 0,160 & 0,008 \\
\hline
$\varepsilon_{per}$ & - & 100,0\% & 9,09\% & 0,45\% \\
\hline
\end{tabular}
\end{center}
\medskip

Assim, o erro relativo da aproximação $X^{(3)} = (0,192, 0,040, 0,808, 1,760)$ obtida pelo método de Jacobi é de aproximadamente $0,45\%$.


\Exercise[title={2,5}]
Determine o erro absoluto ao resolver o sistema a seguir pela eliminação de Gauss:
\begin{multicols}{2}
\begin{enumerate}
\item Sem pivoteamento
\item Com pivoteamento parcial
\end{enumerate}
$\begin{cases}
0,01x+1,00y=1,00\\
0,05x+0,02y=1,00
\end{cases}$
\end{multicols}
Considere que a solução exata é $(\overline{x},\overline{y}) = (\frac{4900}{249}, \frac{200}{249})$, e arredonde o resultado de cada operação (adição, subtração, multiplicação ou divisão), com \textbf{2 dígitos} após a vírgula.
\Answer
\begin{enumerate}
\item Estes são os passos da eliminação de Gauss sem pivoteamento:
\[
\begin{bmatrix}
0,01 & 1,00 & 1,00\\
0,05 & 0,02 & 1,00
\end{bmatrix}
\rightarrow
\begin{bmatrix}
0,01 &  1,00 & 1,00\\
0,00 & -4,98 & -4,00
\end{bmatrix}.
\]
Consequentemente, $y = \frac{-4,00}{-4,98} \approx 0,80$ e
$x = \frac{1,00 - 1,00 \cdot 0,80}{0,01}
= \frac{1,00 - 0,80}{0,01}
\approx \frac{0,20}{0,01}\approx 20,00$. Neste caso, o erro absoluto é:
\begin{align*}
\varepsilon_{abs}
& = ||(x,y) - (\overline{x},\overline{y})||
  = \max \left\{ \left|20,00 - \frac{4900}{249}\right|, \left|0,80 - \frac{200}{249}\right| \right\}\\
& = \max \left\{ \left|20,00 - 19,68\right|, \left|0,80 - 0,80\right| \right\}
  = \max \left\{ \left|0,32\right|, \left|0,00\right| \right\}
  = 0,32.
\end{align*}

\item Estes são os passos da eliminação de Gauss com pivoteamento parcial:
\[
\begin{bmatrix}
0,01 & 1,00 & 1,00\\
0,05 & 0,02 & 1,00
\end{bmatrix}
\rightarrow
\begin{bmatrix}
0,05 & 0,02 & 1,00\\
0,01 & 1,00 & 1,00
\end{bmatrix}
\rightarrow
\begin{bmatrix}
0,05 & 0,02 & 1,00\\
0,00 & 1,00 & 0,80
\end{bmatrix}
\]
Consequentemente, $y = \frac{0,80}{1,00} = 0,80$ e
$x = \frac{1,00 - 0,02 \cdot 0,80}{0,05}
\approx \frac{1,00 - 0,02}{0,05}
\approx \frac{0,98}{0,05}
\approx 19,60$. Neste caso, o erro absoluto é:
\begin{align*}
\varepsilon_{abs}
& = ||(x,y) - (\overline{x},\overline{y})||
  = \max \left\{ \left|19,60 - \frac{4900}{249}\right|, \left|0,80 - \frac{200}{249}\right| \right\}\\
& = \max \left\{ \left|19,60 - 19,68\right|, \left|0,80 - 0,80\right| \right\}
  = \max \left\{ \left|-0,08\right|, \left|0,00\right| \right\}
  = 0,08.
\end{align*}
\end{enumerate}


\Exercise[title={2,5}]
Utilize a forma de Newton do polinômio interpolador para estimar o salário mínimo de 2016, arredondado com \textbf{2 dígitos} após a vírgula, considerando os dados a seguir:
\medskip
\begin{center}
\begin{tabular}{cc}
\hline
Ano & Salário mínimo (R\$) \\
\hline
2014 & 724,00 \\
2015 & 788,00 \\
%2016 & 880 \\
2017 & 937,00 \\
2018 & 954,00 \\
\hline
\end{tabular}
\end{center}
\Answer
A partir dos pontos dados, obtém-se:
\[
\begin{array}{ccccc}
x_i
& y_i=f[x_i]
& f[x_i,x_{i+1}]
& f[x_i,x_{i+1},x_{i+2}]
& f[x_i,\ldots,x_{i+3}]\\
2014 & \mathbf{724,00} \\
     & & \mathbf{64,00} \\
2015 & 788,00 & & \mathbf{3,50} \\
     & & 74,50 & & \mathbf{-5,67}. \\
2017 & 937,00 & & -19,17 \\
     & & 17,00 \\
2018 & 954,00
\end{array}
\]

Então:
\begin{align*}
p(x)
&=724,00
 + 64,00 (x-2014)
 +  3,50 (x-2014)(x-2015)\\
&\quad +(-5,67) (x-2014)(x-2015)(x-2017).
\end{align*}
Usando este polinômio para estimar o valor pedido, resulta que:
\begin{align*}
p(2016)
&=724,00
 + 64,00\cdot (2)
 +  3,50\cdot (2)\cdot (1)\\
&\quad +(-5,67)\cdot (2)\cdot (1)\cdot (-1)\\
& = 870,33.
\end{align*}

\Exercise[title={2,5}]
Obtenha, pelo método de Lagrange, o polinômio $p(x)$ que interpola $f(x) = x^2 \cdot 2^{-x}$ em $x_0 = 1$, $x_1 = 2$ e $x_2 = 4$ e calcule o erro absoluto da aproximação $f(3) \approx p(3)$.
\\(arredonde cada resultado com \textbf{4 dígitos} após a vírgula)
\Answer
Considerando que $f(1) = \frac{1}{2}$, $f(2) = 1$ e $f(4) = 1$, o método de Lagrange permite que o polinômio que interpola $f$ nestes pontos seja descrito da seguinte forma:
\begin{align*}
p(x)
& = \frac{1}{2} L_0(x) + 1 L_1(x) + 1 L_2(x) \\
& = \frac{1}{2} \cdot \frac{(x-2)(x-4)}{(1-2)(1-4)}
  + 1 \cdot \frac{(x-1)(x-4)}{(2-1)(2-4)}
  + 1 \cdot \frac{(x-1)(x-2)}{(4-1)(4-2)}\\
& = \frac{x^2 - 6 x + 8}{6}
    + \frac{-x^2 + 5 x - 4}{2}
    + \frac{x^2 - 3 x + 2}{6}
  = -\frac{x^2}{6} + x - \frac{1}{3}.
\end{align*}
Logo,
\[
p(3) = -\frac{3^2}{6} + 3 - \frac{1}{3} = \frac{7}{6} \approx 1,1667.
\]
e o erro da aproximação $f(3) \approx p(3)$ é $\varepsilon_{abs}
= \left|\frac{7}{6} - \frac{9}{8}\right|
= \frac{1}{24}
\approx \left|1,1667 - 1,125\right|
\approx 0,0417$.


\Exercise[title={2,5}]
Considere os sistemas lineares definidos por
$
\begin{bmatrix}
1 & \mathbf{a}\\3 & 4
\end{bmatrix}
\cdot
\begin{bmatrix}
x_1 \\ x_2
\end{bmatrix}
=
\begin{bmatrix}
5 \\6
\end{bmatrix}$,
sendo $\mathbf{a} \in \R$. Escreva a relação de recorrência do método de Gauss-Seidel na forma matricial,
$X^{(k)} = C \cdot X^{(k-1)} +D$
e mostre que para todo $\mathbf{a}$ tal que $|\mathbf{a}| < 4/3$, o método sempre produzirá sequências convergentes.
\Answer
As equações utilizadas no método de Gauss-Seidel são as seguintes:
\[
\begin{cases}
x_1^{(k)} = 5 - a x_2^{(k-1)}\\
x_2^{(k)} = \frac{6}{4} - \frac{3}{4}x_1^{(k)}
\end{cases}
\]
Substituindo a primeira equação na segunda, resulta que \[
x_2^{(k)}
= \frac{6}{4} - \frac{3}{4}(5 - a x_2^{(k-1)})
= \frac{-9}{4} + \frac{3a}{4} x_2^{(k-1)},
\]
ou seja,
\[
\begin{bmatrix}
x_1^{(k)}\\
x_2^{(k)}
\end{bmatrix}
=
\begin{bmatrix}
0 & -a\\
0 & \frac{3a}{4}
\end{bmatrix}
\cdot
\begin{bmatrix}
x_1^{(k-1)}\\
x_2^{(k-1)}
\end{bmatrix}
+
\begin{bmatrix}
5\\
\frac{-9}{4}
\end{bmatrix}
\]
As sequências produzidas pelo método de Gauss-Seidel serão convergentes se, e somente se, todos os autovalores da matriz $C =
\begin{bmatrix}
0 & -a\\
0 & \frac{3a}{4}
\end{bmatrix}$ tiverem módulo menor do que um. Como o polinômio característico de $C$ é
\[
p_C(\lambda) = \det(\lambda I - C)
=
\begin{vmatrix}
\lambda & a\\
0 & \lambda-\frac{3a}{4}
\end{vmatrix}
= \lambda\left(\lambda-\frac{3a}{4}\right),
\]
conclui-se que os autovalores de $C$ são $\lambda_1 = 0$ e $\lambda_2 = \frac{3a}{4}$. Assim, tem-se a convergência se, e somente se, $\left|\frac{3a}{4}\right| < 1$, isto é, $|a| < \frac{4}{3}$. Note que neste caso, os sistemas lineares dados são possíveis e determinados, pois $\det(A) = 4-3a \neq 0$.
\end{ExerciseList}

\vfill
\begin{center}
BOA PROVA!
\end{center}

\newpage
\restoregeometry
\section*{Respostas}
\shipoutAnswer
\end{document}
