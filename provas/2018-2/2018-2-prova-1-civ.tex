\documentclass[12pt,a4paper]{article}
\usepackage{cmap} % Makes the PDF copiable. See http://tex.stackexchange.com/a/64198/25761
\usepackage[T1]{fontenc}
\usepackage[brazil]{babel}
\usepackage[utf8]{inputenc}
\usepackage{amsmath}
\usepackage{amsfonts}
\usepackage{amssymb}
\usepackage{amsthm}
\usepackage{textcomp} % \degree
\usepackage{gensymb} % \degree
\usepackage[usenames,svgnames,dvipsnames]{xcolor}
\usepackage{hyperref}
\usepackage{multicol}
\usepackage{graphicx}
\usepackage[margin=2cm]{geometry}
\usepackage{systeme}
\usepackage{icomma}

\hypersetup{
    colorlinks = true,
    allcolors = {blue}
}

% TODO: Consider using exsheets
% http://linorg.usp.br/CTAN/macros/latex/contrib/exsheets/exsheets_en.pdf
%
% http://ctan.org/tex-archive/macros/latex/contrib/exercise/
% Options: answerdelayed,lastexercise,noanswer
\usepackage[answerdelayed,lastexercise]{exercise}

\addto\captionsbrazil{%
\def\listexercisename{Lista de exerc\'icios}%
\def\ExerciseName{Exerc\'icio}%
\def\AnswerName{Solu\c{c}\~ao do exerc\'icio}%
\def\ExerciseListName{Ex.}%
\def\AnswerListName{Solu\c{c}\~ao}%
\def\ExePartName{Parte}%
\def\ArticleOf{de\ }%
}

\renewcommand{\ExerciseHeaderTitle}{(\ExerciseTitle)\ }
\renewcommand{\ExerciseListHeader}{%\ExerciseHeaderDifficulty%
\textbf{%\ExerciseListName\
\ExerciseHeaderNB.\ %
%\ --- \ 
\ExerciseHeaderTitle}%
%\ExerciseHeaderOrigin
\ignorespaces}
\renewcommand{\AnswerListHeader}{\textbf{\ExerciseHeaderNB.\ (\AnswerListName)\ }}

\newcommand*\sen{\operatorname{sen}}
\newcommand*\R{\mathbb{R}}

\renewcommand{\theenumi}{\alph{enumi}}
\renewcommand\labelenumi{(\theenumi) }

\newcommand*\tipo{Prova I}
\newcommand*\turma{CIV122-04U}
\newcommand*\disciplina{CAN0001}
\newcommand*\eu{Helder G. G. de Lima}
\newcommand*\data{13/09/2018}

\author{\eu}
\title{\tipo - \disciplina}
\date{\data}

\begin{document}
\thispagestyle{empty}
\newgeometry{margin=2cm,bottom=0.5cm}
\begin{center}
\includegraphics[width=9.0cm]{marca} \\
\textbf{\tipo\ (\disciplina / \turma)} \\
Prof. \eu\footnote{
Este é um material de acesso livre distribuído sob os termos da licença \href{https://creativecommons.org/licenses/by-sa/4.0/deed.pt_BR}{Creative Commons BY-SA 4.0}}
\end{center}

\noindent Nome do(a) aluno(a): \underline{\hspace{9,7cm}} Data: \underline{\data}

%\section*{Instruções}
\begin{center}\fbox{
\begin{minipage}{14cm}

\begin{footnotesize}
\begin{itemize}
\renewcommand{\theenumi}{\Roman{enumi}}
\item Identifique-se em todas as folhas.
\item Mantenha o celular e os demais equipamentos eletrônicos desligados durante a prova.
\item Justifique cada resposta com cálculos ou argumentos baseados na teoria estudada.
\item Sempre que calcular o valor de uma das funções consideradas em um ponto $x$, arredonde o resultado para o número de dígitos especificado, e só então use esse valor (arredondado) nas fórmulas dos métodos iterativos.
\item Resolva apenas os itens de que precisar para somar 10,0 pontos.
\end{itemize}
\end{footnotesize}

\end{minipage}
}
\end{center}

%\section*{Questões}
\begin{ExerciseList}
\Exercise[title={2,5}] Qual é a melhor aproximação de $\overline{x} = 7/20$ (em termos do erro absoluto) que pode ser obtida considerando números que, em binário, têm até 5 dígitos após a vírgula?
\Answer Considerando que $\overline{x} = 7/20 = 0,35$, os passos da conversão para binário são estes:
\begin{description}
\item $2 \times 0,35 = 0,70$
\item $2 \times 0,70 = 1,40$
\item $2 \times 0,40 = 0,80$
\item $2 \times 0,80 = 1,60$
\item $2 \times 0,60 = 1,20$
\item $2 \times 0,20 = 0,40$
\end{description}
Portanto, $\overline{x} = (0,01\overline{0110})_2$. Os dois números mais próximos, que têm apenas 5 dígitos binários são obtidos truncando a representação de $\overline{x}$ em binário, o que resulta em $x = (0,01011)_2$, ou arredondando-a para cima, o que resulta em $y = (0,01100)_2$. A representação destes números na base dez é dada por:
\[
x = 2^{-2}+2^{-4}+2^{-5} = 0,34375
\]
e
\[
y = 2^{-2}+2^{-3} = 0,375
\]
Consequentemente, $\varepsilon_{abs}(x) = |0,34375 - 7/20| = 0,00625$ e $\varepsilon_{abs}(y) = |0,375 - 7/20| = 0,025$, ou seja, $x = (0,01011)_2$ é a melhor aproximação.

\Exercise[title={2,5}]
Seja $f(x) = \sen(x)$. Aplique o método da posição falsa, partindo do intervalo inicial $[a_0, b_0] = [3, 6]$, para obter $x_k$ que satisfaça $|x_k-x_{k-1}| < 0,01$. Estime o erro relativo percentual da aproximação $x_k$ encontrada.

\textit{(Configure a calculadora em radianos e arredonde cada $x_k$ e $f(x_k)$ com 4 dígitos após a vírgula)}
\Answer Considerando que $f(x) = \sen(x)$ é contínua em $[3, 6]$, que $f(3) \approx 0,1411 > 0$ e que $f(6) \approx -0,2794 < 0$, segue do teorema de Bolzano que há uma raiz $\overline{x}$ de $f$ em $(3, 6)$. Se o método da posição falsa for aplicado para aproximá-la, resultarão os seguintes valores:

\begin{center}
\begin{tabular}{cccccccc}
\hline 
$k$ & $a_k$ & $x_k$ & $b_k$ & $f(a_k)$ & $f(x_k)$ & $f(b_k)$ & $|x_k-x_{k-1}|$ \\
\hline
0 & 3,0000 & 4,0067 & 6,0000 & 0,1411 & -0,7612 & -0,2794 & - \\
1 & 3,0000 & 3,1574 & 4,0067 & 0,1411 & -0,0158 & -0,7612 & 0,8493 \\
2 & 3,0000 & 3,1415 & 3,1574 & 0,1411 &  0,0001 & -0,0158 & 0,0159 \\
3 & 3,1415 & 3,1416 & 3,1574 & 0,0001 &  0,0000 & -0,0158 & 0,0001 \\
\hline 
\end{tabular}
\end{center}

Assim, a raiz $\overline{x}$ de $f$ é aproximadamente $x_3 = 3,1416$, com $\varepsilon_{per}(x_k) \approx \frac{0,0001}{3,1416} = 0,0032\%$.

\Exercise[title={2,5}]
Seja $f(x) = \left(\frac{1}{m}\right)^x - x$, para algum $m \in \R$. Mostre que para todo $m > 1$, existe algum $\overline{x} \in (0, 1)$ tal que $f(\overline{x}) = 0$. Calcule uma aproximação $x_k \approx \overline{x}$ pelo método da bisseção no caso em que $m = 9$, de modo que o erro absoluto aproximado seja $|\varepsilon_{abs}(x_k)| \leq 0,01$.

\textit{(Arredonde os valores utilizados no cálculo de cada $x_k$ com 4 dígitos após a vírgula)}
\Answer Considerando que $f(0) = 1 > 0$ e que $f(1) = \frac{1}{m}-1$, resulta que, para todo $m > 1$, $f(1) < 0$. Consequentemente, $f$ muda de sinal no intervalo $[0,1]$, no qual é contínua, e pelo teorema de Bolzano, possui pelo menos uma raiz em $(0,1)$, isto é, existe $\overline{x} \in (0,1)$ tal que $f(\overline{x}) = 0$.

Utilizando o método da bisseção para obter aproximações de $\overline{x}$, encontram-se os seguintes valores:

\begin{center}
\begin{tabular}{cccccccc}
\hline 
$k$ & $a_k$ & $x_k$ & $b_k$ & $f(a_k)$ & $f(x_k)$ & $f(b_k)$ & $\varepsilon_{abs}(x_k)$\\\hline
0 & 0,0000 & 0,5000 & 1,0000 & 1,0000 & -0,1667 & -0,8889 & -\\
1 & 0,0000 & 0,2500 & 0,5000 & 1,0000 & 0,3274 & -0,1667 & 0,2500\\
2 & 0,2500 & 0,3750 & 0,5000 & 0,3274 & 0,0637 & -0,1667 & 0,1250\\
3 & 0,3750 & 0,4375 & 0,5000 & 0,0637 & -0,0551 & -0,1667 & 0,0625\\
4 & 0,3750 & 0,4062 & 0,4375 & 0,0637 & 0,0034 & -0,0551 & 0,0313\\
5 & 0,4062 & 0,4218 & 0,4375 & 0,0034 & -0,0260 & -0,0551 & 0,0156\\
6 & 0,4062 & 0,4140 & 0,4218 & 0,0034 & -0,0113 & -0,0260 & 0,0078\\\hline
\end{tabular}
\end{center}
Portanto, $x_6 = 0,4140 \approx \overline{x}$, com um erro absoluto dentro da tolerância estipulada.

\Exercise[title={2,5}]
Obtenha uma aproximação $x_k$ de uma raiz da função $f(x) = \sen(x)-1/9$ tal que $|f(x_k)| < 10^{-6}$, utilizando o método de Newton-Raphson com $x_0 = 0,5$.

\textit{(Configure a calculadora em radianos e arredonde cada valor com 8 dígitos após a vírgula)}
\Answer
Considerando que, por exemplo, $f(0) = -1/9 < 0 < 0,73 \approx f(1)$ e que $f$ é uma função contínua, resulta do teorema de Bolzano que $f$ possui alguma raiz $\overline{x} \in I = (0, 1)$. Além disso, $f^\prime(x) = \cos(x)$ e $f^{\prime\prime}(x) = -\sen(x)$ são contínuas em $I$, e sendo $f^\prime(x) \neq 0$ para $x \in I$, o método de Newton-Raphson produzirá uma sequência convergente, para toda aproximação inicial $x_0$ em algum subintervalo $\overline{I} \subset (0, 1)$. Usando o valor que foi proposto, $x_0 = 0,5 \in I$, e aplicando o método de Newton-Raphson, obtém-se os seguintes valores:

\begin{center}
\begin{tabular}{cccc}
\hline 
$k$ & $x_k$ & $f(x_k)$ & $f^\prime(x_k)$ \\\hline
0 & 0,50000000 &  0,36831443 & 0,87758256\\
1 & 0,08030794 & -0,03088947 & 0,99677705\\
2 & 0,11129729 & -0,00004345 & 0,99381285\\
3 & 0,11134101 &  0,00000000 & -\\\hline
\end{tabular}
\end{center}

Portanto, $x_3 = 0,11134101$ é o valor aproximado de uma raiz de $f$, e $|f(x_3)| < 10^{-6}$.


\Exercise[title={2,5}]
Considere $f(x) = x^3 + 10x - 10$ e a função de iteração $\varphi(x) = 1 - x^3/10$.
\begin{enumerate}
\item Mostre, por meio de argumentos teóricos, que se $x_0 = 1$ e $x_k = \varphi(x_{k-1})$, para $k \in \mathbb{N}$, então a sequência $\{x_k\}_{k \in \mathbb{N}}$ converge para algum $\overline{x}$ tal que $f(\overline{x}) = 0$. 
\item Utilize a função de iteração dada para calcular uma aproximação $x_k \approx \overline{x}$ cujo erro relativo percentual estimado seja menor ou igual a $0,5\%$.
\end{enumerate}
\textit{(Configure a calculadora em radianos e arredonde cada termo $x_k$ com 4 dígitos após a vírgula)}
\Answer
\begin{enumerate}
\item Como $f(x) = x^3 + 10x - 10 = 0$ equivale a $x = \varphi(x) = 1 - x^3/10$, a função $\varphi$ é, de fato, uma função de iteração para $f$. Além disso, tem-se $\varphi^\prime(x) = -\frac{3}{10}x^2$ e portanto as funções $\varphi$ e $\varphi^\prime$ são contínuas em $\R$.

Considerando que
\[
|\varphi^\prime(x)| < 1
\Leftrightarrow
\left|-\frac{3}{10}x^2\right| < 1
\Leftrightarrow
x^2 < \frac{10}{3}
\Leftrightarrow
x \in I = \left(-\sqrt{\frac{10}{3}}, \sqrt{\frac{10}{3}}\right),
\]
tem-se em particular que, para todo $x \in (-0,5, 1,5) \subset I$, vale $|\varphi^\prime(x)| < 1$. Como
\[
f(0,5) = -4,875 < 0 < 8,375 = f(1,5),
\]
e $f$ é contínua em $[0,5, 1,5]$, segue do teorema de Bolzano que há uma raiz de $f$ em $I$. Essa raiz pode ser obtida pelo método de iteração de ponto fixo como o limite da sequência dada por $x_k = \varphi(x_{k-1})$, para qualquer $x_0 \in I$, inclusive $x_0 = 1$.

\item Os primeiros termos dessa sequência são os seguintes (arredondados no quarto dígito decimal a cada iteração).

\begin{center}
\begin{tabular}{cccc}
\hline
$k$ & $x_k$ & $\varphi(x_k)$ & $\varepsilon_{per}(x_k)$\\
\hline
0 & 1,0000 & 0,9000 & - \\
1 & 0,9000 & 0,9271 & 11,1111\% \\
2 & 0,9271 & 0,9203 &  2,9231\% \\
3 & 0,9203 & 0,9221 &  0,7389\% \\
4 & 0,9221 & 0,9216 &  0,1952\% \\
\hline
\end{tabular}
\end{center}
Portanto um valor aproximado de $\overline{x}$ nas condições exigidas é $x_4 = 0,9221$, que tem um erro relativo percentual de cerca de $0,1952\%$.
\end{enumerate}

\end{ExerciseList}

\vfill
\begin{center}
BOA PROVA!
\end{center}

\newpage
\restoregeometry
\section*{Respostas}
\shipoutAnswer
\end{document}
