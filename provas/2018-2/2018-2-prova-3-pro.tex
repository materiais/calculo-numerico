\documentclass[12pt,a4paper]{article}
\usepackage{cmap} % Makes the PDF copiable. See http://tex.stackexchange.com/a/64198/25761
\usepackage[T1]{fontenc}
\usepackage[brazil]{babel}
\usepackage[utf8]{inputenc}
\usepackage{amsmath}
\usepackage{amsfonts}
\usepackage{amssymb}
\usepackage{amsthm}
\usepackage{textcomp} % \degree
\usepackage{gensymb} % \degree
\usepackage[usenames,svgnames,dvipsnames]{xcolor}
\usepackage{hyperref}
\usepackage{multicol}
\usepackage{graphicx}
\usepackage[margin=2cm]{geometry}
\usepackage{systeme}
\usepackage{icomma}

\hypersetup{
    colorlinks = true,
    allcolors = {blue}
}

% TODO: Consider using exsheets
% http://linorg.usp.br/CTAN/macros/latex/contrib/exsheets/exsheets_en.pdf
%
% http://ctan.org/tex-archive/macros/latex/contrib/exercise/
% Options: answerdelayed,lastexercise,noanswer
\usepackage[answerdelayed,lastexercise]{exercise}

\addto\captionsbrazil{%
\def\listexercisename{Lista de exerc\'icios}%
\def\ExerciseName{Exerc\'icio}%
\def\AnswerName{Solu\c{c}\~ao do exerc\'icio}%
\def\ExerciseListName{Ex.}%
\def\AnswerListName{Solu\c{c}\~ao}%
\def\ExePartName{Parte}%
\def\ArticleOf{de\ }%
}

\renewcommand{\ExerciseHeaderTitle}{(\ExerciseTitle)\ }
\renewcommand{\ExerciseListHeader}{%\ExerciseHeaderDifficulty%
\textbf{%\ExerciseListName\
\ExerciseHeaderNB.\ %
%\ --- \
\ExerciseHeaderTitle}%
%\ExerciseHeaderOrigin
\ignorespaces}
\renewcommand{\AnswerListHeader}{\textbf{\ExerciseHeaderNB.\ (\AnswerListName)\ }}

\renewcommand{\theenumi}{\alph{enumi}}
\renewcommand\labelenumi{(\theenumi) }

\newcommand*\tipo{Prova III}
\newcommand*\turma{PRO112-04U}
\newcommand*\disciplina{CAN0001}
\newcommand*\eu{Helder G. G. de Lima}
\newcommand*\data{27/11/2018}

\author{\eu}
\title{\tipo - \disciplina}
\date{\data}

\begin{document}
\thispagestyle{empty}
\newgeometry{margin=2cm,bottom=0.5cm}
\begin{center}
\includegraphics[width=9.0cm]{marca} \\
\textbf{\tipo\ (\disciplina / \turma)} \\
Prof. \eu\footnote{
Este é um material de acesso livre distribuído sob os termos da licença \href{https://creativecommons.org/licenses/by-sa/4.0/deed.pt_BR}{Creative Commons BY-SA 4.0}}
\end{center}

\noindent Nome do(a) aluno(a): \underline{\hspace{9,7cm}} Data: \underline{\data}

%\section*{Instruções}
\begin{center}\fbox{
\begin{minipage}{14cm}
\begin{footnotesize}
\begin{itemize}
\renewcommand{\theenumi}{\Roman{enumi}}
\item Identifique-se em todas as folhas.
\item Mantenha o celular e os demais equipamentos eletrônicos desligados durante a prova.
\item Justifique cada resposta com cálculos ou argumentos baseados na teoria estudada.
\item Sempre que calcular o valor de uma das funções consideradas em um ponto $x$, arredonde o resultado para o número de dígitos especificado, e só então use esse valor (arredondado) nas fórmulas dos métodos iterativos.
\item Resolva apenas os itens de que precisar para somar 10,0 pontos.
\end{itemize}
\end{footnotesize}
\end{minipage}
}
\end{center}

%\section*{Questões}
\begin{ExerciseList}
\Exercise[title={2,5}]
Obtenha o polinômio de grau no máximo 2 que melhor se ajusta (no sentido dos mínimos quadrados) aos pontos do conjunto
\[
D = \{(-1, 3), (0, 2), (1, 2), (2, 1) \},
\]
e calcule o resíduo quadrático da função obtida.

{\color{blue} \textit{(Utilize números decimais com 2 dígitos após a vírgula)}}
\Answer
Sejam $P_0 = (-1, 3)$, $P_1 = (0, 2)$, $P_2 = (1, 2)$ e $P_3 = (2, 1)$ e denote $g_0(x) = 1$, $g_1(x) = x$ e $g_2(x) = x^2$. Para encontrar uma função da forma $q(x) = a_0 g_0(x) + a_1 g_1(x)+ a_2 g_2(x)$ que melhor se ajusta aos pontos $P_i = (x_i,y_i)$, para $0 \leq i \leq 3$, basta resolver o sistema $A^T A X = A^T B$, em que
\[
A
= \begin{bmatrix}
g_0(x_0) & g_1(x_0) & g_2(x_0) \\
\vdots   & \vdots   & \vdots   \\
g_0(x_3) & g_1(x_3) & g_2(x_3) \\
\end{bmatrix}
= \begin{bmatrix}
1 & x_0 & x_0^2 \\
\vdots & \vdots & \vdots\\
1 & x_3 & x_3^2 \\
\end{bmatrix},
\quad
X =
\begin{bmatrix}
a_0\\a_1\\a_2
\end{bmatrix},
\quad
B = \begin{bmatrix}
y_0 \\
\vdots \\
y_3 \\
\end{bmatrix},
\]
Considerando os dados fornecidos, tem-se:
\[
A^T A
= \begin{bmatrix}
 4                  & \sum_{i=0}^3 x_i   & \sum_{i=0}^3 x_i^2 \\
\sum_{i=0}^3 x_i    & \sum_{i=0}^3 x_i^2 & \sum_{i=0}^3 x_i^3 \\
\sum_{i=0}^3 x_i^2  & \sum_{i=0}^3 x_i^3 & \sum_{i=0}^3 x_i^4
\end{bmatrix}
=\begin{bmatrix}
 1 & 1 & 1 & 1 \\
-1 & 0 & 1 & 2 \\
 1 & 0 & 1 & 4
\end{bmatrix}
\cdot
\begin{bmatrix}
  1 & -1 & 1\\
  1 & 0 & 0\\
  1 & 1 & 1\\
  1 & 2 & 4
\end{bmatrix}
=\begin{bmatrix}
4 & 2 & 6 \\
2 & 6 & 8 \\
6 & 8 & 18
\end{bmatrix},
\]
e
\[
A^T B
= \begin{bmatrix}
 \sum_{i=0}^3 y_i     \\
 \sum_{i=0}^3 x_i y_i \\
 \sum_{i=0}^3 x_i^2 y_i
\end{bmatrix}
= \begin{bmatrix}
 1 & 1 & 1 & 1 \\
-1 & 0 & 1 & 2 \\
 1 & 0 & 1 & 4
\end{bmatrix}
\begin{bmatrix}
3 \\ 2 \\ 2 \\ 1
\end{bmatrix}
= \begin{bmatrix}
8 \\ 1 \\ 9
\end{bmatrix}.
\]
Então,
$
A^T A X = A^T B
\Leftrightarrow
\begin{bmatrix}
4 & 2 & 6 \\
2 & 6 & 8 \\
6 & 8 & 18
\end{bmatrix}
\cdot
\begin{bmatrix}
a_0\\
a_1\\
a_2
\end{bmatrix}
=
\begin{bmatrix}
8 \\ 1 \\ 9
\end{bmatrix}
\Leftrightarrow
\begin{bmatrix}
a_0\\
a_1\\
a_2
\end{bmatrix}
=
\begin{bmatrix}
2,30\\
-0,60\\
0,00
\end{bmatrix}.
$

Portanto, a função é $q(x) = 2,30 - 0,60x + 0,00x^2$, isto é, $q(x) = 2,30 - 0,60x$. O resíduo quadrático é dado por
\[
R = \sum_0^3 (q(x_i) - y_i)^2
  = (2,90 - 3,00)^2 + (2,30 - 2,00)^2 + (1,70 - 2,00)^2 + (1,10 - 1,00)^2 = 0,20.
\]



\Exercise[title={2,5}]
Mostre que a regra do ponto médio fornece o valor exato de $\int_a^b f(x)\ dx$, para toda função afim $f(x) = cx+d$, e que tem um erro de $25\%$ ao aproximar $\int_0^m x^2 \ dx$, para todo $m > 0$.
\Answer
Observe que o valor exato da integral é
\[
I
= \int_a^b (cx+d) \ dx
= \left(c\frac{x^2}{2}+dx\right)\big|_a^b
= c\frac{(b^2-a^2)}{2}+d(b-a).
= \frac{cb^2}{2}-\frac{ca^2}{2}+db-da.
\]
e que a aproximação por meio da regra do ponto médio é
\[
I_1
= f\left(\frac{a+b}{2}\right)\cdot (b-a)
= \left(c\left[\frac{a+b}{2}\right]+d\right)\cdot (b-a)
= c\left[\frac{a+b}{2}\right]\cdot (b-a)+d(b-a)
= I,
\]
ou seja, a regra é exata para todas as funções afins. Por outro lado,
\[
I
= \int_0^m x^2 \ dx
= \left(\frac{x^3}{3}\right)\big|_0^m
= \frac{m^3 - 0^3}{3}
= \frac{m^3}{3}
\]
e o valor aproximado é
\[
I_1
= f\left(\frac{0+m}{2}\right)\cdot (m-0)
= \left(\frac{m}{2}\right)^2 \cdot m
= \frac{m^3}{4}.
\]
Portanto,
\[
|\varepsilon_{per}|
= \left|\frac{\frac{m^3}{4} - \frac{m^3}{3}}{ \frac{m^3}{3} }\right|\times 100 \%
= \left|\frac{\frac{3m^3-4m^3}{12}}{ \frac{m^3}{3} }\right|\times 100 \%
= \frac{3}{12} \times 100 \%
= 25 \%.
\]


\Exercise[title={2,5}]
Deduza os coeficientes de uma regra de Newton-Cotes fechada (de ordem $2$) para aproximar $\int_{-1}^{0,5} f(x) dx$,
sabendo que
$f(-1) = 3,2$,
$f(-0,5) = 2,4$ e
$f( 0,5) = 1,5$.

{\color{blue} \textit{(Utilize números decimais com 3 dígitos após a vírgula)}}
\Answer Se uma regra de quadratura da forma
\[
\int_{-1}^{0,5} f(x) dx
\approx w_0 f(-1) + w_0 f(-0,5) + w_0 f(0,5),
\]
tiver ordem 2, ou seja, for exata para todas as funções polinomiais de grau no máximo 2, então os pesos $w_i$ serão a solução do sistema
\[
\begin{bmatrix}
 1 &  1   & 1 \\
-1 & -0,5 & 0,5 \\
 1 &  0,25 & 0,25
\end{bmatrix}
\cdot
\begin{bmatrix}
w_0\\
w_1\\
w_2
\end{bmatrix}
=
\begin{bmatrix}
0,5-(-1) \\
\frac{(0,5)^2 - (-1)^2}{2} \\
\frac{(0,5)^3 - (-1)^3}{3}
\end{bmatrix}
=
\begin{bmatrix}
1,5 \\
-0,375 \\
0,375
\end{bmatrix}
\]
Assim, resolvendo o sistema, conclui-se que $w_0 = 0$, $w_1 = \frac{9}{8} = 1,125$ e $w_2 = \frac{3}{8} = 0,375$, ou seja,
\[
\int_{-1}^{0,5} f(x) dx
\approx 1,125 \cdot f(-0,5) + 0,375 \cdot f(0,5)
= 1,125 \cdot 2,4 + 0,375 \cdot 1,5
= 3,2625.
\]

\Exercise[title={2,5}]
Sabendo que
$I = \int_{-1}^{5} \cos(t) \, dt
= \operatorname{sen}(1) + \operatorname{sen}(5)
\approx -0,11745$,
determine o erro absoluto (em módulo) da aproximação de $I$ pela regra de Gauss-Legendre com 5 pontos.

{\color{blue} \textit{(Utilize números decimais com 5 dígitos após a vírgula)}}
\Answer
Fazendo a mudança de variáveis $x = 3t + 2$, tem-se $dx = 3 dt$ e
$\int_{-1}^{5} \cos(x)\,dx = 3\int_{-1}^1 \ln(3t + 2)\,dt$.
Assim, pode-se aplicar o método de Gauss-Legendre, arredondando todos os valores relevantes com 5 dígitos após a vírgula, para obter:
\begin{align*}
I & \approx
3 \cdot (
    0,23693 \cdot \cos(3\cdot (-0,90618) + 2) \\
& + 0,47863 \cdot \cos(3\cdot (-0,53847) + 2) \\
& + 0,56889 \cdot \cos(3\cdot ( 0      ) + 2) \\
& + 0,47863 \cdot \cos(3\cdot ( 0,53847) + 2) \\
& + 0,23693 \cdot \cos(3\cdot ( 0,90618) + 2) \\
& =
3\cdot (
  0,23693 \cdot  0,75277
+ 0,47863 \cdot  0,92695
+ 0,56889 \cdot (-0,41615) \\
& \qquad
+ 0,47863 \cdot (-0,88983)
+ 0,23693 \cdot  0,00615)\\
& = 3 \cdot (0,17835 + 0,44367 - 0,23674 - 0,42590 + 0,00146)
= 3 \cdot - 0,03916
= -0,11748
\end{align*}

Assim, $\varepsilon_{abs} = |-0,11748 - (-0,11745)| = 0,00003$.


\Exercise[title={2,5}]
Mostre que a solução aproximada do problema de valor inicial
\[
\begin{cases}
y^\prime(x) = -8 \cdot \dfrac{x}{y(x)}, \quad x \in [0,000, 0,300]\\
y(0,000) = 1,000
\end{cases}
\]
pelo método de Euler explícito, com passo $h=0,050$ produz erros absolutos cada vez maiores ao longo do intervalo $[0,000, 0,300]$. Considere que a solução exata é $y(x) = \sqrt{1-8x^2}$.

{\color{blue} \textit{(Utilize números decimais com 3 dígitos após a vírgula)}}
\Answer
Denotando $f(x,y) = -8 \cdot \dfrac{x}{y}$ e $h=0,050$, pode-se expressar a fórmula do método de Euler explícito da seguinte forma:
\[
y_n
= y_{n-1} + h f(x_{n-1}, y_{n-1})
= y_{n-1} + 0,05 \cdot \left( -8 \cdot \dfrac{x_{n-1}}{y_{n-1}} \right)
= y_{n-1} - 0,4 \cdot \dfrac{x_{n-1}}{y_{n-1}}.
\]
Disto resulta que os valores obtidos a cada passo são os seguintes:
\medskip
\begin{center}
\begin{tabular}{|c|c|r|r|c|}
\hline
$n$ & $x_n$ & $y_n= y_{n-1} - 0,4 x_{n-1} / y_{n-1}$ & $y_{exato}(x_n)$ & $\varepsilon_n = y_n-y_{exato}(x_n)$ \\ \hline\hline
$0$ & $0,000$ & $1,000$                                   & $1,000$ & $0,000$ \\ \hline
$1$ & $0,050$ & $1,000 - 0,4 \cdot 0,000 / 1,000 = 1,000$ & $0,990$ & $0,010$ \\ \hline
$2$ & $0,100$ & $1,000 - 0,4 \cdot 0,050 / 1,000 = 0,980$ & $0,959$ & $0,021$ \\ \hline
$3$ & $0,150$ & $0,980 - 0,4 \cdot 0,100 / 0,980 = 0,939$ & $0,906$ & $0,034$ \\ \hline
$4$ & $0,200$ & $0,939 - 0,4 \cdot 0,150 / 0,939 = 0,875$ & $0,825$ & $0,051$ \\ \hline
$5$ & $0,250$ & $0,875 - 0,4 \cdot 0,200 / 0,875 = 0,784$ & $0,707$ & $0,077$ \\ \hline
$6$ & $0,300$ & $0,784 - 0,4 \cdot 0,250 / 0,784 = 0,656$ & $0,529$ & $0,127$ \\ \hline
\end{tabular}
\end{center}
\medskip
Em particular, percebe-se que os erros absolutos crescem a cada passo.
\end{ExerciseList}

\vfill
\begin{center}
BOA PROVA E BOAS FÉRIAS!
\end{center}

\newpage
\restoregeometry
\section*{Respostas}
\shipoutAnswer
\end{document}
