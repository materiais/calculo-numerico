\documentclass[12pt,a4paper]{article}
\usepackage{cmap} % Makes the PDF copiable. See http://tex.stackexchange.com/a/64198/25761
\usepackage[T1]{fontenc}
\usepackage[brazil]{babel}
\usepackage[utf8]{inputenc}
\usepackage{amsmath}
\usepackage{amsfonts}
\usepackage{amssymb}
\usepackage{amsthm}
\usepackage{textcomp} % \degree
\usepackage{gensymb} % \degree
\usepackage[usenames,svgnames,dvipsnames]{xcolor}
\usepackage{hyperref}
\usepackage{multicol}
\usepackage{graphicx}
\usepackage[margin=2cm]{geometry}
\usepackage{systeme}
\usepackage{icomma}
\usepackage{multicol}

\hypersetup{
    colorlinks = true,
    allcolors = {blue}
}

% TODO: Consider using exsheets
% http://linorg.usp.br/CTAN/macros/latex/contrib/exsheets/exsheets_en.pdf
%
% http://ctan.org/tex-archive/macros/latex/contrib/exercise/
% Options: answerdelayed,lastexercise,noanswer
\usepackage[answerdelayed,lastexercise]{exercise}

\addto\captionsbrazil{%
\def\listexercisename{Lista de exerc\'icios}%
\def\ExerciseName{Exerc\'icio}%
\def\AnswerName{Solu\c{c}\~ao do exerc\'icio}%
\def\ExerciseListName{Ex.}%
\def\AnswerListName{Solu\c{c}\~ao}%
\def\ExePartName{Parte}%
\def\ArticleOf{de\ }%
}

\renewcommand{\ExerciseHeaderTitle}{(\ExerciseTitle)\ }
\renewcommand{\ExerciseListHeader}{%\ExerciseHeaderDifficulty%
\textbf{%\ExerciseListName\
\ExerciseHeaderNB.\ %
%\ --- \ 
\ExerciseHeaderTitle}%
%\ExerciseHeaderOrigin
\ignorespaces}
\renewcommand{\AnswerListHeader}{\textbf{\ExerciseHeaderNB.\ (\AnswerListName)\ }}

\renewcommand{\theenumi}{\alph{enumi}}
\renewcommand\labelenumi{(\theenumi) }

\newcommand*\tipo{Prova III}
\newcommand*\turma{MEC151-03U}
\newcommand*\disciplina{CAN0001}
\newcommand*\eu{Helder G. G. de Lima}
\newcommand*\data{28/11/2018}

\author{\eu}
\title{\tipo - \disciplina}
\date{\data}

\begin{document}
\thispagestyle{empty}
\newgeometry{margin=2cm,bottom=0.5cm}
\begin{center}
\includegraphics[width=9.0cm]{marca} \\
\textbf{\tipo\ (\disciplina / \turma)} \\
Prof. \eu\footnote{
Este é um material de acesso livre distribuído sob os termos da licença \href{https://creativecommons.org/licenses/by-sa/4.0/deed.pt_BR}{Creative Commons BY-SA 4.0}}
\end{center}

\noindent Nome do(a) aluno(a): \underline{\hspace{9,7cm}} Data: \underline{\data}

%\section*{Instruções}
\begin{center}\fbox{
\begin{minipage}{14cm}
\begin{footnotesize}
\begin{itemize}
\renewcommand{\theenumi}{\Roman{enumi}}
\item Identifique-se em todas as folhas.
\item Mantenha o celular e os demais equipamentos eletrônicos desligados durante a prova.
\item Justifique cada resposta com cálculos ou argumentos baseados na teoria estudada.
\item Sempre que calcular o valor de uma das funções consideradas em um ponto $x$, arredonde o resultado para o número de dígitos especificado, e só então use esse valor (arredondado) nas fórmulas dos métodos iterativos.
\item Resolva apenas os itens de que precisar para somar 10,0 pontos.
\end{itemize}
\end{footnotesize}
\end{minipage}
}
\end{center}

%\section*{Questões}
\begin{ExerciseList}
\Exercise[title={2,5}]
Obtenha a função do tipo $f(x) = \alpha \cos(\pi x) + \beta \operatorname{sen}(\frac{\pi}{2} x)$ que melhor se ajusta aos pontos do conjunto (no sentido dos mínimos quadrados)
\[
D = \{(-1, 1), (2, -2), (4, 0), (5, 3)\}
\]
e calcule o resíduo quadrático da função obtida.

{\color{blue} \textit{(Utilize números decimais com 2 dígitos após a vírgula)}}
\Answer
\color{red}
\ldots


\Exercise[title={2,5}]
Mostre que se for usado um passo $h=0,200$ no método de Euler explícito, a solução aproximada do problema de valor inicial
\[
\begin{cases}
y^\prime(x) = \dfrac{y(x)}{2x + 1}, \quad x \in [1,000, 2,000]\\
y(1,000) = 3,000
\end{cases}
\]
terá erros absolutos cada vez maiores ao longo do intervalo $[1,000, 2,000]$. Para o cálculo do erro, considere que $y(x)=\sqrt{6x + 3}$ é a solução exata.

{\color{blue} \textit{(Utilize números decimais com 3 dígitos após a vírgula)}}
\Answer
\color{red}
\ldots


\Exercise[title={2,5}]
Seja $\overline{I} = \int_0^{2,1} x^2(2-x)^2\ dx$ e considere a aproximação $I \approx \overline{I}$ obtida ao repetir a regra $1/3$ de Simpson em $3$ subintervalos. Verifique que o erro relativo percentual da aproximação é menor do que $0,5\%$.

{\color{blue} (Utilize números decimais com 3 dígitos após a vírgula)}
\Answer
\color{red}
\ldots

\Exercise[title={2,5}]
Explique o que é a \textit{ordem de exatidão} de uma regra de quadratura e determine $w_0$ e $w_1$ (se existirem) para que a ordem de exatidão da seguinte regra de quadratura seja 1:
\[
\int_0^4 f(x)\ dx \approx w_0 f(3) + w_1 f(4).
\]

{\color{blue} \textit{(Utilize números decimais com 3 dígitos após a vírgula)}}
\Answer
\color{red}
\ldots



\Exercise[title={2,5}]
Determine o erro relativo percentual da aproximação de $I = \int_{0}^{1} t^{10} \, dt$, pela regra de Gauss-Legendre com 4 pontos.

{\color{blue} (Utilize números decimais com 4 dígitos após a vírgula)}
\Answer
\color{red}
\ldots

\end{ExerciseList}

\vfill
\begin{center}
BOA PROVA E BOAS FÉRIAS!
\end{center}


%\newpage
%\restoregeometry
%\section*{Respostas}
%\shipoutAnswer
\end{document}
