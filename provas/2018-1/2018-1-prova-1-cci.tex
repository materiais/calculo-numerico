\documentclass[12pt,a4paper]{article}
\usepackage{cmap} % Makes the PDF copiable. See http://tex.stackexchange.com/a/64198/25761
\usepackage[T1]{fontenc}
\usepackage[brazil]{babel}
\usepackage[utf8]{inputenc}
\usepackage{amsmath}
\usepackage{amsfonts}
\usepackage{amssymb}
\usepackage{amsthm}
\usepackage{textcomp} % \degree
\usepackage{gensymb} % \degree
\usepackage[usenames,svgnames,dvipsnames]{xcolor}
\usepackage{hyperref}
\usepackage{multicol}
\usepackage{graphicx}
\usepackage[margin=2cm]{geometry}
\usepackage{systeme}

\hypersetup{
    colorlinks = true,
    allcolors = {blue}
}

% TODO: Consider using exsheets
% http://linorg.usp.br/CTAN/macros/latex/contrib/exsheets/exsheets_en.pdf
%
% http://ctan.org/tex-archive/macros/latex/contrib/exercise/
% Options: answerdelayed,lastexercise,noanswer
\usepackage[answerdelayed,lastexercise]{exercise}

\addto\captionsbrazil{%
\def\listexercisename{Lista de exerc\'icios}%
\def\ExerciseName{Exerc\'icio}%
\def\AnswerName{Solu\c{c}\~ao do exerc\'icio}%
\def\ExerciseListName{Ex.}%
\def\AnswerListName{Solu\c{c}\~ao}%
\def\ExePartName{Parte}%
\def\ArticleOf{de\ }%
}

\renewcommand{\ExerciseHeaderTitle}{(\ExerciseTitle)\ }
\renewcommand{\ExerciseListHeader}{%\ExerciseHeaderDifficulty%
\textbf{%\ExerciseListName\
\ExerciseHeaderNB.\ %
%\ --- \
\ExerciseHeaderTitle}%
%\ExerciseHeaderOrigin
\ignorespaces}
\renewcommand{\AnswerListHeader}{\textbf{\ExerciseHeaderNB.\ (\AnswerListName)\ }}

\renewcommand{\theenumi}{\alph{enumi}}
\renewcommand\labelenumi{(\theenumi) }

\newcommand*\tipo{Prova I}
\newcommand*\turma{CCI122-03U}
\newcommand*\disciplina{ANN0001}
\newcommand*\eu{Helder G. G. de Lima}
\newcommand*\data{03/04/2018}

\author{\eu}
\title{\tipo - \disciplina}
\date{\data}

\begin{document}
\thispagestyle{empty}
\newgeometry{margin=2cm,bottom=0.5cm}
\begin{center}
\includegraphics[width=9.0cm]{marca} \\
\textbf{\tipo\ (\disciplina / \turma)} \\
Prof. \eu\footnote{
Este é um material de acesso livre distribuído sob os termos da licença \href{https://creativecommons.org/licenses/by-sa/4.0/deed.pt_BR}{Creative Commons BY-SA 4.0}}
\end{center}

\noindent Nome do(a) aluno(a): \underline{\hspace{9,7cm}} Data: \underline{\data}

%\section*{Instruções}
\begin{center}\fbox{
\begin{minipage}{14cm}

{\footnotesize
\begin{itemize}
\renewcommand{\theenumi}{\Roman{enumi}}
\item Identifique-se em todas as folhas.
\item Mantenha o celular e os demais equipamentos eletrônicos desligados durante a prova.
\item Justifique cada resposta com cálculos ou argumentos baseados na teoria estudada.
\item Sempre que calcular o valor de uma das funções consideradas em um ponto $x$, arredonde o resultado para o número de dígitos especificado, e só então use esse valor (arredondado) nas fórmulas dos métodos iterativos.
\item Resolva apenas os itens de que precisar para somar 10,0 pontos.
\end{itemize}
}

\end{minipage}
}
\end{center}

%\section*{Questões}
\begin{ExerciseList}
\Exercise[title={2,5}]
Seja $\overline{x} = \dfrac{618}{50}$.
\begin{enumerate}
\item Obtenha a representação de $\overline{x}$ em binário, com 8 algarismos corretos após a vírgula.
\item Quantos desses algarismos (binários) são necessários após a vírgula para representar $\overline{x}$ com erro relativo percentual inferior a $5\%$?
\end{enumerate}
\Answer Observe que $\overline{x} = \dfrac{618}{50} = 12,36 = 12 + 0,36$.
\begin{enumerate}
\item Tem-se $12 = 8 + 4 = 1 \cdot 2^3 + 1 \cdot 2^2 +0 \cdot 2^1 + 0 \cdot 2^0 = (1100)_{2}$ e além disso,
\begin{center}
\begin{tabular}{|c|l|l|l|l|l|l|l|l|}
\hline
$\mathbf{x}$
& 0,36 & 0,72 & 0,44 & 0,88 & 0,76 & 0,52 & 0,04 & 0,08 \\ \hline
$\mathbf{2\cdot x}$
& \textbf{0},72
& \textbf{1},44
& \textbf{0},88
& \textbf{1},76
& \textbf{1},52
& \textbf{1},04
& \textbf{0},08
& \textbf{0},16
\\ \hline
\end{tabular}
\end{center}
Logo,
\[
\overline{x}
= \dfrac{618}{50}
= (12,36)_{10}
\approx (1100,01011100)_2.
\]
\item Ao truncar $\overline{x} = 12,65$ para o inteiro $x = 12$, o erro relativo percentual é
\[
\varepsilon_{per} = \frac{|12 - 12,65|}{|12,65|} \times 100\%
                  = \frac{|-0,65|}{|12,65|} \times 100\%
                  \approx 2,91\%.
\]
Então, não são necessários dígitos após a vírgula para que o erro seja menor do que $5\%$.

\textbf{Solução 2}: Observando que
\[
\varepsilon_{per} < 5\%
\Leftrightarrow
\frac{|x-12,36|}{|12,36|} < 0,05
\Leftrightarrow
11,742< x <12,978
\]
e que $12 \in (11,742, 12,978)$, conclui-se que não é preciso nenhum dígito após a vírgula.
\end{enumerate}

\Exercise[title={2,5}] Considere $f(x) = \ln(x) - \dfrac{1}{x}$ e $x_0 = 2$.
\begin{enumerate}
\item Mostre que a função $\varphi_1(x) = \dfrac{1}{\ln(x)}$ é uma função de iteração para $f(x) = 0$.
\item Discuta se é possível garantir que a sequência dada por $x_k = \varphi_1(x_{k-1})$, convergirá para algum $\overline{x}$ tal que $f(\overline{x}) = 0$. Em caso afirmativo, calcule o erro relativo percentual para $x_5$.
\item Mostre que a função $\varphi_2(x) = x - x \cdot f(x)$ é uma função de iteração para $f(x) = 0$.
\item Discuta se é possível garantir que a sequência dada por $x_k = \varphi_2(x_{k-1})$, convergirá para algum $\overline{x}$ tal que $f(\overline{x}) = 0$. Em caso afirmativo, calcule o erro relativo percentual para $x_5$.
\end{enumerate}
(obs: nos itens (b) e (d), arredonde cada $\varphi_i(x)$ com 4 dígitos após a vírgula)
\Answer

\begin{enumerate}
\item Basta observar que $
f(x) = 0
\Leftrightarrow
\ln(x) - \dfrac{1}{x} = 0
\Leftrightarrow
\ln(x) = \dfrac{1}{x}
\Leftrightarrow
x = \dfrac{1}{\ln(x)}$.

\item Como $\varphi_1^\prime(x) = -\frac{1}{x (\ln(x))^2}$, tem-se $|\varphi_1^\prime(x_0)| = -\frac{1}{2 (\ln(2))^2} \approx |-1,0407| > 1$, e não é possível garantir que a sequência gerada a partir desta aproximação inicial convergirá.

\item Basta observar que, para todo $x > 0$, tem-se
\[
f(x) = 0
\Leftrightarrow
x f(x) = 0
\Leftrightarrow
x f(x) + x = x
\Leftrightarrow
x = x - x f(x) = \varphi_2(x).
\]

\item Como $\varphi_2^\prime(x) = x - x f(x) = x - x\left(\ln(x) - \dfrac{1}{x}\right) = x+1-x\ln(x)$, para todo $x>0$, tem-se:
\[
\varphi_2^\prime(x)
= (x+1-x\ln(x))^\prime
= 1-1\ln(x)-x\frac{1}{x}
= -\ln(x).
\]
Assim,
\[
|\varphi_2^\prime(x)| < 1
\Leftrightarrow
|-\ln(x)| < 1
\Leftrightarrow
-1 < \ln(x) < 1
\Leftrightarrow
0,3679 \approx e^{-1} < x < e^1 \approx 2,7183.
\]
Considerando que $f$ e $\varphi_2$ são contínuas em $I = (e^{-1}, e )$ e que $f(e^{-1}) \approx -3,7183 < 0 < 0,6321 \approx f(e)$, segue do teorema de Bolzano que $f$ possui uma raiz $\overline{x}$ em $I$. Como $2 \in I$, conclui-se que a sequência $(x_k)_{k=0}^\infty$ definida por $x_0 = 2$ e $x_k = x_{k-1}+1-x_{k-1}\ln(x_{k-1})$ para $k \geq 1$, é convergente. Os primeiros termos dessa sequência são os seguintes (arredondados no quarto dígito decimal a cada iteração).

\begin{center}
\begin{tabular}{|c|r|r|r|r|r|r|r|}
\hline
$k$   & 0      & 1      & 2      & 3      & 4      & 5 \\ \hline
$x_k = \varphi_2(x_{k-1})$ & 2,0000 & 1,6137 & 1,8415 & 1,7171 & 1,7888 & 1,7486 \\ \hline
\end{tabular}
\end{center}
Logo, o erro relativo percentual em $x_5$ é $\varepsilon_{per} \approx |1,7486 - 1,7888 |/|1,7486| \approx 2,2990\%$.
\end{enumerate}

\Exercise[title={2,5}] Obtenha a única raiz do polinômio $p(x) = x^5-x^3+x+2$ com um erro absoluto estimado menor do que $10^{-3}$, utilizando o método de Newton (versão para polinômios).

(os resultados devem ser arredondados com 4 dígitos após a vírgula)
\Answer Considerando que $p(-2) = -24 < 0 < 1 = f(-1)$, há uma raiz $\overline{x} \in (-2, -1)$. Escolhendo a aproximação inicial $x_0 = -1$, tem-se:
\begin{center}
\begin{tabular}{|c|c|c|c|c|c|c|}
\hline
$x_0$ & $a_5$ & $a_4$ & $a_3$ & $a_2$ & $a_1$ & $a_0$ \\
\hline
-1 & 1 &  0 & -1 &  0 & 1 &  2 \\
\hline
   &   & -1 &  1 &  0 & 0 & -1 \\
\hline
$b_k$  &  1 & -1 &  0 & 0 & 1 & \textbf{1} \\
\hline
   &   & -1 &  2 & -2 & 2 & \\
\hline
   & 1 & -2 &  2 & -2 & \textbf{3} & \\
\hline
\end{tabular}
\end{center}
Logo, $p(x_0) = 1$ e $p^\prime(x_0) = 3$. Assim, $x_1 = -1 - 1/3 \approx -1,3333$.

\begin{center}
\begin{tabular}{|c|c|c|c|c|c|c|}
\hline
$x_0$ & $a_5$ & $a_4$ & $a_3$ & $a_2$ & $a_1$ & $a_0$ \\
\hline
-1,3333 & 1 & 0 & -1 & 0 & 1 & 2 \\
\hline
 &   & -1,3333 & 1,7777 & -1,0369 & 1,3825 & -3,1766 \\
\hline
$b_k$ & 1 & -1,3333 & 0,7777 & -1,0369 & 2,3825 & \textbf{-1,1766} \\
\hline
 &   & -1,3333 & 3,5554 & -5,7773 & 9,0854 &   \\
\hline
$c_k$ & 1 & -2,6666 & 4,3331 & -6,8142 & \textbf{11,4679} & \\
\hline
\end{tabular}
\end{center}
Logo, $p(x_1) = -1,1766$ e $p^\prime(x_1) = 11,4679$. Assim, $x_2 = -1,3333 +1,1766/11,4679 \approx -1,2307$.

\begin{center}
\begin{tabular}{|c|c|c|c|c|c|c|}
\hline
$x_0$ & $a_5$ & $a_4$ & $a_3$ & $a_2$ & $a_1$ & $a_0$ \\
\hline
-1,2307 & 1 & 0 & -1 & 0 & 1 & 2 \\
\hline
 &   & -1,2307 & 1,5146 & -0,6333 & 0,7794 & -2,1899 \\
\hline
$b_k$ & 1 & -1,2307 & 0,5146 & -0,6333 & 1,7794 & \textbf{-0,1899} \\
\hline
 &   & -1,2307 & 3,0292 & -4,3614 & 6,147 & \\
\hline
$c_k$ & 1 & -2,4614 & 3,5438 & -4,9947 & \textbf{7,9264} & \\
\hline
\end{tabular}
\end{center}
Logo, $p(x_2) = -0,1899$ e $p^\prime(x_2) = 7,9264$. Assim,
$x_3 = -1,2307 + 0,1899/7,9264 \approx -1,2067$.

\begin{center}
\begin{tabular}{|c|c|c|c|c|c|c|}
\hline
$x_0$ & $a_5$ & $a_4$ & $a_3$ & $a_2$ & $a_1$ & $a_0$ \\
\hline
-1,2067 & 1 & 0 & -1 & 0 & 1 & 2\\
\hline
&  & -1,2067 & 1,4561 & -0,5504 & 0,6642 & -2,0082\\
\hline
$b_k$ & 1 & -1,2067 & 0,4561 & -0,5504 & 1,6642 & \textbf{-0,0082} \\
\hline
&  & -1,2067 & 2,9122 & -4,0645 & 5,5688 & \\
\hline
$c_k$ & 1 & -2,4134 & 3,3683 & -4,6149 & \textbf{7,233} & \\
\hline
\end{tabular}
\end{center}
Logo, $p(x_3) = -0,1899$ e $p^\prime(x_3) = 7,9264$. Assim,
$x_4 = -1,2307 + 0,1899/7,9264 \approx -1,2067$.

Como $\varepsilon_{abs} \approx |x_4 - x_3| = 0 < 10^{-3}$, $x_4 = -1,2067$ é a aproximação procurada.

\Exercise[title={2,5}] Identifique um intervalo no qual a função $f(x) = \cos(\ln(x))$ tenha um zero $\overline{x} > 1/2$. Aplique o método da posição falsa para obter $x_k \approx \overline{x}$, de modo que $|f(x_k)| < 0,0001$.

(arredonde cada valor de $f(x)$ utilizado com 5 dígitos após a vírgula)
\Answer Atribuindo alguns valores para $x$, obtém-se:
\begin{center}
\begin{tabular}{|r|r|r|r|r|r|}
\hline
$x$    & 2   & 3   & 4    &  5 \\
\hline
$f(x)$ & 0,76924 & 0,45483 & 0,18346 & -0,03863 \\
\hline
\end{tabular}
\end{center}
Assim, pelo teorema de Bolzano, deve existir uma raiz de $f$ no intervalo $I = (4, 5)$.

Estas são as primeiras iterações do método da posição falsa partindo do intervalo inicial $a_0 = 4$ e $b_0 = 5$, com arredondamento no quinto dígito decimal a cada iteração:
\begin{center}
\begin{tabular}{|r|r|r|r|r|r|r|c|}
\hline
$k$ & $a_k$ & $x_k$ & $b_k$ & $f(a_k)$ & $f(x_k)$ & $f(b_k)$ & $f(a_k)\cdot f(x_k)$ \\
\hline
0 & 4 & 4,82606 & 5 & 0,18346 & -0,00323 & -0,03863 & $<0$ \\\hline
1 & 4 & 4,81177 & 4,82606 & 0,18346 & -0,00027 & -0,00323 & $<0$ \\\hline
2 & 4 & 4,81058 & 4,81177 & 0,18346 & -0,00002 & -0,00027 & $<0$ \\\hline
\end{tabular}
\end{center}
\medskip
Nesta etapa, obtém-se a aproximação $x_2 = 4,81058$, com $|f(x_2)| \approx 0,00002 < 0,0001$.


\Exercise[title={2,5}] Utilize o método da bisseção para obter uma raiz de $f(x) = \cos(x)$ no intervalo $[a_0, b_0] = [0, 3]$, com erro relativo percentual estimado inferior a $0,1\%$.

(arredonde cada valor com 5 dígitos após a vírgula)
\Answer Estas são as primeiras iterações do método da bisseção partindo do intervalo inicial $a_0 = 0$ e $b_0 = 3$, com arredondamento no quinto dígito decimal a cada iteração:
\begin{center}
\begin{tabular}{|r|r|r|r|r|r|r|c|r|}
\hline
$k$ & $a_k$ & $x_k$ & $b_k$ & $f(a_k)$ & $f(x_k)$ & $f(b_k)$ & $f(a_k)\cdot f(x_k)$ & $\varepsilon_{per}$ \\
\hline
0 & 0 & 1,5 & 3 & 1 & 0,07074 & -0,98999 & - & -\\\hline
1 & 1,5 & 2,25 & 3 & 0,07074 & -0,62817 & -0,98999 & $<0$ & 33,33333\%\\\hline
2 & 1,5 & 1,875 & 2,25 & 0,07074 & -0,29953 & -0,62817 & $<0$ & 20,00000\%\\\hline
3 & 1,5 & 1,6875 & 1,875 & 0,07074 & -0,11644 & -0,29953 & $<0$ & 11,11111\%\\\hline
4 & 1,5 & 1,59375 & 1,6875 & 0,07074 & -0,02295 & -0,11644 & $<0$ & 5,88235\%\\\hline
5 & 1,5 & 1,54688 & 1,59375 & 0,07074 & 0,02391 & -0,02295 & $>0$ & 3,02997\%\\\hline
6 & 1,54688 & 1,57032 & 1,59375 & 0,02391 & 0,00048 & -0,02295 & $>0$ & 1,49269\%\\\hline
7 & 1,57032 & 1,58204 & 1,59375 & 0,00048 & -0,01124 & -0,02295 & $<0$ & 0,74082\%\\\hline
8 & 1,57032 & 1,57618 & 1,58204 & 0,00048 & -0,00538 & -0,01124 & $<0$ & 0,37178\%\\\hline
9 & 1,57032 & 1,57325 & 1,57618 & 0,00048 & -0,00245 & -0,00538 & $<0$ & 0,18624\%\\\hline
10 & 1,57032 & 1,57179 & 1,57325 & 0,00048 & -0,00099 & -0,00245 & $<0$ & 0,09289\%\\\hline
\end{tabular}
\end{center}
\medskip
Nesta etapa, obtém-se a aproximação $x_{10} = 1,57179$, com $|x_{10}-x_9|/|x_{10}| \approx 0,09289\% < 0,1$.

\end{ExerciseList}

\vspace{0.5cm}
\begin{center}
BOA PROVA!
\end{center}

\newpage
\restoregeometry
\section*{Respostas}
\shipoutAnswer
\end{document}
