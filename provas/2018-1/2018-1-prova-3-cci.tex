\documentclass[12pt,a4paper]{article}
\usepackage{cmap} % Makes the PDF copiable. See http://tex.stackexchange.com/a/64198/25761
\usepackage[T1]{fontenc}
\usepackage[brazil]{babel}
\usepackage[utf8]{inputenc}
\usepackage{amsmath}
\usepackage{amsfonts}
\usepackage{amssymb}
\usepackage{amsthm}
\usepackage{textcomp} % \degree
\usepackage{gensymb} % \degree
\usepackage[usenames,svgnames,dvipsnames]{xcolor}
\usepackage{hyperref}
\usepackage{multicol}
\usepackage{graphicx}
\usepackage[margin=2cm]{geometry}
\usepackage{systeme}
\usepackage{icomma}

\hypersetup{
    colorlinks = true,
    allcolors = {blue}
}

% TODO: Consider using exsheets
% http://linorg.usp.br/CTAN/macros/latex/contrib/exsheets/exsheets_en.pdf
%
% http://ctan.org/tex-archive/macros/latex/contrib/exercise/
% Options: answerdelayed,lastexercise,noanswer
\usepackage[answerdelayed,lastexercise]{exercise}

\addto\captionsbrazil{%
\def\listexercisename{Lista de exerc\'icios}%
\def\ExerciseName{Exerc\'icio}%
\def\AnswerName{Solu\c{c}\~ao do exerc\'icio}%
\def\ExerciseListName{Ex.}%
\def\AnswerListName{Solu\c{c}\~ao}%
\def\ExePartName{Parte}%
\def\ArticleOf{de\ }%
}

\renewcommand{\ExerciseHeaderTitle}{(\ExerciseTitle)\ }
\renewcommand{\ExerciseListHeader}{%\ExerciseHeaderDifficulty%
\textbf{%\ExerciseListName\
\ExerciseHeaderNB.\ %
%\ --- \ 
\ExerciseHeaderTitle}%
%\ExerciseHeaderOrigin
\ignorespaces}
\renewcommand{\AnswerListHeader}{\textbf{\ExerciseHeaderNB.\ (\AnswerListName)\ }}

\newcommand{\fixme}{{\color{red}(...)}}

\renewcommand{\theenumi}{\alph{enumi}}
\renewcommand\labelenumi{(\theenumi) }

\newcommand*\tipo{Prova III}
\newcommand*\turma{CCI122-03U}
\newcommand*\disciplina{ANN0001}
\newcommand*\eu{Helder G. G. de Lima}
\newcommand*\data{03/07/2018}

\author{\eu}
\title{\tipo - \disciplina}
\date{\data}

\begin{document}
\thispagestyle{empty}
\newgeometry{margin=2cm,bottom=0.5cm}
\begin{center}
\includegraphics[width=9.0cm]{marca} \\
\textbf{\tipo\ (\disciplina / \turma)} \\
Prof. \eu\footnote{
Este é um material de acesso livre distribuído sob os termos da licença \href{https://creativecommons.org/licenses/by-sa/4.0/deed.pt_BR}{Creative Commons Atribuição-CompartilhaIgual 4.0 Internacional}}
\end{center}

\noindent Nome do(a) aluno(a): \underline{\hspace{9,7cm}} Data: \underline{\data}

%\section*{Instruções}
\begin{center}\fbox{
\begin{minipage}{14cm}

{\footnotesize
\begin{itemize}
\renewcommand{\theenumi}{\Roman{enumi}}
\item Identifique-se em todas as folhas.
\item Mantenha o celular e os demais equipamentos eletrônicos desligados durante a prova.
\item Justifique cada resposta com cálculos ou argumentos baseados na teoria estudada.
\item Sempre que calcular o valor de uma das funções consideradas em um ponto $x$, arredonde o resultado para o número de dígitos especificado, e só então use esse valor (arredondado) nas fórmulas dos métodos iterativos.
\item Resolva apenas os itens de que precisar para somar 10,0 pontos.
\end{itemize}
}

\end{minipage}
}
\end{center}

%\section*{Questões}
\begin{ExerciseList}
\Exercise[title={1,0}]
Explique o funcionamento e as vantagens do método de Newton-Cotes adaptável.
\Answer
Neste método, depois de aproximar o valor de $\int_a^b f(x)\,dx$, é feita uma estimativa do erro cometido nesta aproximação. Se o erro é maior do que o desejado, o intervalo é subdividido ao meio, e são calculadas aproximações individuais para $\int_a^{\frac{a+b}{2}} f(x)\,dx$ e $\int_{\frac{a+b}{2}}^b f(x)\,dx$. Em cada caso, o erro cometido é avaliado, e usado como critério para decidir se algum dos intervalos (ou ambos) precisa ser dividido ao meio novamente. O processo se repete até que a soma das aproximações das integrais nos subintervalos considerados esteja próxima o bastante do valor exato da integral em $[a,b]$. Uma vantagem deste tipo de abordagem é que ele evita calcular $f(x)$ desnecessariamente em regiões onde é possível alcançar uma boa aproximação sem usar muitos pontos.

\Exercise[title={3,0}]
Seja $f(x) = 1-x^4$. Se for utilizada a regra
$1/3$ de Simpson com repetição, qual será o menor número de \textbf{pontos} distintos em que $f$ precisará ser calculada para que o erro relativo percentual ao aproximar $\int_{-1}^1 f(x)\, dx$ seja de no máximo $1\%$?

{\color{blue} \textit{(Utilize números decimais com 4 dígitos após a vírgula)}}
\Answer O valor exato da integral é
\[
\int_{-1}^1 1-x^4\, dx
= \left(x-\frac{x^5}{5} \right)\Big|_{-1}^1 = \frac{8}{5} = 1,6.
\]
As aproximações obtidas pelo método 1/3 de Simpson são as seguintes:
\medskip
\begin{center}
\begin{tabular}{|c|c|c|c|}
\hline 
Subintervalos & Pontos & Aproximação & Erro (\%) \\ 
\hline 
1 & 3 & $1,3333$ & $16,6688$ \\ 
\hline 
2 & 5 & $1,5833$ & $1,0438$ \\ 
\hline 
3 & 7 & $1,5967$ & $0,2063$ \\ 
\hline 
\end{tabular}
\end{center}
\medskip
Então é preciso calcular $f$ em pelo menos 7 pontos para que o erro relativo percentual não ultrapasse $1\%$.

\Exercise[title={3,0}] Considerando que
$\int_{-1}^3 \sqrt[3]{x}\, dx
= \frac{9 \; \sqrt[3]{3} - 3}{4}
\approx 2,4951$
e que
$\int_{-1}^5 \sqrt[3]{x}\, dx
= \frac{15 \; \sqrt[3]{5} - 3}{4}
\approx 5,6624$, verifique que o erro relativo da aproximação de $\int_{-1}^3 \sqrt[3]{x}\, dx$ pela regra de Gauss-Legendre com 3 pontos é cerca de um terço do erro relativo da aproximação de $\int_{-1}^5 \sqrt[3]{x}\, dx$ pelo mesmo método.

{\color{blue} \textit{(Utilize números decimais com 4 dígitos após a vírgula)}}
\Answer Considerando $x = 2t + 1$, tem-se $
    \int_{-1}^3 \sqrt[3]{x}\, dx
= 2 \int_{-1}^1 \sqrt[3]{2t+1} \,dt
$. Consequentemente, o valor aproximado da integral pode ser calculado pelo método de Gauss-Legendre com 3 pontos com o auxílio da seguinte tabela:
\medskip
\begin{center}
\begin{tabular}{|r|r|r|r|r|}
\hline 
$x_i$ & $t_i = 2x_i+1$ & $\sqrt[3]{t_i}$ & $w_i$ & $w_i \sqrt[3]{t_i}$ \\ 
\hline 
-0,7746 & -0,5492 & -0,8189 & 0,5556 & -0,4550 \\ 
\hline 
 0,0000 &  1,0000 &  1,0000 & 0,8889 &  0,8889 \\ 
\hline 
 0,7746 &  2,5492 &  1,3661 & 0,5556 &  0,7589 \\ 
\hline 
\end{tabular}
\end{center}
\medskip
Assim,
\[
\int_{-1}^3 \sqrt[3]{x}\, dx
\approx 2 \cdot \left( -0,4550 + 0,8889 + 0,7589\right)
= 2,3856,
\]
e o erro relativo desta aproximação é $\varepsilon_1 = -0,0439$.
Analogamente, tomando $x = 3t + 2$, tem-se
$
    \int_{-1}^3 \sqrt[3]{x}\, dx
= 3 \int_{-1}^1 \sqrt[3]{3t+2} \,dt
$. Consequentemente, o valor aproximado da integral pode ser calculado pelo método de Gauss-Legendre com 3 pontos com o auxílio da seguinte tabela:
\medskip
\begin{center}
\begin{tabular}{|r|r|r|r|r|}
\hline 
$x_i$ & $t_i = 3x_i+2$ & $\sqrt[3]{t_i}$ & $w_i$ & $w_i \sqrt[3]{t_i}$ \\ 
\hline 
-0,7746 & -0,3238 & -0,6867 & 0,5556 & -0,3815 \\ 
\hline 
 0,0000 &  2,0000 &  1,2599 & 0,8889 &  1,1199 \\ 
\hline 
 0,7746 &  4,3238 &  1,6291 & 0,5556 &  0,9051 \\ 
\hline 
\end{tabular}
\end{center}
\medskip
Assim,
\[
\int_{-1}^3 \sqrt[3]{x}\, dx
\approx 3 \cdot \left( -0,3815 + 1,1199 + 0,9051 \right)
= 4,9305,
\]
e o erro relativo desta aproximação é $\varepsilon_2 = -0,1293$. Comparando-se os erros relativos de ambas as aproximações, resulta que $\frac{\varepsilon_1}{\varepsilon_2} = \frac{-0,0439}{-0,1293} = 0,3395 \approx 1/3$.

\Exercise[title={3,0}] Dados os pontos $x_0 = 0$, $x_1 = 1$ e $x_2 = 4$, obtenha os pesos $w_i$ para que a aproximação
\[
\int_{0}^{4} f(x)\, dx \approx w_0 f(0) + w_1 f(1) + w_2 f(4)
\]
seja exata para polinômios de grau menor ou igual a dois. Utilize a regra obtida para calcular $\int_{0}^{4} g(x)$ considerando que $g(0) = 2$, $g(1) = 0$ e $g(4) = 3$.
\Answer Se a aproximação
\[
\int_{0}^{4} f(x)\, dx \approx w_0 f(0) + w_1 f(1) + w_2 f(4)
\]
for exata para polinômios de grau menor ou igual a dois então, em particular, ela será exata para os polinômios $1$, $x$ e $x^2$, isto é,
\begin{align*}
\int_{0}^{4} 1\, dx & = 4 = w_0 \cdot 1 + w_1 \cdot 1 + w_2 \cdot 1 \\
\int_{0}^{4} x\, dx & = 8 = w_0 \cdot 0 + w_1 \cdot 1 + w_2 \cdot 4 \\
\int_{0}^{4} x^2\, dx & = \frac{64}{3} = w_0 \cdot 0 + w_1 \cdot 1 + w_2 \cdot 16
\end{align*}
Resolvendo o sistema, chega-se a $w_0 = -2/3$, $w_1 = 32/9$ e $w_2 = 10/9$. Em particular,
\[
\int_{0}^{4} g(x)\, dx \approx -0,6667 \cdot 2 + 3,5556 \cdot 0 + 1,1111 \cdot 3 = 1,9999.
\]
\Exercise[title={3,0}]
Em relação às soluções aproximadas do problema de valor inicial
\[
\begin{cases}
y^\prime(x) = x - y(x), \quad x \in [0,1]\\
y(0) = 2
\end{cases}
\]
pelos métodos de Euler explícito e implícito, com passo $h=0,25$, verifique se é correto afirmar que o maior erro absoluto (em módulo) em ambos os casos ocorre quando $x = 1$, considerando que a solução exata é $y(x) = 3e^{-x}+x-1$.

{\color{blue} \textit{(Utilize números decimais com 3 dígitos após a vírgula)}}
\Answer
Denotando $f(x,y) = x-y$ e $h=0,25$, pode-se expressar a fórmula do método de Euler explícito da seguinte forma:
\[
y_n
= y_{n-1} + h f(x_{n-1}, y_{n-1})
= y_{n-1} + 0,25( x_{n-1} - y_{n-1} )
= 0,25 x_{n-1} + 0,75 y_{n-1}.
\]
Disto resulta que os valores obtidos a cada passo são os seguintes:
\medskip
\begin{center}
\begin{tabular}{|c|c|r|c|c|}
\hline
$n$ & $x_n$ & $y_n = 0,25 x_{n-1} + 0,75 y_{n-1}, n\geq 1$ & $y_{exato}(x_n)$ & $\varepsilon_n = y_n-y_{exato}(x_n)$ \\ \hline\hline
$0$ & $0,000$ & $2,000$                                       & $2,000$ & $0,000$ \\ \hline
$1$ & $0,250$ & $0,25 \cdot 0,000 + 0,75 \cdot 2,000 = 1,500$ & $1,586$ & $0,086$ \\ \hline
$2$ & $0,500$ & $0,25 \cdot 0,250 + 0,75 \cdot 1,500 = 1,188$ & $1,320$ & $0,132$ \\ \hline
$3$ & $0,750$ & $0,25 \cdot 0,500 + 0,75 \cdot 1,188 = 1,016$ & $1,167$ & $0,151$ \\ \hline
$4$ & $1,000$ & $0,25 \cdot 0,750 + 0,75 \cdot 1,016 = 0,950$ & $1,104$ & $\textbf{0,154}$ \\ \hline
\end{tabular}
\end{center}
\medskip
Em particular, o maior erro absoluto ocorre no ponto $x = 1$.

No método de Euler implícito, por sua vez, utiliza-se a relação $y_n = y_{n-1} + h f(x_n, y_n)$ que, no problema considerado, pode ser reescrita  de forma equivalente como:
\[
y_n
= y_{n-1} + 0,25( x_n - y_n )
\Leftrightarrow
1,25y_n = 0,25x_n + y_{n-1}
\Leftrightarrow
y_n = 0,2 x_n + 0,8 y_{n-1}.
\]
Consequentemente, os valores obtidos a cada passo são:
\medskip
\begin{center}
\begin{tabular}{|c|c|r|r|r|}
\hline
$n$ & $x_n$ & $y_n= 0,2 x_n + 0,8 y_{n-1}, n\geq 1$ & $y_{exato}(x_n)$ & $\varepsilon_n = y_n-y_{exato}(x_n)$ \\ \hline\hline
$0$ & $0,000$ & $2,000$                                         & $2,000$ & $0,000$ \\ \hline
$1$ & $0,250$ & $0,200 \cdot 0,250 + 0,800 \cdot 2,000 = 1,650$ & $1,586$ & $-0,064$ \\ \hline
$2$ & $0,500$ & $0,200 \cdot 0,500 + 0,800 \cdot 1,650 = 1,420$ & $1,320$ & $-0,100$ \\ \hline
$3$ & $0,750$ & $0,200 \cdot 0,750 + 0,800 \cdot 1,420 = 1,286$ & $1,167$ & $-0,119$ \\ \hline
$4$ & $1,000$ & $0,200 \cdot 1,000 + 0,800 \cdot 1,286 = 1,229$ & $1,104$ & $\textbf{-0,125}$ \\ \hline
\end{tabular}
\end{center}
\medskip
Novamente, o maior erro absoluto (em módulo) ocorre no ponto $x = 1$. Portanto a afirmação é correta.
\end{ExerciseList}

\vspace{0.4cm}
\begin{center}
BOA PROVA E BOAS FÉRIAS!
\end{center}

\newpage
\restoregeometry
\section*{Respostas}
\shipoutAnswer
\end{document}
