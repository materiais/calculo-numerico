\documentclass[12pt,a4paper]{article}
\usepackage{cmap} % Makes the PDF copiable. See http://tex.stackexchange.com/a/64198/25761
\usepackage[T1]{fontenc}
\usepackage[brazil]{babel}
\usepackage[utf8]{inputenc}
\usepackage{amsmath}
\usepackage{amsfonts}
\usepackage{amssymb}
\usepackage{amsthm}
\usepackage{textcomp} % \degree
\usepackage{gensymb} % \degree
\usepackage[usenames,svgnames,dvipsnames]{xcolor}
\usepackage{hyperref}
\usepackage{multicol}
\usepackage{graphicx}
\usepackage[margin=2cm]{geometry}
\usepackage{systeme}
\usepackage{icomma}

\hypersetup{
    colorlinks = true,
    allcolors = {blue}
}

% TODO: Consider using exsheets
% http://linorg.usp.br/CTAN/macros/latex/contrib/exsheets/exsheets_en.pdf
%
% http://ctan.org/tex-archive/macros/latex/contrib/exercise/
% Options: answerdelayed,lastexercise,noanswer
\usepackage[answerdelayed,lastexercise]{exercise}

\addto\captionsbrazil{%
\def\listexercisename{Lista de exerc\'icios}%
\def\ExerciseName{Exerc\'icio}%
\def\AnswerName{Solu\c{c}\~ao do exerc\'icio}%
\def\ExerciseListName{Ex.}%
\def\AnswerListName{Solu\c{c}\~ao}%
\def\ExePartName{Parte}%
\def\ArticleOf{de\ }%
}

\renewcommand{\ExerciseHeaderTitle}{(\ExerciseTitle)\ }
\renewcommand{\ExerciseListHeader}{%\ExerciseHeaderDifficulty%
\textbf{%\ExerciseListName\
\ExerciseHeaderNB.\ %
%\ --- \ 
\ExerciseHeaderTitle}%
%\ExerciseHeaderOrigin
\ignorespaces}
\renewcommand{\AnswerListHeader}{\textbf{\ExerciseHeaderNB.\ (\AnswerListName)\ }}

\newcommand{\fixme}{{\color{red}(...)}}

\renewcommand{\theenumi}{\alph{enumi}}
\renewcommand\labelenumi{(\theenumi) }

\newcommand*\tipo{Prova III}
\newcommand*\turma{CCI122-03U}
\newcommand*\disciplina{ANN0001}
\newcommand*\eu{Helder G. G. de Lima}
\newcommand*\data{03/07/2018}

\author{\eu}
\title{\tipo - \disciplina}
\date{\data}

\begin{document}
\thispagestyle{empty}
\newgeometry{margin=2cm,bottom=0.5cm}
\begin{center}
\includegraphics[width=9.0cm]{marca} \\
\textbf{\tipo\ (\disciplina / \turma)} \\
Prof. \eu\footnote{
Este é um material de acesso livre distribuído sob os termos da licença \href{https://creativecommons.org/licenses/by-sa/4.0/deed.pt_BR}{Creative Commons Atribuição-CompartilhaIgual 4.0 Internacional}}
\end{center}

\noindent Nome do(a) aluno(a): \underline{\hspace{9,7cm}} Data: \underline{\data}

%\section*{Instruções}
\begin{center}\fbox{
\begin{minipage}{14cm}

{\footnotesize
\begin{itemize}
\renewcommand{\theenumi}{\Roman{enumi}}
\item Identifique-se em todas as folhas.
\item Mantenha o celular e os demais equipamentos eletrônicos desligados durante a prova.
\item Justifique cada resposta com cálculos ou argumentos baseados na teoria estudada.
\item Sempre que calcular o valor de uma das funções consideradas em um ponto $x$, arredonde o resultado para o número de dígitos especificado, e só então use esse valor (arredondado) nas fórmulas dos métodos iterativos.
\item Resolva apenas os itens de que precisar para somar 10,0 pontos.
\end{itemize}
}

\end{minipage}
}
\end{center}

%\section*{Questões}
\begin{ExerciseList}
\Exercise[title={1,0}]
Explique o funcionamento e as vantagens do método de Newton-Cotes adaptável.
\Answer \fixme

\Exercise[title={3,0}] Seja $f(x) = 1-x^4$. Se for utilizada a regra
$1/3$ de Simpson com repetição, qual será o menor número de \textbf{pontos} distintos em que $f$ precisará ser calculada para que o erro relativo percentual ao aproximar $\int_{-1}^1 f(x)\, dx$ seja de no máximo $1\%$?

{\color{blue} \textit{(Utilize números decimais com 4 dígitos após a vírgula)}}
\Answer \fixme


\Exercise[title={3,0}] Considerando que
$\int_{-1}^3 \sqrt[3]{x}\, dx
= \frac{9 \; \sqrt[3]{3} - 3}{4}
\approx 2,4951$
e que
$\int_{-1}^5 \sqrt[3]{x}\, dx
= \frac{15 \; \sqrt[3]{5} - 3}{4}
\approx 5,6624$, verifique que o erro relativo da aproximação de $\int_{-1}^3 \sqrt[3]{x}\, dx$ pela regra de Gauss-Legendre com 3 pontos é cerca de um terço do erro relativo da aproximação de $\int_{-1}^5 \sqrt[3]{x}\, dx$ pelo mesmo método.

{\color{blue} \textit{(Utilize números decimais com 4 dígitos após a vírgula)}}
\Answer \fixme

\Exercise[title={3,0}] Dados os pontos $x_0 = 0$, $x_1 = 1$ e $x_2 = 4$, obtenha os pesos $w_i$ para que a aproximação
\[
\int_{0}^{4} f(x) \approx w_0 f(0) + w_1 f(1) + w_2 f(4)
\]
seja exata para polinômios de grau menor ou igual a dois. Utilize a regra obtida para calcular $\int_{0}^{4} g(x)$ considerando que $g(0) = 2$, $g(1) = 0$ e $g(4) = 3$.
\Answer \fixme

\Exercise[title={3,0}]
Em relação às soluções aproximadas do problema de valor inicial
\[
\begin{cases}
y^\prime(x) = x - y(x), \quad x \in [0,1]\\
y(0) = 2
\end{cases}
\]
pelos métodos de Euler explícito e implícito, com passo $h=0,25$, verifique se é correto afirmar que o maior erro absoluto em ambos os casos ocorre quando $x = 1$, considerando que a solução exata é $y(x) = 3e^{-x}+x-1$.

{\color{blue} \textit{(Utilize números decimais com 3 dígitos após a vírgula)}}
\Answer \fixme
\end{ExerciseList}

\vspace{0.4cm}
\begin{center}
BOA PROVA E BOAS FÉRIAS!
\end{center}

%\newpage
\restoregeometry
%\section*{Respostas}
%\shipoutAnswer
\end{document}
