\documentclass[12pt,a4paper]{article}
\usepackage{cmap} % Makes the PDF copiable. See http://tex.stackexchange.com/a/64198/25761
\usepackage[T1]{fontenc}
\usepackage[brazil]{babel}
\usepackage[utf8]{inputenc}
\usepackage{amsmath}
\usepackage{amsfonts}
\usepackage{amssymb}
\usepackage{amsthm}
\usepackage{textcomp} % \degree
\usepackage{gensymb} % \degree
\usepackage[usenames,svgnames,dvipsnames]{xcolor}
\usepackage{hyperref}
\usepackage{multicol}
\usepackage{graphicx}
\usepackage[margin=2cm]{geometry}
\usepackage{systeme}
\usepackage{icomma}

\hypersetup{
    colorlinks = true,
    allcolors = {blue}
}

% TODO: Consider using exsheets
% http://linorg.usp.br/CTAN/macros/latex/contrib/exsheets/exsheets_en.pdf
%
% http://ctan.org/tex-archive/macros/latex/contrib/exercise/
% Options: answerdelayed,lastexercise,noanswer
\usepackage[answerdelayed,lastexercise]{exercise}

\addto\captionsbrazil{%
\def\listexercisename{Lista de exerc\'icios}%
\def\ExerciseName{Exerc\'icio}%
\def\AnswerName{Solu\c{c}\~ao do exerc\'icio}%
\def\ExerciseListName{Ex.}%
\def\AnswerListName{Solu\c{c}\~ao}%
\def\ExePartName{Parte}%
\def\ArticleOf{de\ }%
}

\renewcommand{\ExerciseHeaderTitle}{(\ExerciseTitle)\ }
\renewcommand{\ExerciseListHeader}{%\ExerciseHeaderDifficulty%
\textbf{%\ExerciseListName\
\ExerciseHeaderNB.\ %
%\ --- \
\ExerciseHeaderTitle}%
%\ExerciseHeaderOrigin
\ignorespaces}
\renewcommand{\AnswerListHeader}{\textbf{\ExerciseHeaderNB.\ (\AnswerListName)\ }}

\newcommand*\R{\mathbb{R}}

\renewcommand{\theenumi}{\alph{enumi}}
\renewcommand\labelenumi{(\theenumi) }

\newcommand*\tipo{Prova II}
\newcommand*\turma{CCI122-03U}
\newcommand*\disciplina{ANN0001}
\newcommand*\eu{Helder G. G. de Lima}
\newcommand*\data{22/05/2018}

\author{\eu}
\title{\tipo - \disciplina}
\date{\data}

\begin{document}
\thispagestyle{empty}
\newgeometry{margin=2cm,bottom=0.5cm}
\begin{center}
\includegraphics[width=9.0cm]{marca} \\
\textbf{\tipo\ (\disciplina / \turma)} \\
Prof. \eu\footnote{
Este é um material de acesso livre distribuído sob os termos da licença \href{https://creativecommons.org/licenses/by-sa/4.0/deed.pt_BR}{Creative Commons Atribuição-CompartilhaIgual 4.0 Internacional}}
\end{center}

\noindent Nome do(a) aluno(a): \underline{\hspace{9,7cm}} Data: \underline{\data}

%\section*{Instruções}
\begin{center}\fbox{
\begin{minipage}{14cm}

{\footnotesize
\begin{itemize}
\renewcommand{\theenumi}{\Roman{enumi}}
\item Identifique-se em todas as folhas.
\item Mantenha o celular e os demais equipamentos eletrônicos desligados durante a prova.
\item Justifique cada resposta com cálculos ou argumentos baseados na teoria estudada.
\item Sempre que calcular o valor de uma das funções consideradas em um ponto $x$, arredonde o resultado para o número de dígitos especificado, e só então use esse valor (arredondado) nas fórmulas dos métodos iterativos.
\item Resolva apenas os itens de que precisar para somar 10,0 pontos.
\end{itemize}
}

\end{minipage}
}
\end{center}

%\section*{Questões}
\begin{ExerciseList}
\Exercise[title={2,0}]
Dê evidências teóricas de que o método de Gauss-Seidel convergirá, se for aplicado a:
\[
\begin{cases}
8x_1-\phantom{1}x_2\phantom{+ 0 x_3 } &= 2\\
\phantom{1}x_1-8x_2+2x_3 &= 2\\
\phantom{0x_1+1}x_2-8x_3 &= 2.
\end{cases}
\]
Obtenha uma solução aproximada, com erro percentual relativo de no máximo $1\%$, considerando $X^{(0)} = (-1,0000, 0,0000, 1,0000)$ e arredondando cada resultado com \textbf{4 dígitos} após a vírgula.
\Answer Considerando que
$|8| > |-1|+|0|$,
$|-8| > |1|+|2|$ e
$|-8| > |0|+|1|$
a matriz de coeficientes do sistema é estritamente diagonalmente dominante, e portanto o método convergirá, qualquer que seja a aproximação inicial escolhida. As equações utilizadas no método de Gauss-Seidel são as seguintes:
\[
\begin{cases}
x_1^{(k)} = (2 + x_2^{(k-1)})/8\\
x_2^{(k)} = (-2 + x_1^{(k)} + 2x_3^{(k-1)})/8\\
x_3^{(k)} = (-2 + x_2^{(k)})/8
\end{cases}
\]
Os valores obtidos a cada iteração são os seguintes:
\medskip
\begin{center}
\begin{tabular}{|c|r|r|r|r|r|}
\hline
$\boldsymbol{k}$     & 0 & 1 & 2 & 3 & 4\\
\hline
$\boldsymbol{x_1^{(k)}}$ & -1,0000 &  0,2500 &  0,2539 &  0,2150 &  0,2132 \\
\hline
$\boldsymbol{x_2^{(k)}}$ &  0,0000 &  0,0313 & -0,2798 & -0,2944 & -0,2951 \\
\hline
$\boldsymbol{x_3^{(k)}}$ &  1,0000 & -0,2461 & -0,2850 & -0,2868 & -0,2869 \\
\hline
$\varepsilon_{abs}$ & - & 1,2500 & 0,3111 & 0,0389 & 0,0018 \\
\hline
$\varepsilon_{per}$ & - & 500,00\% & 109,16\% & 13,21\% & 0,61\% \\
\hline
\end{tabular}
\end{center}
\medskip


\Exercise%[title={2,5}]
Considere o seguinte sistema não linear:
$\begin{cases}
10x_1-x_1^2-x_2^2 = 1\\
4x_1-3x_2^2-12x_2 = 0.
\end{cases}$
\begin{enumerate}
\item \textbf{(0,5)} Reescreva o sistema na forma de um problema de ponto fixo, $X = \varphi(X)$.\\
(obs.: $X = (x_1, x_2) \in \R^2$ e $\varphi(X) = (\varphi_1(X), \varphi_2(X)) \in \R^2$).
\item \textbf{(1,5)} Seja $Q = \{(x_1, x_2) \in \R^2 \mid 0 \leq x_i \leq 1 \}$. Verifique se $\varphi(X) \in Q$ sempre que $X \in Q$. O que isso diz sobre a existência de um ponto fixo de $\varphi$ em $Q$?
\item \textbf{(2,0)} Obtenha uma solução aproximada $X^{(3)}$ do sistema, pelo método de iteração de ponto fixo, partindo da aproximação inicial $X^{(0)}=(x_1^{(0)}, x_2^{(0)}) = (0,0000, 0,0000)$.\\
(arredonde cada iteração com \textbf{4 dígitos} após a vírgula)
\end{enumerate}
\Answer
\begin{enumerate}
\item O sistema é equivalente a
\[
\begin{cases}
x_1 = (x_1^2+x_2^2+1)/10\\
x_2 = \dfrac{4x_1}{3x_2+12}.
\end{cases}
\]
Assim, pode-se considerar $\varphi(x_1, x_2) = \left( \varphi_1(x_1, x_2), \varphi_2(x_1, x_2) \right) = \left( \dfrac{x_1^2+x_2^2+1}{10}, \dfrac{4x_1}{3x_2+12} \right)$.

Alternativamente, a função $\varphi_2$ poderia ser definida por:
\begin{multicols}{3}
\begin{itemize}
\item $\varphi_2(x_1, x_2) = \frac{4x_1-3x_2^2}{12}$
\item $\varphi_2(x_1, x_2) = \frac{4x_1-12x_2}{3x_2}$
\item $\varphi_2(x_1, x_2) = \sqrt{\frac{4x_1-12x_2}{3}}$
\end{itemize}
\end{multicols}
\item Para quaisquer $x_1$ e $x_2$ tais que $0 \leq x_1 \leq 1$ e $0 \leq x_2 \leq 1$, tem-se:
\begin{itemize}
\item $\varphi_1(x_1, x_2) \in [0, 1]$, pois $\dfrac{1}{10} \leq \dfrac{x_1^2+x_2^2+1}{10} \leq \dfrac{3}{10}$ (verifique!)
\item $\varphi_2(x_1, x_2) \in [0, 1]$, pois $0 \leq \dfrac{4x_1}{3x_2+12} \leq \dfrac{1}{3}$ (verifique!)
\end{itemize}

Assim, $\varphi(x_1, x_2) \in Q$ sempre que $(x_1, x_2) \in Q$.

\textbf{Observação:} se $\varphi_2$ fosse uma das outras escolhas mencionadas no item anterior, não seria verdade que $\varphi(x_1, x_2) \in Q$ sempre que $(x_1, x_2) \in Q$ (considere por exemplo quais seriam as imagens dos pontos $(1,0)$ e $(0,1)$) e não haveria a garantida de que há um ponto fixo.
\item Estas são as aproximações obtidas nas primeiras iterações do método do ponto fixo:
\begin{center}
\begin{tabular}{|c|r|r|r|r|}
\hline
$\boldsymbol{k}$     & 0 & 1 & 2 & 3 \\
\hline
$\boldsymbol{x_1^{(k)}}$ & 0,0000 & 0,1000 & 0,1010 & 0,1011\\
\hline
$\boldsymbol{x_2^{(k)}}$ & 0,0000 & 0,0000 & 0,0333 & 0,0334\\
\hline
\end{tabular}
\end{center}
\end{enumerate}

\Exercise%[title={4,0}]
Ao baixar algumas ferramentas de desenvolvimento a partir da internet, um programador percebeu que a velocidade de download estava diminuindo com o passar do tempo. Uma hora após o início do download, só tinham sido baixados $450$ MB. Depois de mais uma hora, o total baixado ainda era de $800$ MB, e nas duas horas seguintes, finalmente chegou a $1050$ MB e $1200$ MB, respectivamente. Sabendo que a transferência de dados durou 5 horas, e considerando o quanto havia sido baixado após 0, 1, 2, 3 e  4 horas, estime o tamanho das ferramentas baixadas:
\begin{enumerate}
\item \textbf{(2,0)} Usando o polinômio interpolador obtido por diferenças divididas (forma de Newton).
\item \textbf{(2,0)} Utilizando a reta que melhor se ajusta, por mínimos quadrados.
\end{enumerate}
\Answer
\begin{enumerate}
\item A partir dos pontos dados, obtém-se:
\[
\begin{array}{ccccccc}
x_i
& y_i=f[x_i]
& f[x_i,x_{i+1}]
& f[x_i,x_{i+1},x_{i+2}]
& f[x_i,\ldots,x_{i+3}]
& f[x_i,\ldots,x_{i+4}]\\
0 & \mathbf{0} \\
  & & \mathbf{450} \\
1 & 450 & & \mathbf{-50} \\
  & & 350 & & \mathbf{0} \\
2 & 800 & & -50 & & \mathbf{0}. \\
  & & 250 & & 0 \\
3 & 1050 & & -50\\
  & & 150\\
4 & 1200
\end{array}
\]

Então:
\begin{align*}
p(x)
&=0
 +450 x
 -50  x(x-1)
 +0   x(x-1)(x-1)
 +0   x(x-1)(x-1)(x-3)\\
& = -50x^2 + 500x.
\end{align*}
Usando este polinômio para estimar o valor pedido, resulta que:
\[
p(5)
=-50\cdot 5^2 + 500\cdot 5
=-50\cdot 25 + 2500
= 1250.
\]
\item Sejam $P_0 = (0,0)$, $P_1 = (1,450)$, $P_2 = (2,800)$, $P_3 = (3,1050)$ e $P_4 = (4,1200)$ e denote $g_0(x) = 1$, $g_1(x) = x$. Para encontrar uma função da forma $r(x) = a_0 g_0(x) + a_1 g_1(x)$ que melhor se ajusta aos pontos $P_i = (x_i,y_i)$, para $0 \leq i \leq 4$, basta resolver o sistema $A^T A X = A^T B$, em que
\[
A
= \begin{bmatrix}
g_0(x_1) & g_1(x_0) \\
\vdots & \vdots\\
g_0(x_5) & g_1(x_4) \\
\end{bmatrix}
= \begin{bmatrix}
1 & x_0 \\
\vdots & \vdots\\
1 & x_4 \\
\end{bmatrix},
\quad
X =
\begin{bmatrix}
a_0\\a_1
\end{bmatrix},
\quad
B = \begin{bmatrix}
y_0 \\
\vdots \\
y_4 \\
\end{bmatrix},
%=
%\begin{bmatrix}
%1 \\ 3 \\ 2 \\ 4 \\ 2
%\end{bmatrix},
\]

\[
A^T A
= \begin{bmatrix}
 5                & \sum_{i=1}^5 x_i \\
\sum_{i=1}^5 x_i  & \sum_{i=1}^5 x_i^2
\end{bmatrix}
=\begin{bmatrix}
1 & 1 & 1 & 1 & 1 \\
0 & 1 & 2 & 3 & 4
\end{bmatrix}
\cdot
\begin{bmatrix}
  1 & 0 \\
  1 & 1 \\
  1 & 2 \\
  1 & 3 \\
  1 & 4
\end{bmatrix}
=\begin{bmatrix}
5 & 10 \\ 10 & 30
\end{bmatrix},
\]
e
\[
A^T B
= \begin{bmatrix}
 \sum_{i=1}^5 y_i \\
 \sum_{i=1}^5 x_i y_i
\end{bmatrix}
= \begin{bmatrix}
1 & 1 & 1 & 1 & 1 \\
0 & 1 & 2 & 3 & 4
\end{bmatrix}
\begin{bmatrix}
0 \\ 450 \\ 800 \\ 1050 \\ 1200
\end{bmatrix}
= \begin{bmatrix}
3500 \\ 10000
\end{bmatrix}.
\]
Então,
$
A^T A X = A^T B
\Leftrightarrow
\begin{bmatrix}
5 & 10 \\ 10 & 30
\end{bmatrix}
\cdot
\begin{bmatrix}
a_0\\
a_1
\end{bmatrix}
=
\begin{bmatrix}
3500 \\ 10000
\end{bmatrix}
\Leftrightarrow
\begin{bmatrix}
a_0\\
a_1
\end{bmatrix}
=
\begin{bmatrix}
100\\
300
\end{bmatrix}.
$

Portanto, a reta é $r(x) = 100 + 300x$ e $r(5) = 100 + 300 \cdot 5 = 1600$.
\end{enumerate}

\Exercise[title={2,0}] Obtenha, pelo método de Lagrange, o polinômio $p(x)$ que interpola $f(x) = x^4+2x-1$ em $x_0 = -1$, $x_1 = 0$ e $x_2 = 1$. Estime o erro absoluto máximo ao aproximar $f(x)$ por $p(x)$ no intervalo $[-1, 1]$.
\Answer Considerando que $f(-1) = -2$, $f(0) = -1$ e $f(1) = 2$, o método de Lagrange permite que o polinômio que interpola $f$ nestes pontos seja descrito da seguinte forma:
\begin{align*}
p(x)
& = -2 L_0(x) -1 L_1(x) + 2 L_2(x) \\
& = -2 \frac{(x-0)(x-1)}{(-1-0)(-1-1)}
  -1 \frac{(x+1)(x-1)}{(0+1)(0-1)}
  +2 \frac{(x+1)(x-0)}{(1+1)(1-0)} \\
& = -2 \left(\frac{x^2-x}{2}\right)
  -1 (1-x^2)
  +2 \left(\frac{x^2+x}{2}\right) \\
&  = x^2+2x-1.
\end{align*}
Logo,
\[
f(x) = (x^2+2x-1) + \frac{f^{(3)}(\xi(x))}{3!}(x+1)x(x-1).
\]
Como $f^\prime(x) = 4x^3+2$, $f^{\prime\prime}(x) = 12x^2$ e $f^{(3)}(x) = 24x$, tem-se em particular que
\[
\varepsilon_{abs}(x)
= \left|f(x) - (x^2+2x-1)\right|
\leq \frac{M}{3!} \left|(x+1)x(x-1)\right|,
\]
em que $M
= \max_{x \in [-1,1]} \left|f^{(3)}(x)\right|
= \max_{x \in [-1,1]} \left|24x\right|
= 24$. Além disso, se $q(x) = (x+1)x(x-1) = x^3-x$ então $q^\prime(x) = 3x^2-1 = 0$ se, e somente se, $x = \pm\frac{\sqrt{3}}{3}$, o que significa que os valores máximo e mínimo de $q(x)$ em $[-1,1]$ ocorrem em um dos pontos $\{ \pm1, \pm\frac{\sqrt{3}}{3} \}$. Como $q(\pm1) = 0$ e $q(\pm\frac{\sqrt{3}}{3}) \approx \mp0,3849$, e $q^{\prime\prime}(\pm\frac{\sqrt{3}}{3}) \pm 2\sqrt{3}$, conclui-se que $-\frac{\sqrt{3}}{3}$ é um ponto de máximo, $\frac{\sqrt{3}}{3}$ é um ponto de mínimo, e o valor máximo de $|q(x)|$ é $0,3849$. Portanto,
\[
\varepsilon_{abs}(x)
\leq \frac{24}{6} \left|x^3 - x\right|
\leq 4 \cdot \max_{x \in [-1,1]} \left|x^3 - x\right|
= 4 \cdot 0,3849
= 1,5396.
\]
\end{ExerciseList}

\vspace{0.5cm}
\begin{center}
BOA PROVA!
\end{center}

\newpage
\restoregeometry
\section*{Respostas}
\shipoutAnswer
\end{document}
