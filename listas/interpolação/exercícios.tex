\documentclass[12pt,a4paper]{article}
\usepackage{cmap} % Makes the PDF copyable. See http://tex.stackexchange.com/a/64198/25761
\usepackage[T1]{fontenc}
\usepackage[brazil]{babel}
\usepackage[utf8]{inputenc}
\usepackage{amsmath}
\usepackage{amsfonts}
\usepackage{amssymb}
\usepackage{amsthm}
\usepackage{textcomp} % \degree
\usepackage{gensymb} % \degree
\usepackage[usenames,svgnames,dvipsnames]{xcolor}
\usepackage{hyperref}
\usepackage{multicol}
\usepackage{graphicx}
\usepackage[margin=2cm]{geometry}
\usepackage{systeme}
\usepackage{icomma} % vírgulas como pontuação vs ponto decimal
\hypersetup{
    colorlinks = true,
    allcolors = {blue}
}

\newcommand{\fixme}{{\color{red}(...)}}
\newcommand*\sen{\operatorname{sen}}

\newcommand*\R{\mathbb{R}}

\newcommand{\IconPc}{\includegraphics[width=1em]{computer.png}}
\newcommand{\IconCalc}{\includegraphics[width=1em]{calculator.png}}
\newcommand{\IconThink}{\includegraphics[width=1em]{pencil.png}}
\newcommand{\IconCheck}{\includegraphics[width=1em]{checkmark.png}}
\newcommand{\IconConcept}{\includegraphics[width=1em]{edit.png}}

\newlength{\SmileysLength}
\setlength{\SmileysLength}{\labelwidth}\addtolength{\SmileysLength}{\labelsep}

\newcommand{\calc}{\hspace*{-\SmileysLength}\makebox[0pt][r]{\IconCalc}%
   \hspace*{\SmileysLength}}
\newcommand{\software}{\hspace*{-\SmileysLength}\makebox[0pt][r]{\IconPc}%
   \hspace*{\SmileysLength}}
\newcommand{\teoria}{\hspace*{-\SmileysLength}\makebox[0pt][r]{\IconThink}%
   \hspace*{\SmileysLength}}
\newcommand{\conceito}{\hspace*{-\SmileysLength}\makebox[0pt][r]{\IconCheck}%
   \hspace*{\SmileysLength}}
\newcommand{\concept}{\hspace*{-\SmileysLength}\makebox[0pt][r]{\IconCheck}%
   \hspace*{\SmileysLength}}

\newcommand*\tipo{Lista de Exercícios - Interpolação}
%\newcommand*\turma{...}
\newcommand*\disciplina{ANN0001/CAN0001}
\newcommand*\eu{Helder G. G. de Lima}
\newcommand*\data{\today}

\author{\eu}
\title{\tipo}
\date{\data}

\begin{document}

\begin{center}
\includegraphics[width=9.0cm]{marca} \\
\textbf{\tipo} \\
Prof. \eu\footnote{
Este é um material de acesso livre distribuído sob os termos da licença \href{https://creativecommons.org/licenses/by-sa/4.0/deed.pt_BR}{Creative Commons BY-SA 4.0}.}
\end{center}

%\section*{Legenda}
%\begin{multicols}{4}
%\begin{itemize}
%\item[] \hspace*{\SmileysLength} \calc \hspace*{-\SmileysLength} Cálculos
%\item[] \hspace*{\SmileysLength} \conceito \hspace*{-\SmileysLength} Conceitos
%\item[] \hspace*{\SmileysLength} \teoria \hspace*{-\SmileysLength} Teoria
%\item[] \hspace*{\SmileysLength} \software \hspace*{-\SmileysLength} Software
%\end{itemize}
%\end{multicols}

\section*{Questões}

\begin{enumerate}
\item Suponha que uma função $f(x)$ assuma os valores dados pela tabela a seguir:
%f(x) = x ln(x) - 3x
\begin{center}
\begin{tabular}{|c|c|c|c|c|c|c|}
\hline
   $x$ &  5    & 10     & 15 & 20 & 25 & 30 \\
\hline
$f(x)$ & -6.95 & -6.97  & -4.38 & -0.09 & 5.47 & 12.04\\
\hline
\end{tabular}
\end{center}

Obtenha o polinômio que interpola os pontos dados, usando os métodos de Lagrange e de Newton. Utilize o polinômio obtido para estimar $f(16)$.
% p(x) = -8.26667×10^-7 x^5 + 0.000094 x^4 - 0.00457 x^3 + 0.13985 x^2 - 1.46223 x - 2.62
% p(16) = -3.63924

\item Obtenha os polinômios de Lagrange $L_i(x)$ correspondentes aos pontos $x_0 = -1$, $x_1 = 3$ e $x_2 = 7$. Mostre que estes polinômios são dois a dois ortogonais (isto é $\langle L_i, L_j \rangle = 0$ para $i \neq j$) com relação ao seguinte produto interno:
\[
\langle f, g \rangle = f(x_0) g(x_0) + f(x_1) g(x_1) + f(x_2) g(x_2).
\]
\item Determine o erro relativo percentual ao aproximar $\sqrt{10}$ através de uma interpolação nos pontos $x_0 = 1$, $x_1 = 4$ e $x_2 = 16$. Compare os gráficos de $f(x) = \sqrt{x}$ e do polinômio interpolador no intervalo $[1,16]$.

\item Alguma das parábolas que possuem eixo de simetria vertical e que contém os pontos $(2, -16)$ e $(5, 5)$ passa pela origem? Em caso afirmativo, qual é esta parábola?
%f(x) = 3x² - 14x
\item Dê um exemplo de uma função polinomial cujo gráfico intercepta o da função $f(x) = 2^x$ pelo menos quatro vezes no intervalo $(-2,3)$.
\item É possível representar uma função que assume os valores a seguir por um polinômio de terceiro grau? Em caso afirmativo, determine esse polinômio usando a forma de Newton.
\begin{center}
\begin{tabular}{|c|c|c|c|c|c|c|}
\hline
   $x$ & -2  & -1 & 0 & 1 &  2 & 3 \\
\hline
$f(x)$ & -39 & -4 & 1 & 0 & 17 & 76\\
\hline
\end{tabular}
\end{center}
%f(x) = 4x³ - 3x² - 2x + 1

\item Seja $f(x) = x^3 - 3x$ e considere os pontos $x_0 = -2$, $x_1 = -1$ e $x_2 = 0$.
\begin{enumerate}
\item Obtenha o polinômio $p(x)$ que interpola $f(x)$ nos pontos dados.
\item Mostre que o erro absoluto da aproximação $f(x) \approx p(x)$, estimado por meio de uma derivada de ordem superior de $f$, coincide com o valor exato do erro.
\item Calcule o erro máximo da aproximação $f(x)\approx p(x)$ no intervalo $[x_0, x_2]$.
\end{enumerate}

\item Seja $p(x)$ o polinômio que interpola os pontos $(-1, 0)$, $(0, 1)$, $(1, y)$, $(3, -20)$. Encontre o valor de $y$ sabendo que o coeficiente de $x^3$ em $p(x)$ é -2.

\item Seja $p:\R \to \R$ o polinômio que interpola os pontos $(x_0, y_0), \ldots, (x_n, y_n)$. Mostre que, para todo $a \in \R$, são verdadeiras as afirmações a seguir:
\begin{enumerate}
\item A translação horizontal $q:\R \to \R$ do gráfico de $p$, definida por por $q(x) = p(x - a)$, para todo $x \in \R$, interpola os pontos $(x_0+a, y_0), \ldots, (x_n+a, y_n)$.
\item A mudança de escala horizontal $q:\R \to \R$ do gráfico de $p$, definida por por $q(x) = p(a \cdot x)$, para todo $x \in \R$, interpola os pontos $(\frac{x_0}{a}, y_0), \ldots, (\frac{x_n}{a}, y_n)$.
\end{enumerate}


\item Um spline cúbico natural interpolador aos nós $x_0 = 3$, $x_1 = 5$ e $x_2 = 7$ é dado por
\[
   s(x) =
   \begin{cases}
      4 - 6x + \frac{x^3}{4}, & \text{ se } 0 \leq x \leq 2,\\
      A + B(x - 2) + C(x - 2)^2 + D(x - 2)^3, & \text{ se } 2 \leq x \leq 7.
   \end{cases}
\]
Utilize as propriedades de continuidade, suavidade e naturalidade do spline e encontre os valores de $A$, $B$, $C$ e $D$.

\item Sejam $f(x) = 2^x$ e considere os pontos $x_0 = 0$, $x_1 = 1$ e $x_2 = 2$. Interpole $f$ nos pontos dados por um spline:
\begin{multicols}{3}
\begin{enumerate}
\item De ordem 0
\item De ordem 1
\item Natural, de ordem 3
\end{enumerate}
\end{multicols}
\item Construa um spline cúbico natural que passe pelos pontos $(0, 3)$, $(2, 0)$, $(3, -1)$ e $(4, 0)$.

\item Utilize as raízes do polinômio de Chebyshev de grau dois, e (se necessário) uma transformação dos intervalos, para construir um polinômio interpolador de grau menor ou igual a um para as seguintes funções nos intervalos indicados:
\begin{multicols}{2}
\begin{enumerate}
   \item $f(x) = 2^x$ no intervalo $[-1, 1]$.
   \item $f(x) = 2^x$ no intervalo $[0, 3]$.
\end{enumerate}
\end{multicols}

\item Repita o exercício anterior com as raízes do polinômio de Chebyshev de grau três, para obter um polinômio interpolador de grau menor ou igual a dois.
\end{enumerate}


\newpage
\section*{Respostas}
\begin{enumerate}
\item Conforme o Wolfram Alpha, o polinômio interpolador é:
\[
p(x) = -8,\!26667 \times 10^{-7} x^5 + 0,\!000094 x^4 - 0,\!00457 x^3 + 0,\!13985 x^2 - 1,\!46223 x - 2,\!62
\]
e $f(16) \approx p(16) \approx -3,\!639$.
%\begin{enumerate}
%\item Lagrange: \fixme
%\item Newton: \fixme
%\end{enumerate}
\item \fixme
\item $p(x) = -\frac{1}{90}x^2 + \frac{7}{18} x + \frac{28}{45}$, $\sqrt{10} \approx p(10) = \frac{17}{5} = 3,\!4$ e $\varepsilon \approx 7,\!5\%$.
\item Sim, $p(x) = 3x^2 - 14x$.
\item $p(x) = \frac{1}{12}x^3 + \frac{1}{4}x^2 + \frac{2}{3} x + 1$
\item Sim, $f(x) = 4x^3 - 3x^2 - 2x + 1$.
\item
\begin{enumerate}
\item $p(x) = -3x^2 - 5x$
\item O erro absoluto no ponto $x$ é dado por
\[
|f(x) - p(x)| = |x^3 + 3x^2 + 2x| = |(x + 2) (x + 1) x|.
\]

\item O valor máximo de $|f(x) - p(x)|$ no intervalo $[-2, 0]$ é de $\frac{2}{3 \sqrt{3}}$ e ocorre no ponto $x = -1 - \frac{1}{\sqrt{3}}$.
\end{enumerate}
\item $y=6$
\item
\begin{enumerate}
\item Basta observar que $q(x_i + a, y_i) = p((x_i + a)-a)) = p(x_i) = y_i$, para $i = 0, \ldots, n$.
\item Basta observar que $q(\frac{x_i}{a}, y_i) = p(a \cdot \frac{x_i}{a}) = p(x_i) = y_i$, para $i = 0, \ldots, n$.
\end{enumerate}

\item $A = -6$, $B = -3$, $C = \frac{3}{2}$ e $D = -\frac{1}{10}$
\item
\begin{enumerate}
\item Utilizando funções constantes em cada subintervalo, o spline é uma função degrau:
\[
s(x) =
\begin{cases}
1, & \text{ se } 0\leq x < 1, \\
2, & \text{ se } 1\leq x < 2, \\
4, & \text{ se } x = 2.
\end{cases}
\]
\textit{Observação}: note que há descontinuidade nos nós.

\item Um spline interpolador de ordem 1 é uma função contínua cuja restrição a cada subintervalo é uma função afim. Neste caso, aplicando a fórmula de Lagrange para interpolar linearmente os valores de $f$ em cada subintervalo, obtém-se:
\[
s(x) =
\begin{cases}
2^0\dfrac{x-1}{0-1} + 2^1\dfrac{x-0}{1-0}, & \text{ se } 0\leq x < 1, \\ \\
2^1\dfrac{x-2}{1-2} + 2^2\dfrac{x-1}{2-1}, & \text{ se } 1\leq x \leq 2,
\end{cases}
\]
ou seja,
\[
s(x) =
\begin{cases}
x + 1, & \text{ se } 0\leq x < 1, \\
2x, & \text{ se } 1\leq x \leq 2.
\end{cases}
\]
\textit{Observação}: note que embora $s$ seja contínua no nó central, sua derivada não está definida nesse ponto, pois as inclinações são diferentes à esquerda e à direita do nó.
\item Resolvendo \href{https://www.ufrgs.br/reamat/CalculoNumerico/livro-py/i1-interpolacao_cubica_segmentada_-_spline.html#x71-1160006.6.1}{um sistema linear}, obtém-se os valores que a derivada do spline natural $s(x)$ assume em cada nó, e a partir desses valores, podem ser calculados os coeficientes das funções cúbicas $s_0$ e $s_1$ que definem o spline nos subintervalos $[x_0, x_1)$ e $[x_1, x_2]$:
\[
s(x) =
\begin{cases}
1 + \frac{3}{4} x + \frac{1}{4}x^3,               & \text{ se } 0\leq x < 1, \\
2 + \frac{3}{2} (x - 1) + \frac{3}{4}(x - 1)^2 - \frac{1}{4}(x - 1)^3, & \text{ se } 1\leq x \leq 2.
\end{cases}
\]
\textit{Observação}: note que há continuidade tanto de $s$ (pois $s_0(1) = 2 = s_1(1)$), quanto de sua primeira derivada ($s_1^{\prime}(1) = \frac{3}{2} = s_2^{\prime}(1)$) e de sua segunda derivada (pois $s_1^{\prime\prime}(1) = \frac{3}{2} = s_2^{\prime\prime}(1)$). Além disso, $s$ realmente é natural, já que $s^{\prime\prime}(0) = 0 = s^{\prime\prime}(2)$.
\end{enumerate}
\item O spline cúbico natural $s$ é definido por:
\[
s(x) =
\begin{cases}
3 - \frac{3}{2}x,               & \text{ se } 0\leq x < 2, \\
-\frac{3}{2} (x - 2) + \frac{1}{2}(x - 2)^3, & \text{ se } 2\leq x < 3, \\
-1 + \frac{3}{2}(x - 3)^2 - \frac{1}{2}(x - 3)^3, & \text{ se } 3\leq x \leq 4.
\end{cases}
\]



\item \begin{enumerate}
   \item O polinômio de Chebyshev de grau dois é dado por $T_2(x) = \cos(2\arccos(x)) = 2 x^2 - 1$ para $x \in [-1, 1]$, e suas raízes são $x_1 = \cos\left(\frac{1}{4}\pi\right) = \frac{\sqrt{2}}{2} \approx 0,7071$ e $x_2 = \cos\left(\frac{3}{4}\pi\right) = -\frac{\sqrt{2}}{2} \approx -0,7071$. Por Lagrange, o polinômio que interpola $f(x) = 2^x$ nesses pontos é dado por:
   \begin{align*}
      p(x)
      & = 2^{0,7071} \left(\frac{x+0,7071}{0,7071+0,7071}\right)
      + 2^{-0,7071} \left(\frac{x-0,7071}{-0,7071-0,7071}\right)\\
      & = 1,1544 (x + 0,7071) - 0,4331 (x - 0,7071) \\
      & = 0,7213 x + 1,1225
   \end{align*}
   \item Como no item anterior, as raízes do polinômio de Chebyshev de grau dois são $x_1 \approx 0,7071$ e $x_2 \approx -0,7071$. Transformando o intervalo $[-1, 1]$ em $[0, 3]$ por meio de
   \[
   \tilde{x} = \frac{1}{2}[(3 - 0) x + (0 + 3)] = \frac{3}{2}(x + 1),
   \]
   obtém-se $\tilde{x}_1 \approx 2,5607$ e $\tilde{x}_2 \approx 0,4394$. Por Lagrange, o polinômio que interpola $f(x) = 2^x$ nesses pontos é dado por:
   \begin{align*}
      p(x)
      & = 2^{2,5607} \left(\frac{x - 0,4394}{2,5607 - 0,4394}\right)
      + 2^{0,4394} \left(\frac{x - 2,5607}{0,4394 - 2,5607}\right)\\
      & = 2,7813 (x - 0,4394) - 0,6392 (x - 2,5607) \\
      & = 2,1421 x + 0,4147
   \end{align*}
\end{enumerate}

\item \begin{enumerate}
   \item O polinômio de Chebyshev de grau três é dado por $T_3(x) = \cos(3\arccos(x)) = 4 x^3 - 3x$ para $x \in [-1, 1]$, e suas raízes são $x_1 = \cos\left(\frac{1}{6}\pi\right) = \frac{\sqrt{3}}{2} \approx 0,8660$ e $x_2 = \cos\left(\frac{3}{6}\pi\right) = 0$ e $x_3 = \cos\left(\frac{5}{6}\pi\right) = - \frac{\sqrt{3}}{2} \approx -0,8660$. O polinômio que interpola $f(x) = 2^x$ nesses pontos é dado por:
   \begin{align*}
      p(x)
      = 0,2475x^2 + 0,7355x + 1.
   \end{align*}
   \item Como no item anterior, as raízes do polinômio de Chebyshev de grau dois são $x_1 \approx 0,8660$ e $x_2 = 0$ e $x_3 \approx -0,8660$. Transformando o intervalo $[-1, 1]$ em $[0, 3]$ por meio de
   \[
   \tilde{x} = \frac{1}{2}[(3 - 0) x + (0 + 3)] = \frac{3}{2}(x + 1),
   \]
   obtém-se $\tilde{x}_1 \approx 2,7990$, $\tilde{x}_2 = 1,5$ e $\tilde{x}_3 \approx 1,1495$. O polinômio que interpola $f(x) = 2^x$ nesses pontos é dado por:
   \begin{align*}
      p(x)
      & = 0,7266x^2 + 0,0565x + 1,1088.
   \end{align*}
\end{enumerate}
\end{enumerate}
\end{document}
