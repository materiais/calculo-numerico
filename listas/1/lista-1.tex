\documentclass[12pt,a4paper]{article}
\usepackage{cmap} % Makes the PDF copyable. See http://tex.stackexchange.com/a/64198/25761
\usepackage[T1]{fontenc}
\usepackage[brazil]{babel}
\usepackage[utf8]{inputenc}
\usepackage{amsmath}
\usepackage{amsfonts}
\usepackage{amssymb}
\usepackage{amsthm}
\usepackage{textcomp} % \degree
\usepackage{gensymb} % \degree
\usepackage[usenames,svgnames,dvipsnames]{xcolor}
\usepackage{hyperref}
\usepackage{multicol}
\usepackage{graphicx}
\usepackage[margin=2cm]{geometry}
\usepackage{systeme}
\usepackage{icomma} % vírgulas como pontuação vs ponto decimal
\usepackage{listings}
\hypersetup{
    colorlinks = true,
    allcolors = {blue}
}

\newcommand{\fixme}{{\color{red}(...)}}
\newcommand*\sen{\operatorname{sen}}
\newcommand*\tg{\operatorname{tg}}

\newcommand*\R{\mathbb{R}}

\newcommand{\IconPc}{\includegraphics[width=1em]{computer.png}}
\newcommand{\IconCalc}{\includegraphics[width=1em]{calculator.png}}
\newcommand{\IconThink}{\includegraphics[width=1em]{pencil.png}}
\newcommand{\IconCheck}{\includegraphics[width=1em]{checkmark.png}}
\newcommand{\IconConcept}{\includegraphics[width=1em]{edit.png}}

\newlength{\SmileysLength}
\setlength{\SmileysLength}{\labelwidth}\addtolength{\SmileysLength}{\labelsep}

\newcommand{\calc}{\hspace*{-\SmileysLength}\makebox[0pt][r]{\IconCalc}%
   \hspace*{\SmileysLength}}
\newcommand{\software}{\hspace*{-\SmileysLength}\makebox[0pt][r]{\IconPc}%
   \hspace*{\SmileysLength}}
\newcommand{\teoria}{\hspace*{-\SmileysLength}\makebox[0pt][r]{\IconThink}%
   \hspace*{\SmileysLength}}
\newcommand{\conceito}{\hspace*{-\SmileysLength}\makebox[0pt][r]{\IconCheck}%
   \hspace*{\SmileysLength}}
\newcommand{\concept}{\hspace*{-\SmileysLength}\makebox[0pt][r]{\IconCheck}%
   \hspace*{\SmileysLength}}

\newcommand*\tipo{1ª Lista de Exercícios}
%\newcommand*\turma{...}
\newcommand*\disciplina{ANN0001/CAN0001}
\newcommand*\eu{Helder G. G. de Lima}
\newcommand*\data{\today}

\author{\eu}
\title{\tipo - \disciplina}
\date{\data}

\begin{document}

\begin{center}
\includegraphics[width=9.0cm]{marca} \\
\textbf{\tipo\ (\disciplina)} \\
Prof. \eu\footnote{
Este é um material de acesso livre distribuído sob os termos da licença \href{https://creativecommons.org/licenses/by-sa/4.0/deed.pt_BR}{Creative Commons BY-SA 4.0}.}
\end{center}

%\section*{Legenda}
%\begin{multicols}{4}
%\begin{itemize}
%\item[] \hspace*{\SmileysLength} \calc \hspace*{-\SmileysLength} Cálculos
%\item[] \hspace*{\SmileysLength} \conceito \hspace*{-\SmileysLength} Conceitos
%\item[] \hspace*{\SmileysLength} \teoria \hspace*{-\SmileysLength} Teoria
%\item[] \hspace*{\SmileysLength} \software \hspace*{-\SmileysLength} Software
%\end{itemize}
%\end{multicols}

\section*{Questões}

\begin{enumerate}
\item \calc
Obtenha a representação dos números a seguir nas bases indicadas, e complete a tabela:
\begin{table}[h]
\centering
%\caption{Conversão entre bases}
\begin{tabular}{|c|c|c|}
\hline
  \textbf{Decimal}
& \textbf{Binário}
& \textbf{Octal} \\ \hline
  $(33,25)_{10}$
& $(100001,01)_2$
& $(41,2)_8$ \\ \hline
?
& $(110,1)_2$
& ? \\ \hline
  $(110,1)_{10}$
& ?
& ? \\ \hline
  ?
& ?
& $(110,1)_8$ \\ \hline
  $(501,375)_{10}$
& ?
& ? \\ \hline
  ?
& ?
& $(35,1)_8$ \\ \hline
  ?
& $(1011,101)_2$
& ? \\ \hline
\end{tabular}
\end{table}
\vspace{-0.8cm}
\item \software Devido à forma como os números são representados no computador, pode ocorrer que dois números reais distintos sejam representados internamente pelo mesmo número, sendo arredondados. Utilize uma linguagem de programação de sua escolha para estimar o maior valor de $\varepsilon$ tal que $1+\varepsilon$ seja arredondado para $1$ através do seguinte algoritmo:
\begin{lstlisting}[mathescape=true]
Defina $\varepsilon = 1$
Enquanto $1 \neq (1 + \varepsilon)$:
    Defina $ \varepsilon = \varepsilon / 2$
Retorne $\varepsilon$
\end{lstlisting}
\vspace{-0.5cm}
\item \software Utilize uma linguagem de programação de sua escolha para identificar qual é a primeira potência de dois cuja representação no computador é idêntica à de seu sucessor. Em outras palavras, encontre o menor valor de $n \in \mathbb{N}$ tal que $(2^n) + 1$ é arredondado para $2^n$.

\item \calc Calcule o erro relativo percentual cometido ao arredondar os seguintes números (conforme as regras da ABNT), conservando apenas 2 algarismos após a vírgula:
\begin{multicols}{5}
\begin{enumerate}
\item $3,999$
\item $\pi$
\item $e$
\item $\frac{37}{8}$
\item $\sqrt{3}$
\item $\sqrt{2}$
\item $0,975$
\item $0,1357$
\item $4,56 \times 10^{-2}$
\item $8 \times 10^{-3}$
\end{enumerate}
\end{multicols}

\item \calc Sabe-se que determinada quantidade $\overline{x}$ só assume valores tais que $0,1 \leq \overline{x} \leq 1,25$. Sabendo que ao medir $\overline{x}$ foi obtida uma aproximação $x \approx \overline{x}$, que satisfaz $0,05 \leq x \leq 0,6$, responda:
\begin{enumerate}
\item Qual é o maior erro absoluto $\varepsilon_{abs} = |\overline{x}-x|$ que pode ter ocorrido? Para quais valores de $\overline{x}$ e de $x$ ocorreria esse erro?
\item Qual é o maior erro relativo $\varepsilon_{rel} = \frac{|\overline{x}-x| }{ |\overline{x}|}$ que pode ter ocorrido? Para quais valores de $\overline{x}$ e de $x$ ocorreria esse erro?
\end{enumerate}

\item \teoria Use o teorema de Bolzano para mostrar que o erro absoluto cometido ao tomar $\tilde{x} = 0,7$ como uma aproximação do zero da função $f(x) = \ln(x) + 3x^6$ é menor que $10^{-3}$.

\item \calc Se um número natural em binário tem 34 dígitos (bits), qual é o mínimo e o máximo de dígitos decimais que ele pode ter?

\item \calc A seguir são apresentadas algumas sequências $\{ x_n \}_{n \in \mathbb{N}}$ que convergem para $\overline{x} = 2$. Determine, em cada caso, qual é o primeiro termo $x_k$ que pode ser tomado como aproximação de $\overline{x}$, para que o erro percentual relativo cometido nesta aproximação seja inferior a $1\%$.
\begin{multicols}{2}
\begin{enumerate}
\item $x_k =
\begin{cases}
\dfrac{{(x_{k-1})}^2 + 4}{2x_{k-1}},& \text{ se } k \geq 2\\
1, & \text{ se } k = 1
\end{cases}$
\item $x_k = 2 + \dfrac{ (-1)^k }{ k^2 }$, para todo $k \geq 1$
\end{enumerate}
\end{multicols}

\item \calc Seja $\{ x_n \}_{n \in \mathbb{N}}$, a sequência definida por $x_n = \dfrac{2 + \cos(\pi n)}{2\pi n}$, para $n\geq 1$. Verifique que, apesar de ser verdade que $x_n \to 0$, é possível obter $|x_n| < 0.1$ sem que ocorra $|x_{n+1}| < 0.1$. O que isso revela sobre o uso de testes como $|\varepsilon_{\text{abs}}| < \text{TOL}$ como critério de parada?

\item \calc Seja $f(x) = (x^2 - 4)(x - 8)$. Determine o menor valor de $n$ tal que a aproximação $x_k$ obtida pelo método da bisseção tenha um erro relativo percentual menor do que 1\% sendo:
\begin{multicols}{2}
\begin{enumerate}
\item $[a_0, b_0] = [1, 4]$
\item $[a_0, b_0] = [6, 9]$
\end{enumerate}
\end{multicols}

\item \calc Encontre o (único) ponto de mínimo da função $f(x) = \cos(x + 1) + \dfrac{x^2}{2}$.
\item \calc
Seja $f(x) = \tg(x) - \sqrt{1 - x^2}$.
\begin{enumerate}
\item Determine, sem realizar as iterações, para quais valores de $n$ o método da bisseção produz uma aproximação $x_n = \frac{a_n+b_n}{2}$ da raiz de $f(x)$, com um erro absoluto menor do que $0.05$, considerando que seja utilizado o intervalo inicial $[a_0, b_0] = [-1, 1]$. Lembre-se que o erro absoluto de $x_k$ é menor do que $|b_k - a_k|/2$.
\item Utilize o método da bisseção para obter um $x_n$ que satisfaça a condição anterior.
\item Execute duas iterações do método de Newton-Raphson usando o valor obtido no item anterior como aproximação inicial.
\item Compare o valor obtido pelo método de Newton-Raphson com os valores de $x_i$ produzidos pelo método da bisseção. Qual foi a primeira iteração em que o erro absoluto foi, efetivamente, menor do que $0.5$?
\end{enumerate}

\item \calc Sejam $f(x) = 1 - \sqrt{x^2 - 1}$ e $\varphi(x) = x + 1 - \sqrt{x^2 - 1}$.
\begin{enumerate}
\item Mostre que os zeros de $f$ são exatamente os pontos-fixos de $\varphi$.
\item Identifique o conjunto $I_1$ dos pontos em que $f$, $\varphi$ e $\varphi^\prime$ são contínuas.
\item Use argumentos teóricos para mostrar que $f$ tem apenas dois zeros: um no intervalo $I_2 = (-1,5, -1,3)$ e outro em $I_3 = (1,4, 1,6)$.
\item Encontre o conjunto $I_4$ formado por todos os valores de $x \in \R$ tais que $|\varphi^\prime(x)| < 1$.
\item Considerando as respostas dos itens anteriores, pode-se afirmar que cada raiz $\overline{x}$ de $f$ é o centro de algum intervalo $I = (\overline{x}-\delta, \overline{x}+\delta)$, no qual qualquer aproximação $x_0 \in I$ produzirá uma sequência convergente, se for usada no método do ponto fixo?
\item Se for possível, aplique o método do ponto fixo com $x_0 = 1,25$ para obter um dos zeros de $f$ com erro relativo percentual (estimado) inferior a $0,1\%$.
\item Comente sobre a possibilidade de utilizar $x_0 = 17/15$ como aproximação inicial.
\end{enumerate}

\item \calc Seja $h: \R \to \R$ definida por $h(x) = \cos(e^x)$.
\begin{enumerate}
\item Mostre que $h$ possui pelo menos uma raiz $\overline{x}$ no intervalo $I = [0, 1]$.
\item Obtenha $x_{k} \approx \overline{x}$ pelo método da posição falsa, com um erro (estimado) de até $0.1\%$.
\item Se forem usados os pontos $x_{-1} = 0$ e $x_0 = 1$ como aproximações iniciais no método da secante, por coincidência, as primeiras aproximações serão idênticas àquelas encontradas pela posição falsa. Qual é a primeira aproximação que difere das que foram obtidas pela posição falsa?
\end{enumerate}
\item \teoria Uma função $\varphi: [a, b] \to [a, b]$ é uma \textit{contração} se existe $\beta \in (0, 1)$ tal que
\[
  |\varphi(x) - \varphi(y)| \leq \beta |x-y|,
\]
para quaisquer $x,y \in [a, b]$. Mostre que se $\varphi: [a, b] \to [a, b]$ é uma contração então:
\begin{enumerate}
  \item $\varphi$ é uma função contínua.
  \item $\varphi$ é não é sobrejetora.
\end{enumerate}
\end{enumerate}

\newpage
\section*{Respostas}
\begin{enumerate}
\item Estas são as representações obtidas nas bases indicadas:
\begin{table}[h]
\centering
\caption{Conversão entre bases}
\begin{tabular}{|c|c|c|}
\hline
  \textbf{Decimal}
& \textbf{Binário}
& \textbf{Octal} \\ \hline
  $(33,25)_{10}$
& $(100001,01)_2$
& $(41,2)_8$ \\ \hline
  $(6,5)_{10}$
& $(110,1)_2$
& $(6,4)_8$ \\ \hline
  $(110,1)_{10}$
& $(1101110.0\overline{0011}...)_2$
& $(156.0\overline{6314})_8$ \\ \hline
  $(72,125)_{10}$
& $(1001000,001)_2$
& $(110,1)_8$ \\ \hline
  $(501,375)_{10}$
& $(111110101,011)_2$
& $(765,3)_8$ \\ \hline
  $(29,125)_{10}$
& $(11101,001)_2$
& $(35,1_8)_8$ \\ \hline
  $(11,625)_{10}$
& $(1011,101)_2$
& $(13,5)_8$ \\ \hline
\end{tabular}
\end{table}

\item Ao testar com o Scilab, Python ou JavaScript, obtém-se $\varepsilon \approx 1.1102230246251565 \times 10 ^{-16}$.

\item Ao testar com o Scilab, ou em JavaScript, obtém-se $n=53$, pois a representação interna de $2^{53} +1$ é a mesma de $2^{53}$.

\item
\begin{multicols}{5}
\begin{enumerate}
\item $0,03\%$
\item $0,05\%$
\item $0,03\%$
\item $0,11\%$
\item $0,12\%$
\item $0,30\%$
\item $0,51\%$
\item $3,17\%$
\item $9,65\%$
\item $25,00\%$
\end{enumerate}
\end{multicols}

\item
\begin{enumerate}
\item O maior erro absoluto possível nesta situação é $\varepsilon_{abs} = |1,25 - 0,05| = 1,2$. Isso pode ser obtido calculando o máximo absoluto da função $f(\overline{x}, x) = |\overline{x} - x|$ sobre o retângulo $D = \{ (\overline{x}, x) \in \R^2 \ |\ 0,1 \leq \overline{x} \leq 1,25 \text{ e } 0,05 \leq x \leq 0,6 \}$.

\item O maior erro relativo possível nesta situação é $\varepsilon_{rel} = \frac{|0,6 - 0,1| }{ |0,1|} = 5$. Isso pode ser obtido usando ferramentas do cálculo diferencial para calcular o máximo absoluto da função $g(\overline{x}, x) = \frac{|\overline{x}-x| }{ |\overline{x}|}$ sobre o retângulo $D = \{ (\overline{x}, x) \in \R^2 \ |\ 0,1 \leq \overline{x} \leq 1,25 \text{ e } 0,05 \leq x \leq 0,6 \}$. Observe que, para um valor de $x$ fixado, o erro relativo aumenta conforme o valor de $\overline{x}$ se aproxima de zero.
\end{enumerate}

\item Basta observar que o intervalo $[a,b] = [0,7-10^{-3}, 0,7+10^{-3}]$ satisfaz as hipóteses do teorema de Bolzano, pois $f(0,699) \approx -0.008 < 0$ e $f(0,701) \approx 0.001 > 0$. Logo, há uma raiz $\overline{x} \in (0,699, 0,701)$, cuja distância até o centro do intervalo é $d = |\overline{x} - 0,7| < 10^{-3}$.
\item Considerando que $2^{33} = 8589934592 \leq n \leq 17179869183 = 2^{34}-1$, resulta que $n$ tem de $10$ a $11$ dígitos.
\item Considerando que $\overline{x} = 2$, têm-se as seguintes equivalências a respeito da aproximação $x_k$:
\[
\varepsilon_{per}(x_k) < 1\%
\Leftrightarrow
\varepsilon_{rel}(x_k) < 0,01
\Leftrightarrow
\frac{|x_k - 2|}{|2|} < 0,01
\Leftrightarrow
|x_k - 2| < 0,02
\Leftrightarrow
1,98 < x_k < 2,02
\]

As tabelas a seguir mostra os primeiros termos de cada sequência, e os respectivos erros, arredondados para o quarto dígito após a vírgula:
\begin{enumerate}
\item
\begin{tabular}{|c|c|c|c|c|}
\hline
$k$ & $x_k$ & $\varepsilon_{abs}$ & $\varepsilon_{rel}$ & $\varepsilon_{per}$ \\
\hline
$1$ & $2,5000$ & $0,5000$ & $0,2500$ & $25,00\%$ \\
\hline
$2$ & $2,0500$ & $0,0500$ & $0,0250$ & $2,50\%$ \\
\hline
$3$ & $2,0006$ & $0,0006$ & $0,0003$ & $0,03\%$ \\
\hline
\end{tabular}

Portanto o primeiro termo que aproxima $\overline{x}$ com menos de $1\%$ de erro é $x_3 = 2,0006$.
\item
\begin{tabular}{|c|c|c|c|c|}
\hline
$k$ & $x_k$ & $\varepsilon_{abs}$ & $\varepsilon_{rel}$ & $\varepsilon_{per}$ \\
\hline
$1$ & $1,0000$ & $1,0000$ & $0,5000$ & $50,00\%$ \\
\hline
$2$ & $2,2500$ & $0,2500$ & $0,1250$ & $12,5\%$ \\
\hline
$3$ & $1,8889$ & $0,8889$ & $0,0556$ & $5,56\%$ \\
\hline
$4$ & $2,0625$ & $0,0625$ & $0,0313$ & $3,13\%$ \\
\hline
$5$ & $1,9600$ & $0,9600$ & $0,0200$ & $2,00\%$ \\
\hline
$6$ & $2,0278$ & $0,0278$ & $0,0139$ & $1,39\%$ \\
\hline
$7$ & $2,9796$ & $0,9796$ & $0,0102$ & $1,02\%$ \\
\hline
$8$ & $2,0156$ & $0,0156$ & $0,0078$ & $0,78\%$ \\
\hline
\end{tabular}

Portanto o primeiro termo que aproxima $\overline{x}$ com menos de $1\%$ de erro é $x_8 = 2,0156$.
\end{enumerate}
\item Os primeiros termos da sequência são os seguintes:
\begin{center}
\begin{tabular}{|c|c|c|c|c|}
\hline
$k$   & $1$   & $2$   & $3$   & $4$\\
\hline
$x_k$ & $0,159$ & $0,239$ & $0,053$ & $0,119$ \\
\hline
\end{tabular}
\end{center}
Como se pode observar, $|x_3| < 0,1$ embora $|x_4| > 0,1$. Assim, o fato de um certo termo de uma sequência convergente estar muito próximo do limite não garante que todos os termos seguintes estarão igualmente (ou mais) próximos do limite. Assim, mesmo que o critério de parada seja satisfeito, não há garantia de que as aproximações subsequentes seriam ainda melhores.
\item
\begin{enumerate}
\item Considerando que $2$ é a única raiz de $f(x)$ em $[1, 4]$, pode-se utilizar $\overline{x} = 2$ ao calcular o erro relativo percentual a cada iteração do método da bisseção. Estes são os valores obtidos:

\begin{tabular}{cccccccc}
\hline
$k$ & $a_k$ & $x_k$ & $b_k$ & $f(a_k)$ & $f(x_k)$ & $f(b_k)$ & $\varepsilon_{rel}(x_k)$ \\
\hline
0 & 1,0000 & 2,5000 & 4,0000 & 21,0000 & -12,3750 & -48,0000 & 25,0000\% \\
1 & 1,0000 & 1,7500 & 2,5000 & 21,0000 &   5,8594 & -12,3750 & 12,5000\%\\
2 & 1,7500 & 2,1250 & 2,5000 &  5,8594 &  -3,0293 & -12,3750 &  6,2500\%\\
3 & 1,7500 & 1,9375 & 2,1250 &  5,8594 &   1,4919 &  -3,0293 &  3,1250\%\\
4 & 1,9375 & 2,0313 & 2,1250 &  1,4919 &  -0,7531 &  -3,0293 &  1,5650\%\\
5 & 1,9375 & 1,9844 & 2,0313 &  1,4919 &   0,3739 &  -0,7531 &  0,7800\%\\
\hline
\end{tabular}

Assim, neste caso $k = 5$.

\textbf{Observação}: Como a raiz exata é conhecida, pode-se deduzir que
\[
\varepsilon_{per}(x_k) < 1\%
\Leftrightarrow
\varepsilon_{rel}(x_k) < 0,01
\Leftrightarrow
\frac{|x_k - 2|}{|2|} < 0,01
\Leftrightarrow
\varepsilon_{abs}(x_k) = |x_k - 2| < 0,02,
\]
e então executar as iterações até que o erro absoluto seja menor do que $0,02$ (o que, de qualquer modo, ocorrerá na quinta iteração). No entanto, se em vez de realizar as iterações for usado que
\[
k > \frac{\log(4-1)-\log(0,02)}{\log(2)} = \log_2 \left(\frac{4-1}{0,02}\right)-1 \approx 6,2
\Rightarrow
k \geq 7
\Rightarrow
\varepsilon_{abs}(x_k) < 0,02
\]
o resultado obtido será maior do que o esperado, pois esta fórmula é deduzida a partir de uma estimativa conservadora do erro absoluto a cada iteração. Mesmo assim, não deixa de ser verdade que, para $k\geq 7$, o erro relativo percentual da aproximação $x_k$ é menor do que $1\%$.

\item Considerando que $\overline{x} = 8$ é a única raiz de $f(x)$ em $[6, 9]$, pode-se utilizar este valor ao calcular o erro relativo percentual a cada iteração do método da bisseção, e obter:

\begin{tabular}{cccccccc}
\hline
$k$ & $a_k$ & $x_k$ & $b_k$ & $f(a_k)$ & $f(x_k)$ & $f(b_k)$ & $\varepsilon_{rel}(x_k)$ \\
\hline
0 & 6,0000 & 7,5000 & 9,0000 & -64,0000 & -26,1250 & 77,0000 & -6,2500\%\\
1 & 7,5000 & 8,2500 & 9,0000 & -26,1250 &  16,0156 & 77,0000 &  3,1250\%\\
2 & 7,5000 & 7,8750 & 8,2500 & -26,1250 &  -7,2520 & 16,0156 & -1,5625\%\\
3 & 7,8750 & 8,0625 & 8,2500 &  -7,2520 &   3,8127 & 16,0156 &  0,7813\%\\
\hline
\end{tabular}

Assim, neste caso $k = 3$.
\end{enumerate}

\item Dica: lembre-se que quando uma função é derivável, deve ocorrer $f^\prime(x) = 0$ em todo $x$ que for um ponto de mínimo local de $f$. Assim, pode-se usar um dos métodos para zeros de funções para obter que $x = 0,9345632107$ é o ponto de mínimo e que $f(x) = 0,0809070571$ é o valor mínimo.
\item Solução com 15 dígitos após a vírgula: $x = 0,649888730553309$.
\item
\begin{enumerate}
\item Basta observar que:
\[
f(x) = 0
\Leftrightarrow
1 - \sqrt{x^2 - 1} = 0
\Leftrightarrow
x + 1 - \sqrt{x^2 - 1} = x
\Leftrightarrow
\varphi(x) = x.
\]
\item As funções $f$, $\varphi$ e $\varphi^\prime$ são contínuas em todos os pontos de seus domínios. As funções $f$ e $\varphi$ têm domínio $(-\infty, -1] \cup [1, +\infty)$, enquanto que $\varphi^\prime(x) = 1 - \frac{x}{\sqrt{x^2 - 1}}$ tem domínio $(-\infty, -1) \cup (1, +\infty)$. Logo, $I_1 = (-\infty, -1) \cup (1, +\infty)$.
\item Basta aplicar o teorema de Bolzano aos intervalos dados e observar que $f$ é crescente em $(-\infty, -1]$ e decrescente em $[1, +\infty)$, pois assim $f$ não pode ter outros zeros.
\end{enumerate}
\item \begin{enumerate}
\item Basta aplicar o teorema de Bolzano, pois $h$ é contínua.
\item A raiz exata é $\overline{x} = \ln(\pi/2) \approx 0.451582705\ldots$.
\end{enumerate}
\item \begin{enumerate}
  \item Dado $c \in [a,b]$, deve-se mostrar que $\varphi$ é contínua em $c$, isto é, que dado $\varepsilon > 0$ arbitrário, existe algum $\delta > 0$ tal que
  \[
    \forall x \in [a,b], \quad |x-c| < \delta \Rightarrow |\varphi(x)-\varphi(c)| < \varepsilon.
  \]
  Para isso, basta considerar $\delta = \varepsilon / \beta$ pois se $|x - c| < \delta$, então
  \[
    |\varphi(x)-\varphi(c)|
    \leq \beta \cdot |x-c|
    < \beta \cdot (\varepsilon / \beta)
    = \varepsilon.
  \]
  \item Se $\varphi$ fosse sobrejetora, todo elemento $y$ do contradomínio $[a, b]$ seria imagem de algum elemento $x$ do domínio $[a,b]$. Em particular, isso valeria para as extremidades do intervalo, de modo que $a = \varphi(x_a)$ e $b = \varphi(x_b)$, para algum $x_a, x_b \in [a,b]$. Neste caso, teríamos uma contradição:
  \[
    |b-a| = | \varphi(x_b) - \varphi(x_a)|
    \leq \beta |b - a|
    < |b - a|.
  \]
  Logo, a contração $\varphi$ não pode ser sobrejetora.
\end{enumerate}
\end{enumerate}
\end{document}
